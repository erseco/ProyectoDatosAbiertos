\documentclass[a4paper,11pt]{book}
\usepackage{listings}
\usepackage[utf8]{inputenc}
\usepackage{titlesec}
\usepackage{fancyhdr}
\usepackage[spanish,es-tabla]{babel}
\usepackage[hidelinks]{hyperref}
\usepackage{xcolor}
\usepackage{pdfpages}
\usepackage{amssymb}
\usepackage{minted}
\usepackage{appendix}


% Información reutilizable
\newcommand{\asunto}{Trabajo de Fin de Máster}
\newcommand{\titulo}{Datos Abiertos en la Universidad}
\newcommand{\tituloEng}{Open Data at University}
\newcommand{\master}{Máster Universitario en Ingeniería Informática}
\newcommand{\keywordsEs}{datos abiertos, web semántica, datos enlazados, CKAN, RDF, OWL, SPARQL}
\newcommand{\keywordsEn}{open data, semantic web, CKAN, RDF, OWL, SPARQL}
\newcommand{\autor}{Germán Martínez Maldonado}
\newcommand{\email}{germaaan@gmail.com}
\newcommand{\tutor}{Juan Julián Merelo Guervós}
\newcommand{\escuela}{Escuela Técnica Superior de Ingenierías Informática y de Telecomunicación}
\newcommand{\universidad}{Universidad de Granada}
\newcommand{\ciudad}{Granada}
\newcommand{\vers}{Versión 0.1}

% Información archivo
\hypersetup{
	pdfauthor = {\autor\ (\email)},
	pdftitle = {\titulo},
	pdfsubject = {\asunto},
	pdfkeywords = {\keywordsEs},
	pdfcreator = {LaTeX con el paquete texlive},
	pdfproducer = {pdflatex}
}

% Estilo de cabeceras
\pagestyle{fancy}
\fancyhf{}
\fancyhead[LO]{\leftmark}
\fancyhead[RE]{\rightmark}
\fancyhead[RO,LE]{\textbf{\thepage}}
\setlength{\headheight}{1.5\headheight}

% Redefinición de comandos
\renewcommand{\listingscaption}{Fragmento de código}
\renewcommand{\chaptermark}[1]{\markboth{\textbf{#1}}{}}
\renewcommand{\sectionmark}[1]{\markright{\textbf{\thesection. #1}}}
\renewcommand{\appendixname}{Anexos}
\renewcommand{\appendixtocname}{Anexos}
\renewcommand{\appendixpagename}{Anexos}
\newcommand\tab[1][0.5cm]{\hspace*{#1}}

% Definición de colores
\definecolor{gray97}{gray}{.97}
\definecolor{gray75}{gray}{.75}
\definecolor{gray45}{gray}{.45}
\definecolor{gray30}{gray}{.94}
\definecolor{lightgray}{rgb}{.9,.9,.9}
\definecolor{darkgray}{rgb}{.4,.4,.4}
\definecolor{purple}{rgb}{0.65, 0.12, 0.82}
\definecolor{background}{HTML}{EEEEEE}
\definecolor{delim}{RGB}{20,105,176}
\colorlet{punct}{red!60!black}
\colorlet{numb}{magenta!60!black}

% Listados
\lstset{
	aboveskip=0.5cm,
	backgroundcolor=\color{gray97},
	basicstyle=\scriptsize\ttfamily,
	breaklines=true,
	commentstyle=\color{gray45},
	frame=Ltb,
	framerule=0.5pt,
	framesep=0pt,
	framexbottommargin=3pt,
	framexleftmargin=0.1cm,
	framextopmargin=3pt,
	keywordstyle=\bfseries,
	numberfirstline = false,
	numbers=left,
	numbersep=6pt,
	numberstyle=\tiny,
	rulesep=.4pt,
	rulesepcolor=\color{black},
	showstringspaces = false,
	stringstyle=\ttfamily,
	literate={á}{{\'a}}1
	         {é}{{\'e}}1
	         {í}{{\'i}}1
	         {ó}{{\'o}}1
	         {ú}{{\'u}}1
	         {ñ}{{\~n}}1
}
 
% Minimizar fragmentado de listados
%\lstnewenvironment{listing}[1][]
%	{\lstset{#1}\pagebreak[0]}{\pagebreak[0]}

% Para que las páginas en blanco no tengan cabecera
\makeatletter
\def\clearpage{%
  \ifvmode
    \ifnum \@dbltopnum =\m@ne
      \ifdim \pagetotal <\topskip
        \hbox{}
      \fi
    \fi
  \fi
  \newpage
  \thispagestyle{empty}
  \write\m@ne{}
  \vbox{}
  \penalty -\@Mi
}
\makeatother

\begin{document}
\input{front/front}
\frontmatter
\begin{center}
{\LARGE\bfseries\titulo}\\
\end{center}
\begin{center}
\autor\
\end{center}

\textbf{Palabras clave: }{\keywordsEs}

\section*{Resumen}

En este {\sf Trabajo de Fin de Máster} se ha propuesto proporcionar un punto final a un sistema de información basado en {\sf grafos RDF}, de forma que los datos contenidos sean comprensibles y razonables automáticamente por máquinas, utilizando una interfaz para la recuperación de información mediante consultas en el lenguaje {\sf SPARQL}. 

\bigskip
Para hacer esto se ha propuesto transformar los conjuntos de datos que se consideren relevantes de los disponibles en el portal de datos abiertos de la {\sf Universidad de Granada}, {\sf OpenData UGR}, al formato triple RDF siguiendo el modelo de una ontología diseñada para describir y representar la información contenida en los datos almacenados; esto se hará utilizando diferentes estándares definidos por el {\sf World Wide Web Council (W3C)} para la definición de la web semántica como son {\sf RDF}, {\sf RDFS} y {\sf OWL}. Será necesario transformar los datos originales (con diferentes distribuciones, todos en formato {\sf CSV}) en un formato compatible con este sistema, principalmente {\sf RDF/XML}, aunque también se utilizará el formato {\sf Turtle (Terse RDF Triple Language)}. Para ello, se usarán scripts escritos en {\sf Python} para ese propósito.

\bigskip
Como resultado del proyecto tendremos una interfaz mediante la cual usando consultas {\sf SPARQL} podremos obtener información que no consta literalmente en nuestros datos almacenados, pero que gracias a los mecanismos de inferencia el sistema puede procesar automáticamente. Esta información puede sernos útil para sacar conclusiones que a simple vista si analizamos los datos en su estado original es muy posible que no percibiéramos, en definitiva, esta información nos sirve para explotar al máximo el conocimiento que alberga el sistema.

\newpage
\begin{center}
{\LARGE\bfseries\tituloEng}\\
\end{center}
\begin{center}
\autor\
\end{center}

\textbf{Keywords: }{\keywordsEn}

\section*{Abstract}

In this {\sf Master’s End Work} it has been proposed to provide an end point to an {\sf RDF graphs} based information system, so that the data contained are understandable and reasonable automatically by machines, using an interface for information retrieval through queries in the {\sf SPARQL} language.
\bigskip

To do this it has been proposed to transform data sets that are considered relevant to those available in the open data portal of the {\sf University of Granada}, {\sf OpenData UGR}, to the RDF triple format following the model of an ontology designed to describe and represent the information contained in the stored data; this will be done using different standards defined by the {\sf World Wide Web Council (W3C)} for the definition of the semantic web such as {\sf RDF}, {\sf RDFS} and {\sf OWL}. It will be necessary to transform the original data (with different distributions, all in {\sf CSV} format) in a format compatible with this system, mainly {\sf RDF/XML}, although the {\sf Turtle} format ({\sf Terse RDF Triple Language}) will also be used. For that, scripts written in {\sf Python} will be used for this purpose.

\bigskip
As a result of the project we will have an interface whereby using {\sf SPARQL} queries we can get information that is not literally stored in our data, but thanks to the inference mechanisms the system can process automatically. This information can be useful to us to draw conclusions that to the naked eye if we analyze the data in its original state is very possible that we didn't perceive. In short, this information serves to exploit to the maximum the knowledge that houses the system.

\newpage
\thispagestyle{empty}
\
\vspace{3cm}

\noindent\rule[-1ex]{\textwidth}{2pt}\\[4.5ex]

Yo, \textbf{\autor}, alumno de la titulación \textbf{\master} de la \textbf{\escuela\ de la \universidad}, autorizo la ubicación de la siguiente copia de mi Trabajo Fin de Máster (\textit{\titulo}) en la biblioteca del centro para que pueda ser consultada por las personas que lo deseen.

\bigskip
Además, este mismo trabajo es realizado bajo licencia \textbf{Creative Commons Attribution-ShareAlike 4.0} (\url{https://creativecommons.org/licenses/by-sa/4.0/}), dando permiso para copiarlo y redistribuirlo en cualquier medio o formato, también de adaptarlo de la forma que se quiera, pero todo esto siempre y cuando se reconozca la autoría y se distribuya con la misma licencia que el trabajo original. El documento en formato {\sf LaTeX} se puede encontrar en el siguiente repositorio de {\sf GitHub}: \url{https://github.com/germaaan/ProyectoDatosAbiertos}.

\vspace{4cm}

\noindent Fdo: \autor

\vspace{2cm}

\begin{flushright}
\ciudad, a \today
\end{flushright}

\newpage
\thispagestyle{empty}
\
\vspace{3cm}

\noindent\rule[-1ex]{\textwidth}{2pt}\\[4.5ex]

D. \textbf{\tutor}, profesor del \textbf{Departamento de Arquitectura y Tecnología de Computadores} de la \textbf{\universidad}.

\vspace{0.5cm}

\vspace{0.5cm}

\textbf{Informa:}

\vspace{0.5cm}

Que el presente trabajo, titulado \textit{\textbf{\titulo}}, ha sido realizado bajo su supervisión por \textbf{\autor}, y 
autoriza la defensa de dicho trabajo ante el tribunal que corresponda.

\vspace{0.5cm}

Y para que conste, expide y firma el presente informe en \ciudad\ a \today.

\vspace{1cm}

\textbf{El tutor:}

\vspace{5cm}

\noindent \textbf{\tutor}

\chapter*{Agradecimientos}
\thispagestyle{empty}

\vspace{1cm}

A toda las personas de las comunidades en las que participo activamente y que me hacen descubrir todos los días cosas nuevas; son una fuente inagotable de inspiración.

\begingroup
\let\cleardoublepage\clearpage
  \tableofcontents
  %\listoffigures
  \listoftables
\endgroup

\newpage
\thispagestyle{empty}
\
\mainmatter
\chapter{Introducción}

\section{Descripción del problema}

Este {\sf Trabajo Fin de Máster} aborda el problema de la obtención de información de los datos contenido en un portal de datos abiertos mediante peticiones a un interfaz en forma de consultas, esto nos permitirá obtener conclusiones sobre la información subyacente en los datos.

\bigskip
Actualmente el portal de datos abiertos de la {\sf Universidad de Granada}\footnote{\url{http://opendata.ugr.es/}} tiene 40 conjuntos de datos con 340 tablas de datos sobre diferentes aspectos de la propia Universidad: matrículas, demanda académica, información salarial, empleo de egresados...

\bigskip
En cumplimiento de la legislación vigente sobre reutilización de la información del \textbf{Sector Público (Ley 37/2007 de 16 de noviembre\footnote{\url{https://sedempr.gob.es/sites/default/files/fileupload/A47160-47165.pdf}} y Real Decreto 1495/2011 de 24 de octubre\footnote{\url{https://sedempr.gob.es/sites/default/files/fileupload/BOE-A-2011-17560\_0.pdf}})} podemos encontrar un gran número de portales abiertos tanto a nivel regional como nacional; sin embargo, no en todos podemos encontrar la misma facilidad para trabajar con los propios datos.

\bigskip
En el caso particular del trabajo con los datos de la {\sf Universidad de Granada}, a través de diferentes hackatones organizados por la {\sf Oficina de Software Libre de la Universidad de Granada} (actuales mantenedores tanto del portal de transparencia como del portal de datos abiertos) se ha venido trabajando con mucho interés en los datos relacionados con las matriculaciones para intentar explicar por ejemplo la tendencia en la elección de titulaciones de determinadas ramas de conocimiento en función del sexo del estudiante. 

\newpage
Es por eso que este tipo de datos serán los primeros en ser adaptados para ser procesados mediante este sistema, pudiendo además extenderse a los mismos datos procedentes de otras universidades para tener una visión más global; y finalmente ampliando a otros conjuntos de datos precisos. 

\bigskip
Para que podamos trabajar con los datos de esa forma, es necesario primero que dichos datos sean \textit{entendibles} por una máquina de forma que intuitivamente representen la realidad del mundo físico, reproduciendo las propiedades que puede tener un entidad y las acciones que puede llevar a cabo; para ello es necesario disponer de información adicional que describan el contenido, el significado y la relación entre los datos. Esto es lo que se conoce como \textbf{Web Semántica}.

\section{Web semántica}
La idea original sobre la Web Semántica fue propuesta por {\sf Tim Berners-Lee} en el año 2001 partiendo de la base de que {\sf la Web actual} estaba demasiado orientada a que sus contenidos fueran leídos por humanos, ya que sus contenidos y documentos tenían un diseño que priorizaba su aspecto sobre su significado. Precisamente esto es lo que se quiere conseguir con {\sf la Web semántica}, darle significado a los datos de forma que no dependan de la apreciación que les pueda dar un lector humano, si no que también puedan ser fácilmente interpretados por las máquinas. Para que esto sea posible, es necesario proporcionar una infraestructura que permita a las máquinas acceder a una estructura común de datos, consiguiendo así que todos los datos queden integrados y enlazados en una especie de base de datos mundial que sea la agregación de innumerables fuentes de datos, pero que gracias a esa estructura común permita una fácil interoperatividad entre distintos sistemas, lo que también se conseguirá gracias al establecimiento de una serie de reglas que permitirán que las máquinas creen hagan uso de unos mecanismo de inferencia que les permitan interpretar los datos que están procesando, 

\bigskip
Para definir esta estructura común de datos se utiliza el modelo de datos {\sf RDF (Resource Description Framework)}, este modelo nos provee de un prototipo estándar para el intercambio de información entre diferentes fuentes de forma que no pierdan su significa y sea posible su interpretación por las máquinas. 

\newpage
En {\sf RDF} la información descriptiva sobre los recursos se construyo en forma de expresiones sujeto-predicado-objeto, lo que se conoce como un \textbf{triple RDF}. Por ejemplo, si tomamos el triple RDF ``Titulación - perteneceA - RamaConocimiento" un caso particular sería ``Informática - perteneceA - IngenieríaArquitectura". Además, para que estos datos se puede referenciar inequívocamente a través de la Web cada elemento tendrá un {\sf URI (Universal Resource Identifier)}.

\bigskip

Otro elemento fundamental a tener en cuenta a la hora de dotar de significa a los datos son las ontologías. Una ontología nos permite hacer una definición formal de los conceptos del dominio de interés con el que estamos tratando, indicando cuáles son sus tipos, propiedades y relaciones de forma que podamos detallar el estado completo de lo que estamos queriendo modelar.

\bigskip
Una vez que ya tenemos la estructura de datos creada, simplemente nos quedará establecer algún mecanismo que nos permita realizar consultas sobre esa información. Para esto la {\sf W3C (World Wide Consortium)} recomienda oficialmente hacer uso de {\sf SPARQL (SPARQL Protocol and RDF Query Language)}. {\sf SPARQL} es una lenguaje estandarizado para la consulta de grafos {\sf RDF} de distintas fuentes de datos que permite hacer búsquedas sobre los recursos utilizando consultas que se asemejan a las típicamente usadas sobre bases de datos relacionales.

\bigskip

Un aspecto que falta por comentar y que es muy importante es el concepto de \textbf{dato enlazado}, más generalmente referido con su traducción en ingles \textbf{linked data}. La idea detrás de los datos enlazados es que además de publicar la información referente a los datos, también se vinculen a otros datos relacionados similares de forma que cuando una máquina procese la información semántica de dichos datos, automáticamente también pueda llegar a información relacionada, pero que no está incluida en la publicación de los datos originales en los datos originales. Esto también se consigue gracias a la existencia de las ontologías, ya que la posibilidad de estudiar el dominio a tratar hace que los datos puedan ser reutilizados e integrados en otros sistemas permitiendo así una interoperabilidad de los mismos.

\bigskip
Para poder usar datos enlazados es imprescindible hacer dos cosas: que nuestros recursos tengan un {\sf URI} que les permita ser identificados de forma inequívoca en {\sf la Web} y, además, que nuestros recursos incluyan enlaces a otros {URI} relacionadas con los datos contenidos en nuestros recursos. Uno de los conjuntos de datos más se suelen usar para enlazar datos es {\sf DBpedia\footnote{\url{http://wiki.dbpedia.org/}}}, un proyecto para extraer datos de {\sf Wikipedia\footnote{\url{https://www.wikipedia.org/}}} de forma que se pueda obtener una versión de la misma transformada en web semántica.

\newpage
\section{La web semántica en otras universidades}

A nivel nacional, aunque si bien es cierto que muchas instituciones públicas con las normativas antes mencionadas se apresuraron a crear sus propios portales de transparencia y/o catálogos de datos abiertos, hoy en día son pocas las universidades que mantienen los datos actualizados y los servicios endpoint {\sf SPARQL} funcionando (los que dispusieran de uno). Parándonos a buscar entre las universidades de mayor relevancia, podemos encontrar con casos como los de la {\sf Universitat Pompeu Fabra} \footnote{\url{https://data.upf.edu/}} o la {\sf Universidad Pablo de Olavide} \footnote{\url{https://datos.upo.gob.es/}}, ambas con datos sin actualizar desde hace varios años y un endpoint de {\sf SPARQL} inoperativo.

\bigskip
Aunque también hay que decidir que la adaptación de este tipo de tecnologías por las universidades en general ha sido pobre, internacionalmente si podemos encontrar auténticos referentes como es la {\sf University of Southampton} \footnote{\url{http://data.southampton.ac.uk/}}, donde podemos encontrar decenas de conjuntos de datos actualizados, además de proveer de un endpoint {\sf SPARQL} totalmente funcional.
\chapter{Objetivos}

El principal objetivo de este {\sf Trabajo Fin de Máster} consiste en obtener información sobre los datos contenidos en el portal de datos abiertos de la {\sf Universidad de Granada} mediante peticiones a una interfaz web. Para conseguir el resultado final esperado debemos cumplir los siguientes objetivos:

\begin{enumerate}
	\item Desarrollo de ontología que permita describir y representar la información contenida en los datos almacenados en dicho portal.
	\item Procesamiento de los conjuntos de datos del portal de datos abiertos de la {\sf Universidad de Granada} para convertirlos del formato {\sf CSV} actual al formato {\sf triple RDF}.
	\item Proveer de un punto de acceso a un sistema de recuperación de datos mediante {\sf SPARQL}, que nos permite recuperar información sobre los datos en {\sf RDF} generados.
\end{enumerate}

Una vez finalizado todo el proceso, además de haber obtenido la posibilidad de obtener la información desde un punto de acceso, tendremos varios conjunto de datos en formato {\sf triple RDF} que queremos que sean públicos al igual que los datos originales en formato {\sf CSV}, por lo que también serán liberados con la licencia original: {\sf Open Data Commons Attribution License}\footnote{\url{http://www.opendefinition.org/licenses/odc-by}}.
\chapter{Planificación inicial del Trabajo}

\section{Modelo de desarrollo}

Si bien no existe una metodología correcta para el desarrollo de ontologías, principalmente debido a que un punto muy influyente la percepción que tengamos nosotros mismos de la \textit{realidad} que queremos modelar, si que existen propuestas para llevar a cabo esta tarea.

\subsection{METHONTOLOGY}

Esta metodología divide el desarrollo de una ontología en las siguientes fases:

\begin{enumerate}
	\item \textbf{Construir el glosario de términos}: incluimos todos los términos relevantes del dominio (conceptos, instancias, atributos, relaciones) y sus descripciones en lenguaje natural.
	\item \textbf{Construir taxonomías de conceptos}: seleccionamos todos los conceptos que hemos definido en nuestro glosario y definimos con la taxonomía la jerarquía en estos conceptos.
	\item \textbf{Construir un diagrama de relaciones binarias}: consiste en establecer las relaciones entre los conceptos de la ontología y sus inversas.
	\item \textbf{Construir el diccionario de conceptos}: especificar qué propiedades describen cada concepto de la taxonomía, las relaciones del diagrama y las instancias de cada uno de los conceptos.
	\item \textbf{Describir en detalle las relaciones binarias}: especificando nombre, origen, destino, cardinalidad y relación inversa.
	\item \textbf{Describir en detalle los atributos de instancia}: especificando nombre, concepto al que pertenece, tipo de valor, rango de valores, cardinalidad.
	\item \textbf{Describir en detalle los atributos de clase}: especificando nombre, concepto donde se define, tipo de valor, rango de valores, cardinalidad.
	\item \textbf{Describir en detalle las constantes}: especificando nombre, tipo de valor, valor y unidad de medida (si es numérica).
	\item \textbf{Definir los axiomas formales}: especificando nombre, descripción de la restricción, expresión lógica a cumplir, conceptos a los que afecta, relaciones a las que afecta y variables que usa.
	\item \textbf{Definir reglas}: especificando nombre, descripción de la regla, expresión en forma si-entonces, conceptos a los que afecta, atributos a los que afecta, relaciones a las que afecta y variables que usa.
	\item \textbf{Describir instancias}: especificando nombre de la instancia, nombre del concepto al que pertenece y los valores de los atributos si se conocen.
\end{enumerate}

\subsection{On-To-Knowledge}

Esta metodología divide el desarrollo de una ontología en las siguientes fases:

\begin{enumerate}
	\item \textbf{Estudio de viabilidad}: se identifica el problema a resolver y sus posibles solucions.
	\item \textbf{Arranque}: se crea un documento de especificaciones de requisitos de la ontología: objetivo, dominio, alcance, aplicaciones que van a hacer uso de ella, fuentes de conocimiento, usuarios y escenarios, preguntas de competencia y ontologías existentes.
	\item \textbf{Refinamiento}: se realiza una clasificación de los términos de la ontología en vista de una posterior formalización con relaciones y axiomas.
	\item \textbf{Evaluación}: se comprueba la utilidad de ontología poniendo a prueba que cumple con los requisitos definidos en el documento de especificaciones.
	\item \textbf{Mantenimiento}: se actualiza la ontología en función de los cambios que se haga en las especificaciones durante el ciclo de vida de la propia ontología.
\end{enumerate}

\subsection{Protégé}

{\it Protégé} es un editor de ontologías de código abierto desarrollado por la {\it Universidad de Stanford} que permite desarrollar ontologías siguiendo una metología creada por los propios desarrolladores de {\it Protégé}. Se componente de las siguientes fases:

\begin{enumerate}
	\item \textbf{Determinar el dominio y el alcance de la ontología}: se establecen cuestiones como para qué se va a utilizar, qué preguntas ha de responder o a quién va dirigida.
	\item \textbf{Considerar la reutilización de ontologías existentes}: se busca reaprovechar recursos de ontologías existentes que se pueden considerar de utilidad para nuestro caso; además, de facilitar la interconexión con otras aplicaciones que hagan uso de otro ontologías.
	\item \textbf{Enumerar términos importantes para la ontología}: realizando una lista de conceptos que se quieren tratar con lo ontología.
	\item \textbf{Definir las clases y su jerarquía}: usando los conceptos de la lista anterior para dar forma a la taxonomía de la ontología.
	\item \textbf{Definir las propiedades de las clases}: para así darle capacidad de representación de información de las clases.
	\item \textbf{Definir las restricciones de las propiedades}: tipo de valores, cardinalidad, dominio, rango...
	\item \textbf{Crear instancias}: indicando los valores de las propiedades de cada una de las instancias de las clases.
\end{enumerate}
	
\subsection{Metodología seleccionada}

Debido a que en el problema que abarca este {\it Trabajo} no se parte de un sistema inexistente, si no que se va a trabajar sobre unos conceptos ya existentes, el uso de metodologías tan formales como {\it METHONTOLOGY} o {\it On-To-Knowledge} se hace algo de difícil de adaptación, es por eso que se han incluido como simples planteamientos teóricos.

\bigskip
En el caso de la ontología usada en {\it Protégé}, aunque existe el mismo problema, si es cierto que se puede adaptar a nuestro caso, por lo que es lo que se va a utilizar en el desarollo.

\section{Gestión de recursos}

\subsection{Personal}

El personal que se encargará del desarrollo de este proyecto es única y exclusivamente el autor del mismo, {\it Germán Martínez Maldonado}, encargándose de todas las fases del mismo: \textit{análisis}, \textit{diseño}, \textit{implementación} y \textit{evaluación de resultados}.

\subsection{Hardware}

El hardware utilizado para el desarrollo del proyecto se compone de dos máquinas diferentes: un ordenador portátil personal en el que se realizará el desarrollo y las pruebas haciendo uso de máquinas virtuales, y un servidor con visibilidad pública que será en el que se desplegará la versión final de proyecto. Las especificaciones de ambos sistemas son las siguientes.

\begin{table}[!ht]
	\centering
	\begin{tabular}{|p{.15\textwidth}|p{.35\textwidth}|p{.35\textwidth}|}
		\hline
		 &
		\textbf{Ordenador personal}&
		\textbf{Servidor de producción}
		\\ \hline
		CPU:&
		Intel Core i7-5500U CPU @ 3GHz&
		Intel Core2 Duo CPU E4400 @ 2GHz
		\\ \hline
		RAM:&
		8 GB&
		4 GB
		\\ \hline
		SO:&
		Ubuntu 16.10 yakkety&
		Ubuntu 16.04 xenial
		\\ \hline
		Kernel:&
		x86\_64 Linux 4.8.0-58-generic&
		x86\_64 Linux 4.4.0-83-generic
		\\ \hline
	\end{tabular}
	\caption{Características hardware utilizado}
	\label{caracteristicas-hardware}
\end{table}

\subsection{Software}

Todo el software que se ha utilizado para el desarrollo del proyecto es software de código abierto en un compromiso con la ciencia abierta y sostenible.

\begin{itemize}
	\item \textbf{Vagrant}\footnote{\url {https://www.vagrantup.com/}}: es una herramienta que nos permite crear fácilmente entornos virtuales en los que probar nuestro proyecto durante el desarollo. 
	\item \textbf{Ansible}\footnote{\url {https://www.ansible.com/}}: es una herramienta para automatizar la configuración de un sistema, instalando todo el software que necesitemos y permitiéndonos adaptarlo a nuestras necesidades. 
	\item \textbf{CKAN}\footnote{\url {https://ckan.org/}}: es una plataforma con la que podemos montar nuestro propio almacén de datos abiertos en el que publicar y visualizar datos en diferentes formatos.	
	\item \textbf{NGINX}\footnote{\url {https://nginx.org/en/}}: es un servidor web ligero, sencillo y ágil que usaremos para albergar los archivos de ontología y recursos producidos como fruto de este proyecto.
	\item \textbf{OpenLink Virtuoso Open-Source Edition}\footnote{\url {http://vos.openlinksw.com/owiki/wiki/VOS/}}: es un servidor ORDBMS abierto que permite el almacenamiento y la gestión de datos en formato RDF, además de proveer de un punto de acceso a un sistema de recuperación de datos mediante SPARQL.
\end{itemize}

\section{Planificación temporal}

El desarrollo de este {\sf Trabajo} se ha llevado a cabo entre principios de enero de 2017 y principios de julio de ese mismo año, habiéndose dividido en las siguientes fases:

\begin{enumerate}
	\item \textbf{Investigación inicial sobre la temática}: antes de empezar es necesario hacer un trabajo previo de investigación para adquirir todos los conocimientos que sean necesarios. metodologías para el desarrollo de ontologías, estándares de {\sf la Web Semántica}, endpoints {\sf SPARQL}...; aunque se irá viendo más en profundidad según se vaya avanzando en el {\sf Trabajo}, este trabajo previo es a lo que se dedicarán los primeros meses.
	\item \textbf{Documentación}: en este proyecto la documentación tiene un papel muy importante, ya que la parte de mayor complejidad corresponde al análisis y diseño del sistema que solo verán representadas en la misma; pero aunque la más importante, solo es una de las partes, así que con el fin de obtener una buena documentación, está se irá realizando simultáneamente con los avances que se vayan produciendo durante todo el desarrollo del proyecto.
	\item \textbf{Análisis y diseño del sistema}: como hemos dicho esta es la parte más importante del proyecto, ya que de ella depende el resto del desarrollo. Con el fin de estudiar todos los aspectos necesarios de la mejor forma posible se emplearán otro par de meses para esta tarea.
	\newpage
	\item \textbf{Desarrollo de los objetivos}: una vez que tenemos el diseño ya finalizado la implementación será bastante sencilla, ya que simplemente deberemos procesar los datos originales para que se adapten al diseño realizado. Con esto hecho solo nos quedará poner el marcha el servidor que albergará los datos y el servidor que proveerá el punto de acceso {\sf SPARQL}. Esta tarea no debería llevar más de un mes.
	\item \textbf{Análisis de resultados y conclusiones}: Con todo el trabajo ya finalizado, solo resta analizar los resultados para comprobar que cumplen los requisitos establecidos y, por fin, ya solo quedaría pasar a escribir las conclusiones y posibles trabajos futuros que se puedan derivar del trabajo realizado. El último mes se empleará en este fin.
\end{enumerate}
\chapter{Estándares de la Web Semántica}

\section{Ontologías}
Aunque en la introducción hemos explicado brevemente lo que era una ontología, lo que básicamente describíamos como una definición formal de los conceptos del dominio de interés con el que estamos tratando de indicar cuáles son sus tipos, propiedades y relaciones para conocer el estado completo de nuestro dominio de conocimiento. Pues para construir estas ontologías podremos usar dos lenguajes independientes, pero que que se suelen usar conjuntamente: {\sf RDF Schema} y {\sf OWL}.

\section{RDF Schema}
{\sf RDF Schema} (o simplemente {\sf RDFS}) es una extensión del {\sf RDF} original que utiliza su misma estructura y que aparece para solucionar problemas básicos en la definición de ontologías como pueden ser la declaración de clases, las restricciones entre ellas y las restricciones de sus propiedades; ya que en la versión original de {\sf RDF} solo eran posible definir tipos (lo que no es exactamente igual que una clase) y las propiedades de los mismos.

\bigskip
Gracias a que {\sf RDFS} nos provee de los elementos básicos y comunes para la descripción de los datos de nuestro dominio, esto nos puede servir para reutilizar conceptos que son comunes en varios dominios distintos. Para conseguir esto {\sf RDFS} nos permite crear esquemas sencillos usando clases y subclases, además de permitirnos definir propiedades, dominios de aplicación y rangos de valores.

\newpage
\subsection{Clases en RDF Schema}
Las clases son conjuntos de recursos que tienen características comunes y una representación en el mundo real. Tenemos tres clases fundamentales en {\sf RDFS} a partir de las que podemos definir nuestras propias clases, siempre posible crear además una jerarquía de clases.

	\begin{itemize}
		\item {\tt rdfs:Class} es la clase que utilizaremos para definir otras clases.
		\item {\tt rdfs:Property} es la clase a partir de la cual definiremos nuevas propiedades que nos permitan describir nuestros recursos.
		\item {\tt rdfs:Resource} todas las cosas descritas en {\sf RDF} son llamadas {\sf recursos} y son instancias de esta clase, es por eso que esta clase nos permite referenciar cualquier clases desde otra clase.
	\end{itemize}

\subsection{Propiedades en RDF Schema}
Las propiedades nos permitirán describir las características que tienen los distintos recursos, además de las propias relaciones que existen entre distintos recursos.

	\begin{itemize}
		\item {\tt rdf:type} nos permite definir el tipo de un determinado recurso, además, un recursos puede ser una instancia de más de una clase.
		\item {\tt rdfs:subClassOf} es la que nos permitirá definir una jerarquía de clases y subclases, además, una clase puede ser subclase de otras subclases.
		\item {\tt rdfs:subPropertyOf}, al igual que podemos crear subclases, también podemos crear subpropiedades para establecer una jerarquía de las mismas.
	\end{itemize}

\subsection{Restricciones en RDF Schema}	

Las restricciones son las que nos permiten definir las clases sobre la que pueden aplicarse determinadas propiedades y posibles valores:

	\begin{itemize}
		\item {\tt rdfs:domain} definiría las clases sobre las que se puede aplicar una propiedad.
		\item {\tt rdfs:range} definiría los valores que puede tener una propiedad.
	\end{itemize}

\newpage
\section{OWL}

{\sf OWL} ({\sf Ontology Web Language}) es un lenguaje de especificación de ontologías extensión de {\sf RDF Schema}, proporcionando un mayor significado y semántica a las ontologías, ya que nos permite definir restricciones sobre las propiedades (como por ejemplo, los valores que puede tomar una clase) y definir axiomas, sentencias que son siempre ciertas y que nos serán muy útiles a la hora de realizar restricciones (como por ejemplo, que un recurso no puede pertenecer a dos clases que sean disjuntas). Utilizando esta jerarquía de clases y propiedades que forman la ontología, {\sf OWL} se basa en lógicas descriptivas para realizar razonamientos.

\subsection{Sublenguajes de OWL}	
Por otra parte, {\sf OWL} se puede clasificar en tres tipos, según su nivel de expresividad:

\begin{itemize}
	\item \textbf{OWL Lite}. Es la versión más simple y nos provee únicamente de los elementos necesarios para definir una jerarquía de clases y propiedades con restricciones básicas como los tipos de valores, cardinalidades o propiedades de las características (inversa, transitiva, simétrica, funcional e inversa funcinal) entre otras.
	\item \textbf{OWL DL}. Posee mayor expresividad que la anterior, permitiendo además que las jerarquías sean razonadas automáticamente por inferencia que además permitan encontrar posibles inconsistencias mediante el uso de diferentes axiomas (como especificar clases disjuntas)
	\item \textbf{OWL Full}. Es la versión con un nivel mayor de expresividad, pero que está basadas en semánticas diferentes a {\sf OWL Lite} y {\sf OWL DL}. Por ejemplo, en {\sf OWL Full}, una clase puede ser simultáneamente un colección de individuos y un individuo en sí mismo, algo que no está permitido en {\sf OWL DL}.
\end{itemize}

\subsection{Clases e instancias en OWL}	
Después de hablar sobre los tipos de clasificación de {\sf OWL}, tenemos que especificar sus componentes generales, por lo que vamos a comenzar por las clases y las instancias.

\bigskip

Todas las clases {\tt (owl:Class)} son a su vez subclases de una única superclase {\tt (owl:Thing)}. Además las subclases e instancias se definen con {\sf RDFS} mediante las propiedades {\tt rdfs:subClassOf} y {\tt rdfs:type} respectivamente. También una clase puede crearse como la intersección de varias clases {\tt (owl:intersectionOf)}, la unión de varias clases {\tt (owl:unionOf)} o el complemento de otra clase {\tt (owl:complementOf)}.

\subsection{Propiedades en OWL}	

Podemos definir propiedades literales de una clase con {\tt owl:DatatypeProperty}, indicando además el tipo de dato de {\sf XML Schema} del que es dicha propiedad; además, también podemos indicar las propiedades que relacionan distintas clases con {\tt owl:ObjectProperty}. Además, {\sf OWL}  nos permite definir diferentes tipos de propiedades:

	\begin{itemize}
		\item \textbf{Transitiva {\tt (owl:TransitiveProperty)}}: si el sujeto A está relacionado con el sujeto B por medio de la propiedad P y por otro lado el sujeto B está relacionado con el sujeto C también por medio de la propiedad P, si la propiedad P está definida como una propiedad transitiva, se puede inferir que el sujeto A está relacionado con el sujeto C por medio de la propiedad P.
		\item \textbf{Simétrica {\tt (owl:SymmetricProperty)}}: si el sujeto A está relacionado con el sujeto B por medio de la propiedad P, si la propiedad P está definida como una propiedad simétrica, se puede inferir que el sujeto B está relacionado con el sujeto A por medio de la propiedad P.
		\item \textbf{Funcional {\tt (owl:FunctionalProperty)}}: si el sujeto A está relacionado con el sujeto B por medio de la propiedad P y por otro lado el sujeto B está relacionado con el sujeto C también por medio de la propiedad P, si la propiedad P está definida como una propiedad funcional, se puede inferir que el sujeto B y el sujeto C son el mismo sujeto.
		\item \textbf{Inversa {\tt (owl:InverseOf)}}: si la propiedad P1 está definida como la propiedad inversa de la propiedad P2, entonces si el sujeto A está relacionado con el sujeto B por medio de la propiedad P1, se puede inferir que el sujeto B está relacionado con el sujeto A por medio de la propiedad P2.
		\item \textbf{Inversa funcional {\tt (owl:InverseFunctionalProperty)}}: si la propiedad P está definida como propiedad inversa funcional, entonces si el sujeto A está relacionado con el sujeto B por medio de la propiedad P, se puede inferir que el sujeto A y el sujeto B son en realidad el mismo sujeto.
	\end{itemize}

\newpage
\subsection{Restricciones y axiomas en OWL}	
Una vez definidas las clases y las propiedades solo nos queda ver como se definen las restricciones sobre las propiedades. Además del dominio y el rango que ya podíamos definir con {\sf RDFS}, {\sf OWL} nos permite hacer restricciones más descriptivas como indicar de qué forma las instancias de una clase pueden tomar como únicos valores las instancias de otra clase ({\tt owl:allValuesFrom}, {\tt owl:someValuesFrom}, {\tt owl:hasValue}) o la cardinalidad de los mismos ({\tt owl:maxCardinality}, {\tt owl:minCardinality}, {\tt owl:cardinality}).

\bigskip
Los axiomas son sentencias que siempre son ciertas, {\sf OWL DL} se basa en la lógica descriptiva para partiendo de la jerarquía y restricciones de clases poder inferir de forma automática razonamientos sobre los datos que tiene. En concreto, podemos definir desde controlar la integridad de nuestros datos mediante la definición de clases o propiedades disjuntas ({\tt owl:disjointWith}, {\tt owl:AllDisjointClasses}, {\tt owl:AllDisjointProperties}); o indicar que nuestras clases y propiedades son equivalentes a clases y propiedades de otros vocabularios ({\tt owl:equivalentClass}, {\tt owl:equivalentProperty}). Esto último es lo que realmente le da utilidad a tener estructuras con un marco común, ya que es lo que permite la interoperabilidad, permiten que existan los datos enlazados.

\section{SPARQL}

{\sf SPARQL} es el lenguaje recomendado por la {\sf W3C} para acceder a datos almacenados en {\sf RDF}. Las consultas {\sf SPARQL} se hacen sobre un almacén de datos {\sf RDF}, pero además de la propia consulta será necesario especificar el espacio de nombres mediante el {\sf URI} de las ontologías cuyos términos vamos a usar a la hora de realizar las consultas. Además, independientemente del lenguaje en el que estén almacenados los datos ({\sf RDF/XML}, {\sf N3}, {\sf Turtle}, {\sf RDFa}) las consultas se escribirán con sintaxis {\sf Turtle}, lo que además de aportar simplicidad tiene otras ventajas como permite abreviar los {\sf URIs} mediante el uso de prefijos.

\bigskip
Al igual que los triples {\sf RDF}, las consultas {\sf SPARQL} se contruyen a partir de tres elementos: sujeto, predicado y objeto; el sujeto y el objeto serían variables que se relacionan a través del predicado, el valor de cada uno de estos pares de variables es el resultado que obtendremos de una consulta {\sf SPARQL}. Todas las variables a utilizar en la consulta (cuyos nombres siempre llevan un signo de interrogación delante para indicar que son variables) son declaradas con la sentencia {\tt SELECT} al indicar los datos que se van a seleccionar de entre toda la información almacenada, será después con la sentencia {\tt WHERE} donde al indicar cómo se relacionan sujetos y predicados cuando filtremos los datos que vamos a obtener. Un ejemplo de consulta simple: 

\begin{minted}{sparql}
PREFIX ugr: <http://cabas.ugr.es/ontology/ugr#>

SELECT ?X ?nombre
WHERE {
    ?X ugr:titulacion ?nombre
}
LIMIT 5
\end{minted}

Esta consulta lo que nos daría como resultado son todos los vínculos de los recursos existentes en nuestros datos que se corresponden al nombre de una titulación. Algo similar a lo que podemos ver en la siguiente tabla.

\begin{table}[!ht]
	\centering
	\begin{tabular}{|p{.45\textwidth}|p{.50\textwidth}|}
		\hline
		\textbf{X} &
		\textbf{nombre}
		\\ \hline
		ENLACE\_A\_RECURSO\#1&
		"TITULACIÓN 1"
		\\ \hline
		ENLACE\_A\_RECURSO\#2&
		"TITULACIÓN 2"
		\\ \hline
		ENLACE\_A\_RECURSO\#3&
		"TITULACIÓN 3"
		\\ \hline
		ENLACE\_A\_RECURSO\#4&
		"TITULACIÓN 4"
		\\ \hline
		ENLACE\_A\_RECURSO\#5&
		"TITULACIÓN 5"
		\\ \hline
	\end{tabular}
	\caption{Resultado consulta de ejemplo}
	\label{consulta-ejemplo}
\end{table}

\subsection{Operadores en SPARQL}

Además de las sentencias {\tt SELECT} y {\tt WHERE}, tenemos varios operadores que podemos utilizar a la hora de crear consultas; no obstante, ni siquiera todas las consultas tienen por qué ser de selección, podemos realizar consultas de otros tipos diferentes:

\begin{itemize}
	\item {\tt CONSTRUCT}: nos devuelve el resultado en forma de triples {\sf RDF} con el formato que le hayamos indicado como plantilla.
	\item {\tt ASK}: nos devuelve como resultado si la consulta que hemos realizado tiene solución o no, pero a diferencia de {\tt SELECT} no nos devuelve la información correspondiente.
	\item {\tt DESCRIBE}: devuelve toda la información contenida sobre un recurso solicitado.
\end{itemize}

Y como ya decíamos, hay una serie de operadores que nos permiten hacer consultas de una mayor complejidad:

\begin{itemize}
	\item {\tt UNION}: nos permite obtener el conjunto de resultados de dos consultas distintas, como por ejemplo, todos los resultados de dos consultas {\tt SELECT} distintas se podrían agrupar en una misma respuesta.
	\item {\tt OPTIONAL}: nos permite obtener como resultado el conjunto de recursos que cumplan alguna de las condiciones indicadas en la clausula {\tt WHERE}.
	\item {\tt FILTER}: nos permite obtener como resultado el conjunto de recursos que cumplan con las condiciones indicadas en la clausula {\tt WHERE} añadiendo además criterios de selección arbitrarios.
\end{itemize}

Y para terminar lo único que nos faltaría es hablar sobre los modificadores que podemos usar en las consultas {\sf SPARQL}:

\begin{itemize}
	\item {\tt ORDER BY}: nos permite indicar bajo qué criterio queremos que se ordenen los resultados obtenidos.
	\item {\tt DISTINCT}: nos permite obtener una solución sin soluciones duplicadas.
	\item {\tt REDUCED}: nos permite obtener un resultado en que las soluciones podrán aparecer duplicadas o no en función de otros aspectos, como que el conjunto fuera demasiado grande para comprobar uno por uno que no hay duplicados.
	\item {\tt LIMIT}: nos permite indicar el número máximo de resultados que queremos obtener.
	\item {\tt OFFSET}: nos permite indicar el número de resultados que queremos despreciar antes de devolver la solución, como podría ser usar {\tt OFFSET 10} para devolver a partir de 11º resultado (considerando que haya más de 10 resultados.)
\end{itemize}

\section{Razonamiento e inferencia}

\begin{quote}``\textit{Se entiende por razonamiento a la facultad que permite resolver problemas, extraer conclusiones y aprender de manera consciente de los hechos, estableciendo conexiones causales y lógicas necesarias entre ellos."}\footnote{https://es.wikipedia.org/wiki/Razonamiento}.\end{quote}

\bigskip
Hemos dicho que el principal objetivo es conseguir que los datos sean entendibles por las máquinas, que puedan razonar sobre ellos y sacar conclusiones; precisamente para eso nos basaremos en la lógica. La lógica estudia las condiciones bajo las cuales siguiendo una serie de pasos elementales se puede pasar de una serie de premisas a una conclusión, cuando estamos tratando con máquinas el problema que encontramos en su forma de procesar la lógica es que no dispone de los mecanismos adecuados para conocer cómo y en qué orden deben realizarse esos pasos elementales, algo que si es capaz de hacer la mente humana de forma natural. 

\bigskip
Esto es lo que intenta solucionar la inteligencia artificial, y en nuestro caso concreto es lo que queremos conseguir, que a través de las ontologías definidas se puedan realizar inferencias sobre la información proporcionada por la ontología descrita, que podamos obtener resultados producto del razonamiento que lleven a que la máquina saque sus propias conclusiones.

\bigskip
Por ejemplo, si tenemos varias clases que entre sus datos cuentan con la propiedad \textbf{hombres} y la propiedad \textbf{mujeres} en referencia al número de hombres o mujeres que se han matriculado en una determinada titulación, proceden de un mismo país o cualquier caso similar, nuestro razonamiento humano nos hace saber que los \textbf{hombres} y las \textbf{mujeres} son \textbf{personas}, por lo que si quisiéramos saber cuántas \textbf{personas} hay matriculadas en una titulación, lo que estaríamos haciendo es sumar el número de \textbf{hombres} y de \textbf{mujeres}; sin embargo, este razonamiento no es obvio para una máquina, necesita una regla que le diga que los \textbf{hombres} y las \textbf{mujeres} son \textbf{personas}.

\bigskip
Aquí es donde vuelve a entrar en juego {\sf RDFS}, ya que como hemos indicado antes, su vocabulario nos permite establecer jerarquías, por lo tanto podríamos especificar una jerarquía en la que tenemos una propiedad que es \textbf{persona} y luego indicar que \textbf{hombres} y \textbf{mujeres} son subpropiedades de \textbf{persona}. Gracias a esta propiedad que acabamos de definir, si la máquina quiere contar número de \textbf{personas}, pero estas no están definidas explícitamente como tales, podrá inferir que lo que debe hacer es contar los \textbf{hombres} y las \textbf{mujeres}.
\chapter{Metodología}

\section{Dominio y alcance de la ontología}

La ontología propuesta tiene el objetivo de describir la información que está contenida en los diferentes conjuntos de datos del portal de datos abiertos de la {\sf Universidad de Granada}. Por eso el dominio que abarcará será el datos de carácter universitario como son información demanda académica, matriculaciones o tasas académicas entre otros.
\bigskip
Esto lleva a que la ontología sea definida con el objetivo final de que pueda cubrir todos los conceptos que consideremos interesantes para luego obtener información que, aunque quizás no se pueda percibir a simple vista, la obtendremos de forma simplificada a través de las consultas que realizaremos al sistema.

\section{Ontologías existentes}

Aunque trabajamos con información típica que podemos encontrar en cualquier universidad, la organización propia no suele ser la misma de un organismo a otro. En cualquier caso, podemos encontrar ontologías existentes similares a la que vamos a desarrollar que se han utilizado en otras universidades como la {\sf Universitat Pompeu Fabra} o {\sf Universidad Pablo de Olavide}, pero el primer problema que encontramos y que impide que pudiéramos reutilizar conceptos definidos en su ontología es que al tener la mayoría de datos de tipo general (como son las matrículas) de forma individualizada (los datos no están agrupados, son los datos anonimizados que representan a personas físicas), no se pueden adaptar a nuestros orígenes de datos, donde todos los datos están agrupados.

\newpage
En cualquier caso, como otro los factores que queremos conseguir es fomentar la interoperabilidad entre sistemas sí que vamos a utilizar otras ontologías de nivel superior, como son {\sf DBpedia}\footnote{\url{http://dbpedia.org/ontology/}} o  {\sf Wikidata}\footnote{\url{https://www.wikidata.org/wiki/}}, la importancia que tiene esta última es que por ejemplo nos permitiría cruzar nuestros datos con la información contenida en la {\sf Wikipedia}, lo que podría ser muy interesante de cara a posibles futuros trabajos en los que se usara la ontología aquí desarrollada.

\section{Descripción de los conjuntos de datos}

Los conjuntos de datos con los que se ha decidido trabajar son los siguientes:

\begin{itemize}
	\item \textbf{Demanda académica: procedimientos acceso} contiene información relativa al número total de solicitudes de matrícula demandadas en la universidad. Datos desde el curso 2012/2013 hasta el curso 2014/2015.
	\item \textbf{Demanda académica: titulaciones} contiene información sobre la demanda de matrícula en relación con las plazas ofertadas en titulaciones oficiales de grado en la universidad. Datos desde el curso 2014/2015 hasta el curso 2014/2015.
	\item \textbf{Matrículas: grado} contiene información relacionada con las matrículas de titulaciones de grado realizadas en la universidad, agrupándola por rama de conocimiento, titulación y sexo del estudiante. Datos desde el curso 2010/2011 hasta el curso 2014/2015.
	\item \textbf{Matrículas: posgrado} contiene información relacionada con las matrículas de titulaciones de posgrado realizadas en la universidad, agrupándola por rama de conocimiento, titulación y sexo del estudiante. Datos desde el curso 2010/2011 hasta el curso 2014/2015.
	\item \textbf{Número medio de créditos} contiene información relacionada con el número medio de créditos de los estudiantes de la universidad, agrupándola por rama de conocimiento, plan de estudios, número medio de créditos matriculados, número medio de créditos presentados y número medio de créditos superados. Datos desde el curso 2012/2013 hasta el curso 2013/2014.
	\item \textbf{Oferta de titulaciones: doctorado} contiene información relativa a la oferta de titulaciones para estudios de doctorado, agrupándola por titulación, rama de conocimiento, centro y campus. Datos desde el curso 2013/2014 hasta el curso 2015/2016.
	\item \textbf{Oferta de titulaciones: grado} contiene información relativa a la oferta de titulaciones para estudios de grado, agrupándola por titulación, rama de conocimiento, centro y campus. Datos desde el curso 2013/2014 hasta el curso 2015/2016.
	\item \textbf{Oferta de titulaciones: másteres oficiales} contiene información relativa a la oferta de titulaciones para estudios de másteres oficiales, agrupándola por titulación, rama de conocimiento, centro y campus. Datos desde el curso 2013/2014 hasta el curso 2015/2016.
	\item \textbf{Origen geográfico de estudiantes por país} contiene información relativa al origen geográfico de los estudiantes de la universidad, agrupándola por país de origen y sexo del estudiante. Datos desde el curso 2013/2014 hasta el curso 2015/2016.
	\item \textbf{Origen geográfico de estudiantes por provincia} contiene información relativa al origen geográfico de los estudiantes de la universidad, agrupándola por provincia de origen y sexo del estudiante. Datos desde el curso 2013/2014 hasta el curso 2015/2016.
	\item \textbf{Tasas académicas por titulaciones} contiene información relativa a las tasas académicas de los estudiantes según la titulación que estén estudiando en la universidad, agrupándola por titulación, tasa de rendimiento, tasa de éxito, tasa de abandono inicial, tasa de eficiencia, tasa de graduación y tasa de abandono. Datos desde el curso 2011/2012 hasta el curso 2015/2016.
\end{itemize}

El motivo de haber seleccionado solo estos 11 conjuntos de datos de entre los 40 totales, es que dado la gran cantidad de datos que podemos encontrar en el portal, podemos encontrarnos datos de todo tipo; podemos encontrar datos de fácil interpretación como son los datos relacionados con matrículas, pero también podemos encontranos por ejemplo datos de carácter económico que utiliza una serie de códigos que nos permiten hacer un uso tan natural de la información que contienen, por eso en una primera aproximación solo vamos a incluir datos de los que como personas podríamos sacar información válida a simple vista. Además, si no se tienen datos de varios años tampoco se considerará un conjunto de datos interesante ya que no se puede comprobar si esos datos representan un hecho anecdótico o una tendencia.

\newpage
\subsection{Descripción de las clases del sistema}

Aclarado este punto, lo siguiente es concretar que cada uno de estos conjuntos de datos se corresponderá con una clase en nuestra ontología.

\begin{itemize}
	\item \textbf{Demanda académica: procedimientos acceso} $\rightarrow$ {\tt DemandaAcademicaAcceso}
	\item \textbf{Demanda académica: titulaciones} $\rightarrow$ {\tt DemandaAcademicaTitulacion}
	\item \textbf{Matrículas: grado} $\rightarrow$ {\tt MatriculasGrado}
	\item \textbf{Matrículas: posgrado} $\rightarrow$ {\tt MatriculasPosgrado}
	\item \textbf{Número medio de créditos} $\rightarrow$ {\tt NumMedioCreditos}
	\item \textbf{Oferta de titulaciones: doctorado} $\rightarrow$ {\tt OfertaTitulacionDoctorado}
	\item \textbf{Oferta de titulaciones: grado} $\rightarrow$ {\tt OfertaTitulacionGrado}
	\item \textbf{Oferta de titulaciones: másteres oficiales} $\rightarrow$ {\tt OfertaTitulacionMaster}
	\item \textbf{Origen geográfico de estudiantes por país} $\rightarrow$ {\tt OrigenPais}
	\item \textbf{Origen geográfico de estudiantes por provincia} $\rightarrow$ {\tt OrigenProvincia}
	\item \textbf{Tasas académicas por titulaciones} $\rightarrow$ {\tt TasasAcademicasTitulacion}
\end{itemize}

\section{Descripción de las propiedades}

Con las clases definidas, lo siguiente sería definir sus atributos, las relaciones y restricciones para tener la estructura completa de nuestro sistema. Para esto usaremos {\tt owl:DatatypeProperty} que es subclases de la clase {\tt rdf:Property} definida en el estándar {\sf RDF}. Con {\tt owl:DatatypeProperty} podemos vincular valores del tipo que definamos con instancias de las clases que hemos definido.

\bigskip
En lo que se refiere a los atributos, debemos tener en cuenta que hay atributos que se repiten en varias clases, por lo que tendremos que especificarlo en los dominios de dichas propiedades.

\begin{itemize}
	\item \textbf{Campus} representa cada uno de los campus universitarios en los que está dividida la universidad.
	\item \textbf{Centro} representa cada uno de los centros de la universidad, ya sean facultades, escuelas o centros adscritos.
	\item \textbf{Número medio de créditos matriculados}: representa el número medio de créditos matriculados por los estudiantes matriculados en titulaciones de una misma rama de conocimiento.
	\item \textbf{Número medio de créditos presentados}: representa el número medio de créditos a los que estudiantes matriculados en titulaciones de una misma rama de conocimiento se han presentado a las evaluaciones oficiales.
	\item \textbf{Número medio de créditos superados}: representa el número medio de créditos a los que estudiantes matriculados en titulaciones de una misma rama de conocimiento han superado las evaluaciones oficiales.
	\item \textbf{Cupo general} representa el número de matrículas realizadas por el grupo de estudiantes que pertenece al cupo general de estudiantes.
	\item \textbf{Curso} representa el curso académico al que pertenecen los datos.
	\item \textbf{Deportistas} representa el número de matrículas realizadas por el grupo de estudiantes que pertenece al cupo de deportistas de alto nivel o alto rendimiento.
	\item \textbf{Discapacitados} representa el número de matrículas realizadas por el grupo de estudiantes que pertenece al cupo de personas con minusvalías reconocidas.
	\item \textbf{Doctorado} representa las titulaciones de doctorado que se ofertan en la universidad.
	\item \textbf{Estado} representa el estado de las solicitudes de matrículas demandadas en la universidad.
	\item \textbf{Grado} representa las titulaciones de grado que se ofertan en la universidad.
	\item \textbf{Hombres} representa el número de estudiantes del sexo masculino que estudian alguna titulación en la universidad.
	\item \textbf{Máster} representa las titulaciones de másteres oficiales que se ofertan en la universidad.
	\item \textbf{Mayores 25} representa el número de matrículas realizadas por el grupo de estudiantes que pertenece al cupo de personas mayores de 25 años.
	\item \textbf{Mayores 40 y 45 } representa el número de matrículas realizadas por el grupo de estudiantes que pertenece al cupo de personas mayores de 40 y de 45 años.
	\item \textbf{Mujeres} representa el número de estudiantes del sexo femenino que estudian alguna titulación en la universidad.
	\item \textbf{País de origen} representa el país de origen de los estudiantes matriculados en alguna titulación en la universidad.
	\item \textbf{Personas} representa el número de estudiantes que estudian alguna titulación en la universidad.
	\item \textbf{Plan de estudios} representa el plan de estudios al que pertenece una titulación.
	\item \textbf{Plazas ofertadas} representa el número de plazas ofertadas para una determinada titulación.
	\item \textbf{Provincia} representa la provincia de origen de los estudiantes matriculados en alguna titulación en la universidad.
	\item \textbf{Rama} representa la rama de conocimiento a la que pertenece alguna titulación.
	\item \textbf{Tasa de abandono} representa el porcentaje entre el número total de estudiantes de nuevo ingreso en una titulación que debieron obtener el título el año académico anterior y que no se han matriculado ni en ese año académico ni en el anterior.
	\item \textbf{Tasa de abandono inicial} representa el porcentaje entre los estudiantes matriculados en una determinada titulación en un curso académico que no se matricularon en dicha titulación en los dos años siguientes y el número total de estudiantes que accedieron a esa misma titulación en ese mismo curso académico.
	\item \textbf{Tasa de eficiencia} representa el porcentaje entre el número total de créditos de la titulación a los que deberían haberse matriculado el conjunto de estudiantes graduados en un año académico y el número total de créditos que se matricularon finalmente.
	\item \textbf{Tasa de graduación} representa el porcentaje de estudiantes que finalizan una titulación en el tiempo previsto por el plan de estudios o en un año académico más y el número de estudiantes que entraron en esa misma titulación.
	\item \textbf{Tasa de rendimiento} representa el porcentaje entre el número total de créditos superados (menos los créditos adaptados, convalidados y reconocidos) por los estudiantes de una titulación y el número total de créditos matriculados.
	\item \textbf{Tipo de procedimiento} representa la forma de acceso por la que los estudiantes han realizado la solicitud de matrícula en la universidad.
	\item \textbf{Titulación} representa las titulaciones de grado que se ofertan en la universidad.
	\item \textbf{Titulados} representa el número de matrículas realizadas por el grupo de estudiantes que pertenece al cupo de titulados universitarios.
	\item \textbf{Universidad} Universidad compuesta por una serie de campus que a su vez se componen de diversos centros.
\end{itemize}

Con todos los atributos descritos ahora solo nos queda indicar cómo los vamos a referenciar en nuestra ontología y el tipo de dato que serán.

\begin{itemize}
	\item \textbf{Campus} $\rightarrow$ {\tt campus}. Tipo de dato cadena de caracteres.
	\item \textbf{Centro} $\rightarrow$ {\tt centro}. Tipo de dato cadena de caracteres.
	\item \textbf{Número medio de créditos matriculados} $\rightarrow$ {\tt creditosMatriculados}. Tipo de dato numérico real no negativo.
	\item \textbf{Número medio de créditos presentados} $\rightarrow$ {\tt creditosPresentados}. Tipo de dato numérico real no negativo.
	\item \textbf{Número medio de créditos superados} $\rightarrow$ {\tt creditosSuperados}. Tipo de dato numérico real no negativo.
	\item \textbf{Cupo general} $\rightarrow$ {\tt cupoGral}. Tipo de dato numérico entero no negativo.
	\item \textbf{Curso} $\rightarrow$ {\tt curso}. Tipo de dato cadena de caracteres.
	\item \textbf{Deportistas} $\rightarrow$ {\tt deportistas}. Tipo de dato numérico entero no negativo.
	\item \textbf{Discapacitados} $\rightarrow$ {\tt discapacitados}. Tipo de dato numérico entero no negativo.
	\item \textbf{Doctorado} $\rightarrow$ {\tt doctorado}. Tipo de dato cadena de caracteres.
	\item \textbf{Estado} $\rightarrow$ {\tt estado}. Tipo de dato cadena de caracteres.
	\item \textbf{Grado} $\rightarrow$ {\tt grado}. Tipo de dato cadena de caracteres.
	\item \textbf{Hombres} $\rightarrow$ {\tt hombres}. Tipo de dato numérico entero no negativo.
	\item \textbf{Máster} $\rightarrow$ {\tt master}. Tipo de dato cadena de caracteres.
	\item \textbf{Mayores 25} $\rightarrow$ {\tt mayor25}. Tipo de dato numérico entero no negativo.
	\item \textbf{Mayores 40 y 45 } $\rightarrow$ {\tt mayor40}. Tipo de dato numérico entero no negativo.
	\item \textbf{Mujeres} $\rightarrow$ {\tt mujeres}. Tipo de dato numérico entero no negativo.
	\item \textbf{País de origen} $\rightarrow$ {\tt pais}. Tipo de dato cadena de caracteres.
	\item \textbf{Personas} $\rightarrow$ {\tt personas}. Tipo de dato numérico entero no negativo.
	\item \textbf{Plan de estudios} $\rightarrow$ {\tt planEstudios}. Tipo de dato cadena de caracteres.
	\item \textbf{Plazas ofertadas} $\rightarrow$ {\tt plazasOfertadas}. Tipo de dato numérico entero no negativo.
	\item \textbf{Provincia} $\rightarrow$ {\tt provincia}. Tipo de dato cadena de caracteres.
	\item \textbf{Rama de conocimiento} $\rightarrow$ {\tt ramaConocimiento}. Tipo de dato cadena de caracteres.
	\item \textbf{Tasa de abandono} $\rightarrow$ {\tt tasaAbandono}. Tipo de dato numérico real no negativo.
	\item \textbf{Tasa de abandono inicial} $\rightarrow$ {\tt tasaAbandonoInicial}. Tipo de dato numérico real no negativo.
	\item \textbf{Tasa de eficiencia} $\rightarrow$ {\tt tasaEficiencia}. Tipo de dato numérico real no negativo.
	\item \textbf{Tasa de graduación} $\rightarrow$ {\tt tasaGraduacion}. Tipo de dato numérico real no negativo.
	\item \textbf{Tasa de rendimiento} $\rightarrow$ {\tt tasaRendimiento}. Tipo de dato numérico real no negativo.
	\item \textbf{Tipo de procedimiento} $\rightarrow$ {\tt tipoProcedimiento}. Tipo de dato cadena de caracteres.
	\item \textbf{Titulación} $\rightarrow$ {\tt titulacion}. Tipo de dato cadena de caracteres.
	\item \textbf{Titulados} $\rightarrow$ {\tt titulados}. Tipo de dato numérico entero no negativo
	\item \textbf{Universidad} $\rightarrow$ {\tt universidad}. Tipo de dato cadena de caracteres.
\end{itemize}

\newpage
\section{Vocabularios usados}

En el diseño de la ontología todas las clases son del lenguaje definido propio, sí que se han usado otros vocabularios para definir los metadatos de la ontología ({\sf Dublin Core}\footnote{\url{http://dublincore.org/documents/2012/06/14/dcmi-terms/}}, {\sf FOAF}\footnote{\url{http://xmlns.com/foaf/spec/}}, {\sf Creative Commons}\footnote{\url{https://creativecommons.org/ns}} y {\sf VANN}\footnote{\url{http://vocab.org/vann/}}). \bigskip

Para la definición del propio vocabulario (cuyo prefijo de espacio de nombres preferido es \textbf{ugr}), solo se han utilizado los estándares definidos {\sf RDF}, {\sf RDFS}, {\sf OWL} y {\sf XMLS}\footnote{\url{https://www.w3.org/2001/XMLSchema}} (para los tipos de datos); sin embargo, para conseguir el objetivo de conseguir disponer de datos enlazados en las diferentes propiedades se referencian propiedades equivalentes dentro de otros vocabularios de nivel superior como {\sf DBpedia}\footnote{\url{http://dbpedia.org/ontology/}} {\sf Wikidata}\footnote{\url{https://www.wikidata.org/wiki/}} o {\sf Scheme.org}\footnote{\url{http://schema.org/}}.

\section{Definición de las clases del sistema}

Las tablas 3.1 a 3.11 contienen la lista de propiedades de cada una de las clases, así como un ejemplo de una instancia de la clase en formato {\sf RDF/XML}. 

\section{Definición de las propiedades del sistema y sus restricciones}

Las tablas 3.12 a 3.41 contienen el dominio (las clases en las que se encuentran presentes), el rango (el tipo de dato), subpropiedad de (en caso de que lo sea), las propiedades equivalentes en otros vocabularios de nivel superior y su descripción dentro de la ontología en formato {\sf RDF/XML}.
\bigskip

\begin{table}[!ht]
	\centering
	\begin{tabular}{|p{.17\textwidth}|p{.9\textwidth}|}
		\hline
		\multicolumn{2}{|l|}{Clase: \textbf{DemandaAcademicaAcceso}}
		\\ \hline
		Propiedades:&
		\begin{itemize}
			\item tipoProcedimiento
			\item estado
			\item hombres
			\item mujeres
			\item curso
		\end{itemize}
		\\ \hline
		Ejemplo:&
		\textless rdf:Description rdf:about=``1213\#1"\textgreater \newline
		\tab \textless rdf:type rdf:resource=``\#DemandaAcademicaAcceso"\ /\textgreater \newline
		\tab \textless ugr:tipoProcedimiento\textgreater \newline\tab\tab CONVOCATORIA ORDINARIA DE JUNIO: PRUEBA DE ACCESO A LA UNIVERSIDAD PARA ESTUDIANTES PROVENIENTES DE BACHILLERATO Y DE CICLOS FORMATIVOS DE GRADO SUPERIOR \newline\tab\textless /ugr:tipoProcedimiento\textgreater \newline
		\tab \textless ugr:estado\textgreater \newline\tab\tab PRESENTADOS FASE GENERAL\newline\tab\textless /ugr:estado\textgreater  \newline
		\tab \textless ugr:hombres rdf:datatype=``\&xsd;nonNegativeInteger"\textgreater \newline\tab\tab2264 \newline \tab \textless /ugr:hombres\textgreater \newline
		\tab \textless ugr:mujeres rdf:datatype=``\&xsd;nonNegativeInteger"\textgreater \newline\tab\tab2877 \newline \tab \textless /ugr:mujeres\textgreater  \newline
		\tab \textless ugr:curso\textgreater \newline\tab\tab2012/2013\newline\tab\textless /ugr:curso\textgreater  \newline
		\textless /rdf:Description\textgreater 
		\\ \hline
	\end{tabular}
	\caption{Clase DemandaAcademicaAcceso}
	\label{clase-demandaacademicaacceso}
\end{table}

\begin{table}[!ht]
	\centering
	\begin{tabular}{|p{.17\textwidth}|p{.9\textwidth}|}
		\hline
		\multicolumn{2}{|l|}{Clase: \textbf{DemandaAcademicaTitulacion}}
		\\ \hline
		Propiedades:&
		\begin{itemize}
			\item titulacion
			\item plazasOfertadas
			\item cupoGral
			\item mayor25
			\item mayor40
			\item titulados
			\item discapacitados
			\item deportistas
			\item curso
		\end{itemize}
		\\ \hline
		Ejemplo:&
		\textless rdf:Description rdf:about=``1415\#1"\textgreater \newline
		\tab \textless rdf:type rdf:resource=``\#DemandaAcademicaTitulacion"\ /\textgreater \newline
		\tab \textless ugr:titulacion\textgreater \newline\tab\tab ADMINISTRACIÓN Y DIRECCIÓN DE EMPRESAS\newline\tab\textless /ugr:titulacion\textgreater \newline
		\tab \textless ugr:plazasOfertadas rdf:datatype=``\&xsd;nonNegativeInteger"\textgreater  \newline \tab \tab 281\newline\tab\textless /ugr:plazasOfertadas\textgreater \newline
		\tab \textless ugr:cupoGral rdf:datatype=``\&xsd;nonNegativeInteger"\textgreater \newline \tab \tab 271\newline\tab\textless /ugr:cupoGral\textgreater 
		\tab \newline \tab \textless ugr:mayor25 rdf:datatype=``\&xsd;nonNegativeInteger"\textgreater \newline \tab \tab 6\newline\tab\textless /ugr:mayor25\textgreater 
		\tab \newline \tab \textless ugr:mayor40 rdf:datatype=``\&xsd;nonNegativeInteger"\textgreater \newline \tab \tab 0\newline\tab\textless /ugr:mayor40\textgreater 
		\tab \newline \tab \textless ugr:titulados rdf:datatype=``\&xsd;nonNegativeInteger"\textgreater \newline \tab \tab 3\newline\tab\textless /ugr:titulados\textgreater 
		\tab \newline \tab 
		\textless ugr:discapacitados rdf:datatype=``\&xsd;nonNegativeInteger"\textgreater \newline \tab \tab 1\newline\tab\textless /ugr:discapacitados\textgreater 
		\tab \newline \tab \textless ugr:deportistas rdf:datatype=``\&xsd;nonNegativeInteger"\textgreater \newline \tab \tab 1\newline\tab\textless /ugr:deportistas\textgreater 
		\tab \newline \tab \textless ugr:curso\textgreater \newline\tab\tab2014/2015\newline\tab\textless /ugr:curso\textgreater \newline
		\textless /rdf:Description\textgreater 
		\\ \hline
	\end{tabular}
	\caption{Clase DemandaAcademicaTitulacion}
	\label{clase-demandaacademicatitulacion}
\end{table}

\begin{table}[!ht]
	\centering
	\begin{tabular}{|p{.17\textwidth}|p{.9\textwidth}|}
		\hline
		\multicolumn{2}{|l|}{Clase: \textbf{MatriculasGrado}}
		\\ \hline
		Propiedades:&
		\begin{itemize}
			\item ramaConocimiento
			\item titulacion
			\item hombres
			\item mujeres
			\item curso
		\end{itemize}
		\\ \hline
		Ejemplo:&
		\textless rdf:Description rdf:about=``1011\#1"\textgreater \newline
		\tab \textless rdf:type rdf:resource=``\#MatriculasGrado"\ /\textgreater \newline
		\tab \textless ugr:ramaConocimiento\textgreater \newline \tab\tab ARTES Y HUMANIDADES\newline\tab\textless /ugr:ramaConocimiento\textgreater \newline
		\tab \textless ugr:titulacion\textgreater \newline\tab\tab GRADO EN BELLAS ARTES\newline\tab\textless /ugr:titulacion\textgreater \newline
		\tab \textless ugr:hombres rdf:datatype=``\&xsd;nonNegativeInteger"\textgreater \newline\tab\tab70\newline\tab\textless /ugr:hombres\textgreater 
		\tab \newline\tab\textless ugr:mujeres rdf:datatype=``\&xsd;nonNegativeInteger"\textgreater \newline\tab\tab159\newline\tab\textless /ugr:mujeres\textgreater 
		\tab \newline\tab\textless ugr:curso\textgreater \newline\tab\tab2010/2011\newline\tab\textless /ugr:curso\textgreater \newline
		\textless /rdf:Description\textgreater 
		\\ \hline
	\end{tabular}
	\caption{Clase MatriculasGrado}
	\label{clase-matriculasgrado}
\end{table}

\begin{table}[!ht]
	\centering
	\begin{tabular}{|p{.17\textwidth}|p{.9\textwidth}|}
		\hline
		\multicolumn{2}{|l|}{Clase: \textbf{MatriculasPosgrado}}
		\\ \hline
		Propiedades:&
		\begin{itemize}
			\item titulacion
			\item hombres
			\item mujeres
			\item curso
		\end{itemize}
		\\ \hline
		Ejemplo:&
		\textless rdf:Description rdf:about=``1011\#1"\textgreater \newline
		\tab \textless rdf:type rdf:resource=``\#MatriculasPosgrado"\ /\textgreater \newline
		\tab \textless ugr:titulacion\textgreater \newline\tab\tab MASTER ERASMUS MUNDUS EN EL COLOR EN LA INFORMATICA Y LA TECNOLOGIA DE LOS MEDIOS / ERASMUS MUNDUS IN COLOR IN INFORMATICS AND MEDIA TECHNOLOGY (CIMET) \newline\tab\textless /ugr:titulacion\textgreater \newline
		\tab \textless ugr:hombres rdf:datatype=``\&xsd;nonNegativeInteger"\textgreater \newline\tab\tab9\newline\tab\textless /ugr:hombres\textgreater 
		\tab \newline\tab\textless ugr:mujeres rdf:datatype=``\&xsd;nonNegativeInteger"\textgreater \newline\tab\tab9\newline\tab\textless /ugr:mujeres\textgreater 
		\tab \newline\tab\textless ugr:curso\textgreater \newline\tab\tab2010/2011\newline\tab\textless /ugr:curso\textgreater \newline
		\textless /rdf:Description\textgreater 
		\\ \hline
	\end{tabular}
	\caption{Clase MatriculasPosgrado}
	\label{clase-matriculasposgrado}
\end{table}

\begin{table}[!ht]
	\centering
	\begin{tabular}{|p{.17\textwidth}|p{.9\textwidth}|}
		\hline
		\multicolumn{2}{|l|}{Clase: \textbf{NumMedioCreditos}}
		\\ \hline
		Propiedades:&
		\begin{itemize}
			\item planEstudios
			\item ramaConocimiento
			\item creditosMatriculados
			\item creditosPresentados
			\item creditosSuperados
			\item curso
		\end{itemize}
		\\ \hline
		Ejemplo:&
		\textless rdf:Description rdf:about=``1213\#1"\textgreater \newline
		\tab \textless rdf:type rdf:resource=``\#NumMedioCreditos"\ /\textgreater 
		\tab \newline\tab\textless ugr:planEstudios\textgreater \newline\tab\tab PRIMER/SEGUNDO CICLO\newline\tab\textless /ugr:planEstudios\textgreater 
		\tab \newline\tab\textless ugr:ramaConocimiento\textgreater \newline\tab\tab ARTES Y HUMANIDADES\newline\tab\textless /ugr:ramaConocimiento\textgreater 
		\newline \tab \textless ugr:creditosMatriculados rdf:datatype=``\&xsd;decimal"\textgreater \newline\tab\tab 52.40\newline\tab\textless /ugr:creditosMatriculados\textgreater \newline
		\tab \textless ugr:creditosPresentados rdf:datatype=``\&xsd;decimal"\textgreater \newline\tab\tab 40.06\newline\tab\textless /ugr:creditosPresentados\textgreater \newline
		\tab \textless ugr:creditosSuperados rdf:datatype=``\&xsd;decimal"\textgreater \newline\tab\tab 35.08\newline\tab\textless /ugr:creditosSuperados\textgreater \newline
		\tab \textless ugr:curso\textgreater \newline\tab\tab 2012/2013\newline\tab\textless /ugr:curso\textgreater 
		\newline\textless /rdf:Description\textgreater 
		\\ \hline
	\end{tabular}
	\caption{Clase NumMedioCreditos}
	\label{clase-nummediocreditos}
\end{table}

\begin{table}[!ht]
	\centering
	\begin{tabular}{|p{.17\textwidth}|p{.9\textwidth}|}
		\hline
		\multicolumn{2}{|l|}{Clase: \textbf{OfertaTitulacionDoctorado}}
		\\ \hline
		Propiedades:&
		\begin{itemize}
			\item ramaConocimiento
			\item titulacion
			\item campus
			\item centro
			\item curso
		\end{itemize}
		\\ \hline
		Ejemplo:&
		\textless rdf:Description rdf:about=``1314\#1"\textgreater \newline
		\tab \textless rdf:type rdf:resource=``\#OfertaTitulacionDoctorado"\ /\textgreater 
		\newline \tab \textless ugr:ramaConocimiento\textgreater \newline\tab\tab ARTES Y HUMANIDADES\newline\tab\textless /ugr:ramaConocimiento\textgreater 
		\newline\tab \textless ugr:titulacion\textgreater \newline\tab\tab PROGRAMA DE DOCTORADO EN BIOMEDICINA\newline\tab \textless/ugr:titulacion\textgreater 
		\newline\tab \textless ugr:campus\textgreater \newline\tab\tab ESCUELA DE DOCTORADO\newline\tab\textless /ugr:campus\textgreater 
		\newline\tab \textless ugr:centro\textgreater \newline\tab\tab UGR\newline\tab\textless /ugr:centro\textgreater 
		\newline\tab \textless ugr:curso\textgreater \newline\tab\tab 2013/2014\newline\tab\textless /ugr:curso\textgreater 
		\newline\textless /rdf:Description\textgreater 
		\\ \hline
	\end{tabular}
	\caption{Clase OfertaTitulacionDoctorado}
	\label{clase-ofertatitulaciondoctorado}
\end{table}

\begin{table}[!ht]
	\centering
	\begin{tabular}{|p{.17\textwidth}|p{.9\textwidth}|}
		\hline
		\multicolumn{2}{|l|}{Clase: \textbf{OfertaTitulacionGrado}}
		\\ \hline
		Propiedades:&
		\begin{itemize}
			\item ramaConocimiento
			\item titulacion
			\item campus
			\item centro
			\item curso
		\end{itemize}
		\\ \hline
		Ejemplo:&
		\textless rdf:Description rdf:about=``1314\#1"\textgreater \newline
		\tab \textless rdf:type rdf:resource=``\#OfertaTitulacionGrado"\ /\textgreater 
		\newline \tab \textless ugr:ramaConocimiento\textgreater \newline\tab\tab CIENCIAS\newline\tab\textless /ugr:ramaConocimiento\textgreater 
		\newline\tab \textless ugr:titulacion\textgreater \newline\tab\tab BIOLOGIA\newline\tab \textless/ugr:titulacion\textgreater 
		\newline\tab \textless ugr:campus\textgreater \newline\tab\tab FUENTENUEVA\newline\tab\textless /ugr:campus\textgreater 
		\newline\tab \textless ugr:centro\textgreater \newline\tab\tab FACULTAD DE CIENCIAS\newline\tab\textless /ugr:centro\textgreater 
		\newline\tab \textless ugr:curso\textgreater \newline\tab\tab 2013/2014\newline\tab\textless /ugr:curso\textgreater 
		\newline\textless /rdf:Description\textgreater 
		\\ \hline
	\end{tabular}
	\caption{Clase OfertaTitulacionGrado}
	\label{clase-ofertatitulaciongrado}
\end{table}

\begin{table}[!ht]
	\centering
	\begin{tabular}{|p{.17\textwidth}|p{.9\textwidth}|}
		\hline
		\multicolumn{2}{|l|}{Clase: \textbf{OfertaTitulacionMaster}}
		\\ \hline
		Propiedades:&
		\begin{itemize}
			\item ramaConocimiento
			\item titulacion
			\item campus
			\item centro
			\item curso
		\end{itemize}
		\\ \hline
		Ejemplo:&
		\textless rdf:Description rdf:about=``1314\#1"\textgreater \newline
		\tab \textless rdf:type rdf:resource=``\#OfertaTitulacionMaster"\ /\textgreater 
		\newline \tab \textless ugr:ramaConocimiento\textgreater \newline\tab\tab  ARTES Y HUMANIDADES\newline\tab\textless /ugr:ramaConocimiento\textgreater 
		\newline\tab \textless ugr:titulacion\textgreater \newline\tab\tab MASTER UNIVERSITARIO EN ARQUEOLOGIA (M71.56.1)\newline\tab \textless/ugr:titulacion\textgreater 
		\newline\tab \textless ugr:campus\textgreater \newline\tab\tab CARTUJA\newline\tab\textless /ugr:campus\textgreater 
		\newline\tab \textless ugr:centro\textgreater \newline\tab\tab FACULTAD DE FILOSOFÍA Y LETRAS\newline\tab\textless /ugr:centro\textgreater 
		\newline\tab \textless ugr:curso\textgreater \newline\tab\tab 2013/2014\newline\tab\textless /ugr:curso\textgreater 
		\newline\textless /rdf:Description\textgreater 
		\\ \hline
	\end{tabular}
	\caption{Clase OfertaTitulacionMaster}
	\label{clase-ofertatitulacionmaster}
\end{table}

\begin{table}[!ht]
	\centering
	\begin{tabular}{|p{.17\textwidth}|p{.9\textwidth}|}
		\hline
		\multicolumn{2}{|l|}{Clase: \textbf{OrigenPais}}
		\\ \hline
		Propiedades:&
		\begin{itemize}
			\item pais
			\item hombres
			\item mujeres
			\item curso
		\end{itemize}
		\\ \hline
		Ejemplo:&
		\textless rdf:Description rdf:about=``1314\#1"\textgreater 
		\tab \newline\tab \textless rdf:type rdf:resource=``\#OrigenPais"\ /\textgreater 
		\newline \tab \textless ugr:pais\textgreater \newline\tab\tab ALBANIA\newline\tab\textless /ugr:pais\textgreater 
		\newline\tab \textless ugr:hombres rdf:datatype=``\&xsd;nonNegativeInteger"\textgreater \newline\tab\tab 3\newline\tab\textless /ugr:hombres\textgreater 
		\newline\tab \textless ugr:mujeres rdf:datatype=``\&xsd;nonNegativeInteger"\textgreater \newline\tab\tab 2\newline\tab\textless /ugr:mujeres\textgreater 
		\newline\tab \textless ugr:curso\textgreater \newline\tab\tab 2013/2014\newline\tab\textless /ugr:curso\textgreater 
		\newline\textless /rdf:Description\textgreater 
		\\ \hline
	\end{tabular}
	\caption{Clase OrigenPais}
	\label{clase-origenpais}
\end{table}

\begin{table}[!ht]
	\centering
	\begin{tabular}{|p{.17\textwidth}|p{.9\textwidth}|}
		\hline
		\multicolumn{2}{|l|}{Clase: \textbf{OrigenProvincia}}
		\\ \hline
		Propiedades:&
		\begin{itemize}
			\item provincia
			\item hombres
			\item mujeres
			\item curso
		\end{itemize}
		\\ \hline
		Ejemplo:&
		\textless rdf:Description rdf:about=``1314\#1"\textgreater 
		\tab \newline\tab \textless rdf:type rdf:resource=``\#OrigenProvincia"\ /\textgreater 
		\newline \tab \textless ugr:pais\textgreater \newline\tab\tab ALAVA\newline\tab\textless /ugr:pais\textgreater 
		\newline\tab \textless ugr:hombres rdf:datatype=``\&xsd;nonNegativeInteger"\textgreater \newline\tab\tab 24\newline\tab\textless /ugr:hombres\textgreater 
		\newline\tab \textless ugr:mujeres rdf:datatype=``\&xsd;nonNegativeInteger"\textgreater \newline\tab\tab 47\newline\tab\textless /ugr:mujeres\textgreater 
		\newline\tab \textless ugr:curso\textgreater \newline\tab\tab 2013/2014\newline\tab\textless /ugr:curso\textgreater 
		\newline\textless /rdf:Description\textgreater 
		\\ \hline
	\end{tabular}
	\caption{Clase OrigenProvincia}
	\label{clase-origenprovincia}
\end{table}

\begin{table}[!ht]
	\centering
	\begin{tabular}{|p{.17\textwidth}|p{.9\textwidth}|}
		\hline
		\multicolumn{2}{|l|}{Clase: \textbf{TasasAcademicasTitulacion}}
		\\ \hline
		Propiedades:&
		\begin{itemize}
			\item titulacion
			\item tasaRendimiento
			\item tasaExito
			\item tasaAbandonoInicial
			\item tasaEficiencia
			\item tasaGraduacion
			\item tasaAbandono
		\end{itemize}
		\\ \hline
		Ejemplo:&
		\textless rdf:Description rdf:about=``1516\#1"\textgreater 
		\newline\tab \textless rdf:type rdf:resource=``\#TasasAcademicasTitulacion"\ /\textgreater 
		\newline\tab \textless ugr:titulacion\textgreater \newline\tab\tab GRADUADO EN BIOLOGIA\newline\tab\textless /ugr:titulacion\textgreater 
		\newline\tab \textless ugr:tasaRendimiento rdf:datatype=``\&xsd;decimal"\textgreater \newline\tab\tab 74.02\newline\tab\textless /ugr:tasaRendimiento\textgreater 
		\newline\tab \textless ugr:tasaExito rdf:datatype=``\&xsd;decimal"\textgreater \newline\tab\tab 83.12\newline\tab\textless /ugr:tasaExito\textgreater 
		\newline\tab \textless ugr:tasaAbandonoInicial rdf:datatype=``\&xsd;decimal"\textgreater \newline\tab\tab 12.27\newline\tab\textless /ugr:tasaAbandonoInicial\textgreater 
		\newline\tab \textless ugr:tasaEficiencia rdf:datatype=``\&xsd;decimal"\textgreater \newline\tab\tab 98\newline\tab\textless /ugr:tasaEficiencia\textgreater 
		\newline\tab \textless ugr:tasaGraduacion rdf:datatype=``\&xsd;decimal"\textgreater \newline\tab\tab 32.73\newline\tab\textless /ugr:tasaGraduacion\textgreater 
		\newline\tab \textless ugr:tasaAbandono rdf:datatype=``\&xsd;decimal"\textgreater \newline\tab\tab30\newline\tab\textless /ugr:tasaAbandono\textgreater 
		\newline\tab \textless ugr:curso\textgreater \newline\tab\tab 2015/2016\newline\tab\textless /ugr:curso\textgreater 
		\newline\textless /rdf:Description\textgreater 
		\\ \hline
	\end{tabular}
	\caption{Clase TasasAcademicasTitulacion}
	\label{clase-tasasacademicastitulacion}
\end{table}

\begin{table}[!ht]
	\centering
	\begin{tabular}{|p{.25\textwidth}|p{.9\textwidth}|}
		\hline
		\multicolumn{2}{|l|}{Propiedad: \textbf{campus}}
		\\ \hline
		Dominio:&
		\begin{itemize}
			\item OfertaTitulacionDoctorado
			\item OfertaTitulacionGrado
			\item OfertaTitulacionMaster
		\end{itemize}
		\\ \hline
		Rango:&
		http://www.w3.org/2001/XMLSchema\#string
		\\ \hline
		Subpropiedad de:&
		universidad
		\\ \hline
		Propiedades \newline equivalentes:&
		\begin{itemize}
			\item \url{http://dbpedia.org/ontology/campus}
			\item \url{https://www.wikidata.org/wiki/Q209465}
		\end{itemize}
		\\ \hline
		Descripción:&
		\textless owl:DatatypeProperty rdf:about=``\#campus"\textgreater\newline 
		\tab\textless rdfs:label xml:lang=``es"\textgreater\newline
		\tab\tab Campus\newline
		\tab\textless /rdfs:label\textgreater\newline
		\tab\textless rdfs:subPropertyOf\newline
		\tab\tab rdf:resource=``\#universidad"\ /\textgreater\newline
		\tab\textless rdfs:range\newline
		\tab\tab rdf:resource=``\&xsd;string"\ /\textgreater\newline
		\tab\textless rdfs:domain\newline
		\tab\tab rdf:resource=``\#OfertaTitulacionDoctorado"\ /\textgreater\newline
		\tab\textless rdfs:domain\newline
		\tab\tab rdf:resource=``\#OfertaTitulacionGrado"\ /\textgreater\newline
		\tab\textless rdfs:domain\newline
		\tab\tab rdf:resource=``\#OfertaTitulacionMaster"\ /\textgreater\newline
		\tab\textless owl:equivalentProperty\newline
		\tab\tab rdf:resource=``http://dbpedia.org/ontology/campus"\ /\textgreater\newline
		\tab\textless owl:equivalentProperty\newline
		\tab\tab rdf:resource=``https://www.wikidata.org/wiki/Q209465"\ /\textgreater\newline
		\textless /owl:DatatypeProperty\textgreater
		\\ \hline
	\end{tabular}
	\caption{Propiedad campus}
	\label{propiedad-campus}
\end{table}

\begin{table}[!ht]
	\centering
	\begin{tabular}{|p{.25\textwidth}|p{.9\textwidth}|}
		\hline
		\multicolumn{2}{|l|}{Propiedad: \textbf{centro}}
		\\ \hline
		Dominio:&
		\begin{itemize}
			\item OfertaTitulacionDoctorado
			\item OfertaTitulacionGrado
			\item OfertaTitulacionMaster
		\end{itemize}
		\\ \hline
		Rango:&
		http://www.w3.org/2001/XMLSchema\#string
		\\ \hline
		Subpropiedad de:&
		campus
		\\ \hline
		Descripción:&
		\textless owl:DatatypeProperty rdf:about=``\#centro"\textgreater\newline 
		\tab\textless rdfs:label xml:lang=``es"\textgreater\newline
		\tab\tab Centro\newline
		\tab\textless /rdfs:label\textgreater\newline
		\tab\textless rdfs:subPropertyOf\newline
		\tab\tab rdf:resource=``\#campus"\ /\textgreater\newline
		\tab\textless rdfs:range\newline
		\tab\tab rdf:resource=``\&xsd;string"\ /\textgreater\newline
		\tab\textless rdfs:domain\newline
		\tab\tab rdf:resource=``\#OfertaTitulacionDoctorado"\ /\textgreater\newline
		\tab\textless rdfs:domain\newline
		\tab\tab rdf:resource=``\#OfertaTitulacionGrado"\ /\textgreater\newline
		\tab\textless rdfs:domain\newline
		\tab\tab rdf:resource=``\#OfertaTitulacionMaster"\ /\textgreater\newline
		\textless /owl:DatatypeProperty\textgreater
		\\ \hline
	\end{tabular}
	\caption{Propiedad centro}
	\label{propiedad-centro}
\end{table}

\begin{table}[!ht]
	\centering
	\begin{tabular}{|p{.25\textwidth}|p{.9\textwidth}|}
		\hline
		\multicolumn{2}{|l|}{Propiedad: \textbf{creditosMatriculados}}
		\\ \hline
		Dominio:&
		\begin{itemize}
			\item NumMedioCreditos
		\end{itemize}
		\\ \hline
		Rango:&
		http://www.w3.org/2001/XMLSchema\#decimal
		\\ \hline
		Descripción:&
		\textless owl:DatatypeProperty rdf:about=``\#creditosMatriculados"\textgreater\newline 
		\tab\textless rdfs:label xml:lang=``es"\textgreater\newline
		\tab\tab Número medio de créditos matriculados\newline
		\tab\textless /rdfs:label\textgreater\newline
		\tab\textless rdfs:range\newline
		\tab\tab rdf:resource=``\&xsd;decimal"\ /\textgreater\newline
		\tab\textless owl:minCardinality \newline
		\tab\tab rdf:datatype=``\&xsd;decimal"\textgreater0\newline
		\tab\textless /owl:minCardinality\textgreater\newline
		\tab\textless rdfs:domain\newline
		\tab\tab rdf:resource=``\#NumMedioCreditos"\ /\textgreater\newline
		\textless /owl:DatatypeProperty\textgreater
		\\ \hline
	\end{tabular}
	\caption{Propiedad creditosMatriculados}
	\label{propiedad-creditosmatriculados}
\end{table}

\begin{table}[!ht]
	\centering
	\begin{tabular}{|p{.25\textwidth}|p{.9\textwidth}|}
		\hline
		\multicolumn{2}{|l|}{Propiedad: \textbf{creditosPresentados}}
		\\ \hline
		Dominio:&
		\begin{itemize}
			\item NumMedioCreditos
		\end{itemize}
		\\ \hline
		Rango:&
		http://www.w3.org/2001/XMLSchema\#decimal
		\\ \hline
		Descripción:&
		\textless owl:DatatypeProperty rdf:about=``\#creditosPresentados"\textgreater\newline 
		\tab\textless rdfs:label xml:lang=``es"\textgreater\newline
		\tab\tab Número medio de créditos presentados\newline
		\tab\textless /rdfs:label\textgreater\newline
		\tab\textless rdfs:range\newline
		\tab\tab rdf:resource=``\&xsd;decimal"\ /\textgreater\newline
		\tab\textless owl:minCardinality \newline
		\tab\tab rdf:datatype=``\&xsd;decimal"\textgreater0\newline
		\tab\textless /owl:minCardinality\textgreater\newline
		\tab\textless rdfs:domain\newline
		\tab\tab rdf:resource=``\#NumMedioCreditos"\ /\textgreater\newline
		\textless /owl:DatatypeProperty\textgreater
		\\ \hline
	\end{tabular}
	\caption{Propiedad creditosPresentados}
	\label{propiedad-creditospresentados}
\end{table}

\begin{table}[!ht]
	\centering
	\begin{tabular}{|p{.25\textwidth}|p{.9\textwidth}|}
		\hline
		\multicolumn{2}{|l|}{Propiedad: \textbf{creditosSuperados}}
		\\ \hline
		Dominio:&
		\begin{itemize}
			\item NumMedioCreditos
		\end{itemize}
		\\ \hline
		Rango:&
		http://www.w3.org/2001/XMLSchema\#decimal
		\\ \hline
		Descripción:&
		\textless owl:DatatypeProperty rdf:about=``\#creditosSuperados"\textgreater\newline 
		\tab\textless rdfs:label xml:lang=``es"\textgreater\newline
		\tab\tab Número medio de créditos superados\newline
		\tab\textless /rdfs:label\textgreater\newline
		\tab\textless rdfs:range\newline
		\tab\tab rdf:resource=``\&xsd;decimal"\ /\textgreater\newline
		\tab\textless owl:minCardinality \newline
		\tab\tab rdf:datatype=``\&xsd;decimal"\textgreater0\newline
		\tab\textless /owl:minCardinality\textgreater\newline
		\tab\textless rdfs:domain\newline
		\tab\tab rdf:resource=``\#NumMedioCreditos"\ /\textgreater\newline
		\textless /owl:DatatypeProperty\textgreater
		\\ \hline
	\end{tabular}
	\caption{Propiedad creditosSuperados}
	\label{propiedad-creditossuperados}
\end{table}

\begin{table}[!ht]
	\centering
	\begin{tabular}{|p{.25\textwidth}|p{.9\textwidth}|}
		\hline
		\multicolumn{2}{|l|}{Propiedad: \textbf{cupoGral}}
		\\ \hline
		Dominio:&
		\begin{itemize}
			\item DemandaAcademicaTitulacion
		\end{itemize}
		\\ \hline
		Rango:&
		http://www.w3.org/2001/XMLSchema\#decimal
		\\ \hline
		Descripción:&
		\textless owl:DatatypeProperty rdf:about=``\#cupoGral"\textgreater\newline 
		\tab\textless rdfs:label xml:lang=``es"\textgreater\newline
		\tab\tab Cupo general\newline
		\tab\textless /rdfs:label\textgreater\newline
		\tab\textless rdfs:range\newline
		\tab\tab rdf:resource=``\&xsd;decimal"\ /\textgreater\newline
		\tab\textless owl:minCardinality \newline
		\tab\tab rdf:datatype=``\&xsd;decimal"\textgreater0\newline
		\tab\textless /owl:minCardinality\textgreater\newline
		\tab\textless rdfs:domain\newline
		\tab\tab rdf:resource=``\#DemandaAcademicaTitulacion"\ /\textgreater\newline
		\textless /owl:DatatypeProperty\textgreater
		\\ \hline
	\end{tabular}
	\caption{Propiedad cupoGral}
	\label{propiedad-cupogral}
\end{table}

\begin{table}[!ht]
	\centering
	\begin{tabular}{|p{.25\textwidth}|p{.9\textwidth}|}
		\hline
		\multicolumn{2}{|l|}{Propiedad: \textbf{curso}}
		\\ \hline
		Dominio:&
		Todas las clases de ``ugr"
		\\ \hline
		Rango:&
		http://www.w3.org/2001/XMLSchema\#string
		\\ \hline
		Propiedades \newline equivalentes:&
		\begin{itemize}
			\item \url{http://purl.org/dc/terms/coverage}
		\end{itemize}
		\\ \hline
		Descripción:&
		\textless owl:DatatypeProperty rdf:about=``\#curso"\textgreater\newline 
		\tab\textless rdfs:label xml:lang=``es"\textgreater\newline
		\tab\tab Curso\newline
		\tab\textless /rdfs:label\textgreater\newline
		\tab\textless rdfs:range\newline
		\tab\tab rdf:resource=``\&xsd;string"\ /\textgreater\newline
		\tab\textless rdfs:domain\newline
		\tab\tab rdf:resource=``\#DemandaAcademicaAcceso"\ /\textgreater\newline
		\tab\textless rdfs:domain\newline
		\tab\tab rdf:resource=``\#DemandaAcademicaTitulacion"\ /\textgreater\newline
		\tab\textless rdfs:domain\newline
		\tab\tab rdf:resource=``\#MatriculasGrado"\ /\textgreater\newline
		\tab\textless rdfs:domain\newline
		\tab\tab rdf:resource=``\#MatriculasPosgrado"\ /\textgreater\newline
		\tab\textless rdfs:domain\newline
		\tab\tab rdf:resource=``\#NumMedioCreditos"\ /\textgreater\newline
		\tab\textless rdfs:domain\newline
		\tab\tab rdf:resource=``\#OfertaTitulacionDoctorado"\ /\textgreater\newline
		\tab\textless rdfs:domain\newline
		\tab\tab rdf:resource=``\#OfertaTitulacionGrado"\ /\textgreater\newline
		\tab\textless rdfs:domain\newline
		\tab\tab rdf:resource=``\#OfertaTitulacionMaster"\ /\textgreater\newline
		\tab\textless rdfs:domain\newline
		\tab\tab rdf:resource=``\#OrigenPais"\ /\textgreater\newline
		\tab\textless rdfs:domain\newline
		\tab\tab rdf:resource=``\#OrigenProvincia"\ /\textgreater\newline
		\tab\textless rdfs:domain\newline
		\tab\tab rdf:resource=``\#TasasAcademicasTitulacion"\ /\textgreater\newline
		\tab\textless owl:equivalentProperty\newline
		\tab\tab rdf:resource=``http://purl.org/dc/terms/coverage"\  /\textgreater\newline
		\textless /owl:DatatypeProperty\textgreater
		\\ \hline
	\end{tabular}
	\caption{Propiedad curso}
	\label{propiedad-curso}
\end{table}

\begin{table}[!ht]
	\centering
	\begin{tabular}{|p{.25\textwidth}|p{.9\textwidth}|}
		\hline
		\multicolumn{2}{|l|}{Propiedad: \textbf{deportistas}}
		\\ \hline
		Dominio:&
		\begin{itemize}
			\item DemandaAcademicaTitulacion
		\end{itemize}
		\\ \hline
		Rango:&
		http://www.w3.org/2001/XMLSchema\#nonNegativeInteger
		\\ \hline
		Propiedades \newline equivalentes:&
		\begin{itemize}
			\item \url{http://dbpedia.org/ontology/Athlete}
			\item \url{http://schema.org/athlete}
			\item \url{https://www.wikidata.org/wiki/Q2066131}
		\end{itemize}
		\\ \hline
		Descripción:&
		\textless owl:DatatypeProperty rdf:about=``\#deportistas"\textgreater\newline 
		\tab\textless rdfs:label xml:lang=``es"\textgreater\newline
		\tab\tab Deportistas\newline
		\tab\textless /rdfs:label\textgreater\newline
		\tab\textless rdfs:range\newline
		\tab\tab rdf:resource=``\&xsd;nonNegativeInteger"\ /\textgreater\newline
		\tab\textless rdfs:domain\newline
		\tab\tab rdf:resource=``\#DemandaAcademicaTitulacion"\ /\textgreater\newline
		\tab\textless owl:equivalentProperty\newline
		\tab\tab rdf:resource=``http://dbpedia.org/ontology/Athlete"\  /\textgreater\newline
		\tab\textless owl:equivalentProperty\newline
		\tab\tab rdf:resource=``http://schema.org/athlete"\  /\textgreater\newline
		\tab\textless owl:equivalentProperty\newline
		\tab\tab rdf:resource=``https://www.wikidata.org/wiki/Q2066131"\  /\textgreater\newline
		\textless /owl:DatatypeProperty\textgreater
		\\ \hline
	\end{tabular}
	\caption{Propiedad deportistas}
	\label{propiedad-deportistas}
\end{table}


\begin{table}[!ht]
	\centering
	\begin{tabular}{|p{.25\textwidth}|p{.9\textwidth}|}
		\hline
		\multicolumn{2}{|l|}{Propiedad: \textbf{discapacitados}}
		\\ \hline
		Dominio:&
		\begin{itemize}
			\item DemandaAcademicaTitulacion
		\end{itemize}
		\\ \hline
		Rango:&
		http://www.w3.org/2001/XMLSchema\#nonNegativeInteger
		\\ \hline
		Propiedades \newline equivalentes:&
		\begin{itemize}
			\item \url{https://www.wikidata.org/wiki/Q15978181}
		\end{itemize}
		\\ \hline
		Descripción:&
		\textless owl:DatatypeProperty rdf:about=``\#discapacitados"\textgreater\newline 
		\tab\textless rdfs:label xml:lang=``es"\textgreater\newline
		\tab\tab Discapacitados\newline
		\tab\textless /rdfs:label\textgreater\newline
		\tab\textless rdfs:range\newline
		\tab\tab rdf:resource=``\&xsd;nonNegativeInteger"\ /\textgreater\newline
		\tab\textless rdfs:domain\newline
		\tab\tab rdf:resource=``\#DemandaAcademicaTitulacion"\ /\textgreater\newline
		\tab\textless owl:equivalentProperty\newline
		\tab\tab rdf:resource=``https://www.wikidata.org/wiki/Q15978181"\  /\textgreater\newline
		\textless /owl:DatatypeProperty\textgreater
		\\ \hline
	\end{tabular}
	\caption{Propiedad discapacitados}
	\label{propiedad-discapacitados}
\end{table}

\begin{table}[!ht]
	\centering
	\begin{tabular}{|p{.25\textwidth}|p{.9\textwidth}|}
		\hline
		\multicolumn{2}{|l|}{Propiedad: \textbf{doctorado}}
		\\ \hline
		Dominio:&
		\begin{itemize}
			\item OfertaTitulacionDoctorado
		\end{itemize}
		\\ \hline
		Rango:&
		http://www.w3.org/2001/XMLSchema\#string
		\\ \hline
		Subpropiedad de:&
		titulacion
		\\ \hline
		Propiedades \newline equivalentes:&
		\begin{itemize}
			\item \url{http://dbpedia.org/page/Doctorate}
			\item \url{https://www.wikidata.org/wiki/Q849697}
		\end{itemize}
		\\ \hline
		Descripción:&
		\textless owl:DatatypeProperty rdf:about=``\#doctorado"\textgreater\newline 
		\tab\textless rdfs:label xml:lang=``es"\textgreater\newline
		\tab\tab Doctorado\newline
		\tab\textless /rdfs:label\textgreater\newline
		\tab\textless rdfs:subPropertyOf\newline
		\tab\tab rdf:resource=``\#titulacion"\ /\textgreater\newline
		\tab\textless rdfs:range\newline
		\tab\tab rdf:resource=``\&xsd;string"\ /\textgreater\newline
		\tab\textless rdfs:domain\newline
		\tab\tab rdf:resource=``\#OfertaTitulacionDoctorado"\ /\textgreater\newline
		\tab\textless owl:equivalentProperty\newline
		\tab\tab rdf:resource=``http://dbpedia.org/page/Doctorate"\  /\textgreater\newline
		\tab\textless owl:equivalentProperty\newline
		\tab\tab rdf:resource=``https://www.wikidata.org/wiki/Q849697"\  /\textgreater\newline
		\textless /owl:DatatypeProperty\textgreater
		\\ \hline
	\end{tabular}
	\caption{Propiedad doctorado}
	\label{propiedad-doctorado}
\end{table}

\begin{table}[!ht]
	\centering
	\begin{tabular}{|p{.25\textwidth}|p{.9\textwidth}|}
		\hline
		\multicolumn{2}{|l|}{Propiedad: \textbf{estado}}
		\\ \hline
		Dominio:&
		\begin{itemize}
			\item DemandaAcademicaAcceso
		\end{itemize}
		\\ \hline
		Rango:&
		http://www.w3.org/2001/XMLSchema\#string
		\\ \hline
		Descripción:&
		\textless owl:DatatypeProperty rdf:about=``\#estado"\textgreater\newline 
		\tab\textless rdfs:label xml:lang=``es"\textgreater\newline
		\tab\tab Estado\newline
		\tab\textless /rdfs:label\textgreater\newline
		\tab\textless rdfs:range\newline
		\tab\tab rdf:resource=``\&xsd;string"\ /\textgreater\newline
		\tab\textless rdfs:domain\newline
		\tab\tab rdf:resource=``\#DemandaAcademicaAcceso"\ /\textgreater\newline
		\textless /owl:DatatypeProperty\textgreater
		\\ \hline
	\end{tabular}
	\caption{Propiedad estado}
	\label{propiedad-estado}
\end{table}

\begin{table}[!ht]
	\centering
	\begin{tabular}{|p{.25\textwidth}|p{.9\textwidth}|}
		\hline
		\multicolumn{2}{|l|}{Propiedad: \textbf{grado}}
		\\ \hline
		Dominio:&
		\begin{itemize}
			\item DemandaAcademicaTitulacion
			\item MatriculasGrado
			\item OfertaTitulacionGrado
			\item TasasAcademicasTitulacion
		\end{itemize}
		\\ \hline
		Rango:&
		http://www.w3.org/2001/XMLSchema\#string
		\\ \hline
		Subpropiedad de:&
		titulacion
		\\ \hline
		Propiedades \newline equivalentes:&
		\begin{itemize}
			\item \url{http://dbpedia.org/page/Bachelor\%27s_degree}
			\item \url{https://www.wikidata.org/wiki/Q6008527}
		\end{itemize}
		\\ \hline
		Descripción:&
		\textless owl:DatatypeProperty rdf:about=``\#grado"\textgreater\newline 
		\tab\textless rdfs:label xml:lang=``es"\textgreater\newline
		\tab\tab Grado\newline
		\tab\textless /rdfs:label\textgreater\newline
		\tab\textless rdfs:subPropertyOf\newline
		\tab\tab rdf:resource=``\#titulacion"\ /\textgreater\newline
		\tab\textless rdfs:range\newline
		\tab\tab rdf:resource=``\&xsd;string"\ /\textgreater\newline
		\tab\textless rdfs:domain\newline
		\tab\tab rdf:resource=``\#DemandaAcademicaTitulacion"\ /\textgreater\newline
		\tab\textless rdfs:domain\newline
		\tab\tab rdf:resource=``\#MatriculasGrado"\ /\textgreater\newline
		\tab\textless rdfs:domain\newline
		\tab\tab rdf:resource=``\#OfertaTitulacionGrado"\ /\textgreater\newline
		\tab\textless rdfs:domain\newline
		\tab\tab rdf:resource=``\#TasasAcademicasTitulacion"\ /\textgreater\newline
		\tab\textless owl:equivalentProperty\newline
		\tab\tab rdf:resource=``http://dbpedia.org/page/Bachelor\%27s\_degree"\  /\textgreater\newline
		\tab\textless owl:equivalentProperty\newline
		\tab\tab rdf:resource=``https://www.wikidata.org/wiki/Q6008527"\  /\textgreater\newline
		\textless /owl:DatatypeProperty\textgreater
		\\ \hline
	\end{tabular}
	\caption{Propiedad grado}
	\label{propiedad-grado}
\end{table}

\begin{table}[!ht]
	\centering
	\begin{tabular}{|p{.25\textwidth}|p{.9\textwidth}|}
		\hline
		\multicolumn{2}{|l|}{Propiedad: \textbf{hombres}}
		\\ \hline
		Dominio:&
		\begin{itemize}
			\item DemandaAcademicaAcceso
			\item MatriculasGrado
			\item MatriculasPosgrado
			\item OrigenPais
			\item OrigenProvincia
		\end{itemize}
		\\ \hline
		Rango:&
		http://www.w3.org/2001/XMLSchema\#nonNegativeInteger
		\\ \hline
		Subpropiedad de:&
		personas
		\\ \hline
		Propiedades \newline equivalentes:&
		\begin{itemize}
			\item \url{http://dbpedia.org/page/Man}
			\item \url{https://schema.org/Male}
			\item \url{https://www.wikidata.org/wiki/Q8441}
		\end{itemize}
		\\ \hline
		Descripción:&
		\textless owl:DatatypeProperty rdf:about=``\#hombres"\textgreater\newline 
		\tab\textless rdfs:label xml:lang=``es"\textgreater\newline
		\tab\tab Hombres\newline
		\tab\textless /rdfs:label\textgreater\newline
		\tab\textless rdfs:subPropertyOf\newline
		\tab\tab rdf:resource=``\#personas"\ /\textgreater\newline
		\tab\textless rdfs:range\newline
		\tab\tab rdf:resource=``\&xsd;nonNegativeInteger"\ /\textgreater\newline
		\tab\textless rdfs:domain\newline
		\tab\tab rdf:resource=``\#DemandaAcademicaAcceso"\ /\textgreater\newline
		\tab\textless rdfs:domain\newline
		\tab\tab rdf:resource=``\#MatriculasGrado"\ /\textgreater\newline
		\tab\textless rdfs:domain\newline
		\tab\tab rdf:resource=``\#MatriculasPosgrado"\ /\textgreater\newline
		\tab\textless rdfs:domain\newline
		\tab\tab rdf:resource=``\#OrigenPais"\ /\textgreater\newline
		\tab\textless rdfs:domain\newline
		\tab\tab rdf:resource=``\#OrigenProvincia"\ /\textgreater\newline
		\tab\textless owl:equivalentProperty\newline
		\tab\tab rdf:resource=``http://dbpedia.org/page/Man"\  /\textgreater\newline
		\tab\textless owl:equivalentProperty\newline
		\tab\tab rdf:resource=``https://schema.org/Male"\  /\textgreater\newline
		\tab\textless owl:equivalentProperty\newline
		\tab\tab rdf:resource=``https://www.wikidata.org/wiki/Q8441"\  /\textgreater\newline
		\textless /owl:DatatypeProperty\textgreater
		\\ \hline
	\end{tabular}
	\caption{Propiedad hombres}
	\label{propiedad-hombres}
\end{table}

\begin{table}[!ht]
	\centering
	\begin{tabular}{|p{.25\textwidth}|p{.9\textwidth}|}
		\hline
		\multicolumn{2}{|l|}{Propiedad: \textbf{master}}
		\\ \hline
		Dominio:&
		\begin{itemize}
			\item MatriculasPosgrado
			\item OfertaTitulacionMaster
		\end{itemize}
		\\ \hline
		Rango:&
		http://www.w3.org/2001/XMLSchema\#string
		\\ \hline
		Subpropiedad de:&
		titulacion
		\\ \hline
		Propiedades \newline equivalentes:&
		\begin{itemize}
			\item \url{http://dbpedia-live.openlinksw.com/page/Master's_degree}
			\item \url{https://www.wikidata.org/wiki/Q183816}
		\end{itemize}
		\\ \hline
		Descripción:&
		\textless owl:DatatypeProperty rdf:about=``\#master"\textgreater\newline 
		\tab\textless rdfs:label xml:lang=``es"\textgreater\newline
		\tab\tab Máster\newline
		\tab\textless /rdfs:label\textgreater\newline
		\tab\textless rdfs:subPropertyOf\newline
		\tab\tab rdf:resource=``\#titulacion"\ /\textgreater\newline
		\tab\textless rdfs:range\newline
		\tab\tab rdf:resource=``\&xsd;string"\ /\textgreater\newline
		\tab\textless rdfs:domain\newline
		\tab\tab rdf:resource=``\#MatriculasPosgrado"\ /\textgreater\newline
		\tab\textless rdfs:domain\newline
		\tab\tab rdf:resource=``\#OfertaTitulacionMaster"\ /\textgreater\newline
		\tab\textless owl:equivalentProperty\newline
		\tab\tab rdf:resource=\newline\tab\tab``http://dbpedia-live.openlinksw.com/page/Master's\_degree"\  /\textgreater\newline
		\tab\textless owl:equivalentProperty\newline
		\tab\tab rdf:resource=``https://www.wikidata.org/wiki/Q183816"\  /\textgreater\newline
		\textless /owl:DatatypeProperty\textgreater
		\\ \hline
	\end{tabular}
	\caption{Propiedad master}
	\label{propiedad-master}
\end{table}

\begin{table}[!ht]
	\centering
	\begin{tabular}{|p{.25\textwidth}|p{.9\textwidth}|}
		\hline
		\multicolumn{2}{|l|}{Propiedad: \textbf{mayor25}}
		\\ \hline
		Dominio:&
		\begin{itemize}
			\item DemandaAcademicaTitulacion
		\end{itemize}
		\\ \hline
		Rango:&
		http://www.w3.org/2001/XMLSchema\#nonNegativeInteger
		\\ \hline
		Descripción:&
		\textless owl:DatatypeProperty rdf:about=``\#mayor25"\textgreater\newline 
		\tab\textless rdfs:label xml:lang=``es"\textgreater\newline
		\tab\tab Mayores de 25\newline
		\tab\textless /rdfs:label\textgreater\newline
		\tab\textless rdfs:range\newline
		\tab\tab rdf:resource=``\&xsd;nonNegativeInteger"\ /\textgreater\newline
		\tab\textless rdfs:domain\newline
		\tab\tab rdf:resource=``\#DemandaAcademicaTitulacion"\ /\textgreater\newline
		\textless /owl:DatatypeProperty\textgreater
		\\ \hline
	\end{tabular}
	\caption{Propiedad mayor25}
	\label{propiedad-mayor25}
\end{table}

\begin{table}[!ht]
	\centering
	\begin{tabular}{|p{.25\textwidth}|p{.9\textwidth}|}
		\hline
		\multicolumn{2}{|l|}{Propiedad: \textbf{mayor40}}
		\\ \hline
		Dominio:&
		\begin{itemize}
			\item DemandaAcademicaTitulacion
		\end{itemize}
		\\ \hline
		Rango:&
		http://www.w3.org/2001/XMLSchema\#nonNegativeInteger
		\\ \hline
		Descripción:&
		\textless owl:DatatypeProperty rdf:about=``\#mayor40"\textgreater\newline 
		\tab\textless rdfs:label xml:lang=``es"\textgreater\newline
		\tab\tab Mayores de 40 y 45\newline
		\tab\textless /rdfs:label\textgreater\newline
		\tab\textless rdfs:range\newline
		\tab\tab rdf:resource=``\&xsd;nonNegativeInteger"\ /\textgreater\newline
		\tab\textless rdfs:domain\newline
		\tab\tab rdf:resource=``\#DemandaAcademicaTitulacion"\ /\textgreater\newline
		\textless /owl:DatatypeProperty\textgreater
		\\ \hline
	\end{tabular}
	\caption{Propiedad mayor40}
	\label{propiedad-mayor40}
\end{table}

\begin{table}[!ht]
	\centering
	\begin{tabular}{|p{.25\textwidth}|p{.9\textwidth}|}
		\hline
		\multicolumn{2}{|l|}{Propiedad: \textbf{mujeres}}
		\\ \hline
		Dominio:&
		\begin{itemize}
			\item DemandaAcademicaAcceso
			\item MatriculasGrado
			\item MatriculasPosgrado
			\item OrigenPais
			\item OrigenProvincia
		\end{itemize}
		\\ \hline
		Rango:&
		http://www.w3.org/2001/XMLSchema\#nonNegativeInteger
		\\ \hline
		Subpropiedad de:&
		personas
		\\ \hline
		Propiedades \newline equivalentes:&
		\begin{itemize}
			\item \url{http://dbpedia.org/page/Woman}
			\item \url{https://schema.org/Female}
			\item \url{https://www.wikidata.org/wiki/Q467}
		\end{itemize}
		\\ \hline
		Descripción:&
		\textless owl:DatatypeProperty rdf:about=``\#mujeres"\textgreater\newline 
		\tab\textless rdfs:label xml:lang=``es"\textgreater\newline
		\tab\tab Mujeres\newline
		\tab\textless /rdfs:label\textgreater\newline
		\tab\textless rdfs:subPropertyOf\newline
		\tab\tab rdf:resource=``\#personas"\ /\textgreater\newline
		\tab\textless rdfs:range\newline
		\tab\tab rdf:resource=``\&xsd;nonNegativeInteger"\ /\textgreater\newline
		\tab\textless rdfs:domain\newline
		\tab\tab rdf:resource=``\#DemandaAcademicaAcceso"\ /\textgreater\newline
		\tab\textless rdfs:domain\newline
		\tab\tab rdf:resource=``\#MatriculasGrado"\ /\textgreater\newline
		\tab\textless rdfs:domain\newline
		\tab\tab rdf:resource=``\#MatriculasPosgrado"\ /\textgreater\newline
		\tab\textless rdfs:domain\newline
		\tab\tab rdf:resource=``\#OrigenPais"\ /\textgreater\newline
		\tab\textless rdfs:domain\newline
		\tab\tab rdf:resource=``\#OrigenProvincia"\ /\textgreater\newline
		\tab\textless owl:equivalentProperty\newline
		\tab\tab rdf:resource=``http://dbpedia.org/page/Woman"\  /\textgreater\newline
		\tab\textless owl:equivalentProperty\newline
		\tab\tab rdf:resource=``https://schema.org/Female"\  /\textgreater\newline
		\tab\textless owl:equivalentProperty\newline
		\tab\tab rdf:resource=``https://www.wikidata.org/wiki/Q467"\  /\textgreater\newline
		\textless /owl:DatatypeProperty\textgreater
		\\ \hline
	\end{tabular}
	\caption{Propiedad mujeres}
	\label{propiedad-mujeres}
\end{table}

\begin{table}[!ht]
	\centering
	\begin{tabular}{|p{.25\textwidth}|p{.9\textwidth}|}
		\hline
		\multicolumn{2}{|l|}{Propiedad: \textbf{pais}}
		\\ \hline
		Dominio:&
		\begin{itemize}
			\item OrigenPais
		\end{itemize}
		\\ \hline
		Rango:&
		http://www.w3.org/2001/XMLSchema\#string
		\\ \hline
		Propiedades \newline equivalentes:&
		\begin{itemize}
			\item \url{http://dbpedia.org/ontology/country}
			\item \url{http://schema.org/Country}
			\item \url{https://www.wikidata.org/wiki/Q6256}
		\end{itemize}
		\\ \hline
		Descripción:&
		\textless owl:DatatypeProperty rdf:about=``\#pais"\textgreater\newline 
		\tab\textless rdfs:label xml:lang=``es"\textgreater\newline
		\tab\tab Pais de origen\newline
		\tab\textless /rdfs:label\textgreater\newline
		\tab\textless rdfs:range\newline
		\tab\tab rdf:resource=``\&xsd;string"\ /\textgreater\newline
		\tab\textless rdfs:domain\newline
		\tab\tab rdf:resource=``\#OrigenPais"\ /\textgreater\newline
		\tab\textless owl:equivalentProperty\newline
		\tab\tab rdf:resource=``http://dbpedia.org/ontology/country"\  /\textgreater\newline
		\tab\textless owl:equivalentProperty\newline
		\tab\tab rdf:resource=``http://schema.org/Country"\  /\textgreater\newline
		\tab\textless owl:equivalentProperty\newline
		\tab\tab rdf:resource=``https://www.wikidata.org/wiki/Q6256"\  /\textgreater\newline
		\textless /owl:DatatypeProperty\textgreater
		\\ \hline
	\end{tabular}
	\caption{Propiedad pais}
	\label{propiedad-pais}
\end{table}

\begin{table}[!ht]
	\centering
	\begin{tabular}{|p{.25\textwidth}|p{.9\textwidth}|}
		\hline
		\multicolumn{2}{|l|}{Propiedad: \textbf{personas}}
		\\ \hline
		Rango:&
		http://www.w3.org/2001/XMLSchema\#nonNegativeInteger
		\\ \hline
		Propiedades \newline equivalentes:&
		\begin{itemize}
			\item \url{http://dbpedia.org/ontology/person}
			\item \url{http://schema.org/Person}
			\item \url{https://www.wikidata.org/wiki/Q215627}
		\end{itemize}
		\\ \hline
		Descripción:&
		\textless owl:DatatypeProperty rdf:about=``\#personas"\textgreater\newline 
		\tab\textless rdfs:label xml:lang=``es"\textgreater\newline
		\tab\tab Personas\newline
		\tab\textless /rdfs:label\textgreater\newline
		\tab\textless rdfs:range\newline
		\tab\tab rdf:resource=``\&xsd;nonNegativeInteger"\ /\textgreater\newline
		\tab\textless owl:equivalentProperty\newline
		\tab\tab rdf:resource=``http://dbpedia.org/ontology/person"\  /\textgreater\newline
		\tab\textless owl:equivalentProperty\newline
		\tab\tab rdf:resource=``http://schema.org/Person"\  /\textgreater\newline
		\tab\textless owl:equivalentProperty\newline
		\tab\tab rdf:resource=``https://www.wikidata.org/wiki/Q215627"\  /\textgreater\newline
		\textless /owl:DatatypeProperty\textgreater
		\\ \hline
	\end{tabular}
	\caption{Propiedad personas}
	\label{propiedad-personas}
\end{table}

\begin{table}[!ht]
	\centering
	\begin{tabular}{|p{.25\textwidth}|p{.9\textwidth}|}
		\hline
		\multicolumn{2}{|l|}{Propiedad: \textbf{planEstudios}}
		\\ \hline
		Dominio:&
		\begin{itemize}
			\item NumMedioCreditos
		\end{itemize}
		\\ \hline
		Rango:&
		http://www.w3.org/2001/XMLSchema\#string
		\\ \hline
		Descripción:&
		\textless owl:DatatypeProperty rdf:about=``\#planEstudios"\textgreater\newline 
		\tab\textless rdfs:label xml:lang=``es"\textgreater\newline
		\tab\tab Plan de estudios\newline
		\tab\textless /rdfs:label\textgreater\newline
		\tab\textless rdfs:range\newline
		\tab\tab rdf:resource=``\&xsd;string"\ /\textgreater\newline
		\tab\textless rdfs:domain\newline
		\tab\tab rdf:resource=``\#NumMedioCreditos"\ /\textgreater\newline
		\textless /owl:DatatypeProperty\textgreater
		\\ \hline
	\end{tabular}
	\caption{Propiedad planEstudios}
	\label{propiedad-planestudios}
\end{table}

\begin{table}[!ht]
	\centering
	\begin{tabular}{|p{.25\textwidth}|p{.9\textwidth}|}
		\hline
		\multicolumn{2}{|l|}{Propiedad: \textbf{plazasOfertadas}}
		\\ \hline
		Dominio:&
		\begin{itemize}
			\item DemandaAcademicaTitulacion
		\end{itemize}
		\\ \hline
		Rango:&
		http://www.w3.org/2001/XMLSchema\#nonNegativeInteger
		\\ \hline
		Descripción:&
		\textless owl:DatatypeProperty rdf:about=``\#plazasOfertadas"\textgreater\newline 
		\tab\textless rdfs:label xml:lang=``es"\textgreater\newline
		\tab\tab Plazas ofertadas\newline
		\tab\textless /rdfs:label\textgreater\newline
		\tab\textless rdfs:range\newline
		\tab\tab rdf:resource=``\&xsd;nonNegativeInteger"\ /\textgreater\newline
		\tab\textless rdfs:domain\newline
		\tab\tab rdf:resource=``\#DemandaAcademicaTitulacion"\ /\textgreater\newline
		\textless /owl:DatatypeProperty\textgreater
		\\ \hline
	\end{tabular}
	\caption{Propiedad plazasOfertadas}
	\label{propiedad-plazasOfertadas}
\end{table}

\begin{table}[!ht]
	\centering
	\begin{tabular}{|p{.25\textwidth}|p{.9\textwidth}|}
		\hline
		\multicolumn{2}{|l|}{Propiedad: \textbf{provincia}}
		\\ \hline
		Dominio:&
		\begin{itemize}
			\item OrigenProvincia
		\end{itemize}
		\\ \hline
		Rango:&
		http://www.w3.org/2001/XMLSchema\#string
		\\ \hline
		Propiedades \newline equivalentes:&
		\begin{itemize}
			\item \url{http://dbpedia.org/ontology/province}
			\item \url{http://schema.org/State}
			\item \url{https://www.wikidata.org/wiki/Q34876}
		\end{itemize}
		\\ \hline
		Descripción:&
		\textless owl:DatatypeProperty rdf:about=``\#provincia"\textgreater\newline 
		\tab\textless rdfs:label xml:lang=``es"\textgreater\newline
		\tab\tab Provincia\newline
		\tab\textless /rdfs:label\textgreater\newline
		\tab\textless rdfs:range\newline
		\tab\tab rdf:resource=``\&xsd;string"\ /\textgreater\newline
		\tab\textless rdfs:domain\newline
		\tab\tab rdf:resource=``\#OrigenProvincia"\ /\textgreater\newline
		\tab\textless owl:equivalentProperty\newline
		\tab\tab rdf:resource=``http://dbpedia.org/ontology/province"\  /\textgreater\newline
		\tab\textless owl:equivalentProperty\newline
		\tab\tab rdf:resource=``http://schema.org/State"\  /\textgreater\newline
		\tab\textless owl:equivalentProperty\newline
		\tab\tab rdf:resource=``https://www.wikidata.org/wiki/Q34876"\  /\textgreater\newline
		\textless /owl:DatatypeProperty\textgreater
		\\ \hline
	\end{tabular}
	\caption{Propiedad provincia}
	\label{propiedad-provincia}
\end{table}

\begin{table}[!ht]
	\centering
	\begin{tabular}{|p{.25\textwidth}|p{.9\textwidth}|}
		\hline
		\multicolumn{2}{|l|}{Propiedad: \textbf{ramaConocimiento}}
		\\ \hline
		Dominio:&
		\begin{itemize}
			\item MatriculasGrado
			\item MatriculasPosgrado
			\item NumMedioCreditos
			\item OfertaTitulacionDoctorado
			\item OfertaTitulacionGrado
			\item OfertaTitulacionMaster
		\end{itemize}
		\\ \hline
		Rango:&
		http://www.w3.org/2001/XMLSchema\#string
		\\ \hline
		Descripción:&
		\textless owl:DatatypeProperty rdf:about=``\#ramaConocimiento"\textgreater\newline 
		\tab\textless rdfs:label xml:lang=``es"\textgreater\newline
		\tab\tab Rama de conocimiento\newline
		\tab\textless /rdfs:label\textgreater\newline
		\tab\textless rdfs:range\newline
		\tab\tab rdf:resource=``\&xsd;string"\ /\textgreater\newline
		\tab\textless rdfs:domain\newline
		\tab\tab rdf:resource=``\#MatriculasGrado"\ /\textgreater\newline
		\tab\textless rdfs:domain\newline
		\tab\tab rdf:resource=``\#MatriculasPosgrado"\ /\textgreater\newline
		\tab\textless rdfs:domain\newline
		\tab\tab rdf:resource=``\#NumMedioCreditos"\ /\textgreater\newline
		\tab\textless rdfs:domain\newline
		\tab\tab rdf:resource=``\#OfertaTitulacionDoctorado"\ /\textgreater\newline
		\tab\textless rdfs:domain\newline
		\tab\tab rdf:resource=``\#OfertaTitulacionGrado"\ /\textgreater\newline
		\tab\textless rdfs:domain\newline
		\tab\tab rdf:resource=``\#OfertaTitulacionMaster"\ /\textgreater\newline
		\textless /owl:DatatypeProperty\textgreater
		\\ \hline
	\end{tabular}
	\caption{Propiedad ramaConocimiento}
	\label{propiedad-ramaconocimiento}
\end{table}

\begin{table}[!ht]
	\centering
	\begin{tabular}{|p{.25\textwidth}|p{.9\textwidth}|}
		\hline
		\multicolumn{2}{|l|}{Propiedad: \textbf{tasaAbandono}}
		\\ \hline
		Dominio:&
		\begin{itemize}
			\item TasasAcademicasTitulacion
		\end{itemize}
		\\ \hline
		Rango:&
		http://www.w3.org/2001/XMLSchema\#decimal
		\\ \hline
		Descripción:&
		\textless owl:DatatypeProperty rdf:about=``\#tasaAbandono"\textgreater\newline 
		\tab\textless rdfs:label xml:lang=``es"\textgreater\newline
		\tab\tab Tasa de abandono\newline
		\tab\textless /rdfs:label\textgreater\newline
		\tab\textless rdfs:range\newline
		\tab\tab rdf:resource=``\&xsd;decimal"\ /\textgreater\newline
		\tab\textless owl:minCardinality \newline
		\tab\tab rdf:datatype=``\&xsd;decimal"\textgreater0\newline
		\tab\textless /owl:minCardinality\textgreater\newline
		\tab\textless rdfs:domain\newline
		\tab\tab rdf:resource=``\#TasasAcademicasTitulacion"\ /\textgreater\newline
		\textless /owl:DatatypeProperty\textgreater
		\\ \hline
	\end{tabular}
	\caption{Propiedad tasaAbandono}
	\label{propiedad-tasaabandono}
\end{table}

\begin{table}[!ht]
	\centering
	\begin{tabular}{|p{.25\textwidth}|p{.9\textwidth}|}
		\hline
		\multicolumn{2}{|l|}{Propiedad: \textbf{tasaAbandonoInicial}}
		\\ \hline
		Dominio:&
		\begin{itemize}
			\item TasasAcademicasTitulacion
		\end{itemize}
		\\ \hline
		Rango:&
		http://www.w3.org/2001/XMLSchema\#decimal
		\\ \hline
		Descripción:&
		\textless owl:DatatypeProperty rdf:about=``\#tasaAbandonoInicial"\textgreater\newline 
		\tab\textless rdfs:label xml:lang=``es"\textgreater\newline
		\tab\tab Tasa de abandono inicial\newline
		\tab\textless /rdfs:label\textgreater\newline
		\tab\textless rdfs:range\newline
		\tab\tab rdf:resource=``\&xsd;decimal"\ /\textgreater\newline
		\tab\textless owl:minCardinality \newline
		\tab\tab rdf:datatype=``\&xsd;decimal"\textgreater0\newline
		\tab\textless /owl:minCardinality\textgreater\newline
		\tab\textless rdfs:domain\newline
		\tab\tab rdf:resource=``\#TasasAcademicasTitulacion"\ /\textgreater\newline
		\textless /owl:DatatypeProperty\textgreater
		\\ \hline
	\end{tabular}
	\caption{Propiedad tasaAbandonoInicial}
	\label{propiedad-tasaabandonoinicial}
\end{table}

\begin{table}[!ht]
	\centering
	\begin{tabular}{|p{.25\textwidth}|p{.9\textwidth}|}
		\hline
		\multicolumn{2}{|l|}{Propiedad: \textbf{tasaEficiencia}}
		\\ \hline
		Dominio:&
		\begin{itemize}
			\item TasasAcademicasTitulacion
		\end{itemize}
		\\ \hline
		Rango:&
		http://www.w3.org/2001/XMLSchema\#decimal
		\\ \hline
		Descripción:&
		\textless owl:DatatypeProperty rdf:about=``\#tasaEficiencia"\textgreater\newline 
		\tab\textless rdfs:label xml:lang=``es"\textgreater\newline
		\tab\tab Tasa de eficiencia\newline
		\tab\textless /rdfs:label\textgreater\newline
		\tab\textless rdfs:range\newline
		\tab\tab rdf:resource=``\&xsd;decimal"\ /\textgreater\newline
		\tab\textless owl:minCardinality \newline
		\tab\tab rdf:datatype=``\&xsd;decimal"\textgreater0\newline
		\tab\textless /owl:minCardinality\textgreater\newline
		\tab\textless rdfs:domain\newline
		\tab\tab rdf:resource=``\#TasasAcademicasTitulacion"\ /\textgreater\newline
		\textless /owl:DatatypeProperty\textgreater
		\\ \hline
	\end{tabular}
	\caption{Propiedad tasaEficiencia}
	\label{propiedad-tasaeficiencia}
\end{table}

\begin{table}[!ht]
	\centering
	\begin{tabular}{|p{.25\textwidth}|p{.9\textwidth}|}
		\hline
		\multicolumn{2}{|l|}{Propiedad: \textbf{tasaGraduacion}}
		\\ \hline
		Dominio:&
		\begin{itemize}
			\item TasasAcademicasTitulacion
		\end{itemize}
		\\ \hline
		Rango:&
		http://www.w3.org/2001/XMLSchema\#decimal
		\\ \hline
		Descripción:&
		\textless owl:DatatypeProperty rdf:about=``\#tasaGraduacion"\textgreater\newline 
		\tab\textless rdfs:label xml:lang=``es"\textgreater\newline
		\tab\tab Tasa de graduación\newline
		\tab\textless /rdfs:label\textgreater\newline
		\tab\textless rdfs:range\newline
		\tab\tab rdf:resource=``\&xsd;decimal"\ /\textgreater\newline
		\tab\textless owl:minCardinality \newline
		\tab\tab rdf:datatype=``\&xsd;decimal"\textgreater0\newline
		\tab\textless /owl:minCardinality\textgreater\newline
		\tab\textless rdfs:domain\newline
		\tab\tab rdf:resource=``\#TasasAcademicasTitulacion"\ /\textgreater\newline
		\textless /owl:DatatypeProperty\textgreater
		\\ \hline
	\end{tabular}
	\caption{Propiedad tasaGraduacion}
	\label{propiedad-tasagraduacion}
\end{table}

\begin{table}[!ht]
	\centering
	\begin{tabular}{|p{.25\textwidth}|p{.9\textwidth}|}
		\hline
		\multicolumn{2}{|l|}{Propiedad: \textbf{tasaRendimiento}}
		\\ \hline
		Dominio:&
		\begin{itemize}
			\item TasasAcademicasTitulacion
		\end{itemize}
		\\ \hline
		Rango:&
		http://www.w3.org/2001/XMLSchema\#decimal
		\\ \hline
		Descripción:&
		\textless owl:DatatypeProperty rdf:about=``\#tasaRendimiento"\textgreater\newline 
		\tab\textless rdfs:label xml:lang=``es"\textgreater\newline
		\tab\tab Tasa de rendimiento\newline
		\tab\textless /rdfs:label\textgreater\newline
		\tab\textless rdfs:range\newline
		\tab\tab rdf:resource=``\&xsd;decimal"\ /\textgreater\newline
		\tab\textless owl:minCardinality \newline
		\tab\tab rdf:datatype=``\&xsd;decimal"\textgreater0\newline
		\tab\textless /owl:minCardinality\textgreater\newline
		\tab\textless rdfs:domain\newline
		\tab\tab rdf:resource=``\#TasasAcademicasTitulacion"\ /\textgreater\newline
		\textless /owl:DatatypeProperty\textgreater
		\\ \hline
	\end{tabular}
	\caption{Propiedad tasaRendimiento}
	\label{propiedad-tasarendimiento}
\end{table}

\begin{table}[!ht]
	\centering
	\begin{tabular}{|p{.25\textwidth}|p{.9\textwidth}|}
		\hline
		\multicolumn{2}{|l|}{Propiedad: \textbf{tipoProcedimiento}}
		\\ \hline
		Dominio:&
		\begin{itemize}
			\item DemandaAcademicaAcceso
		\end{itemize}
		\\ \hline
		Rango:&
		http://www.w3.org/2001/XMLSchema\#string
		\\ \hline
		Descripción:&
		\textless owl:DatatypeProperty rdf:about=``\#tipoProcedimiento"\textgreater\newline 
		\tab\textless rdfs:label xml:lang=``es"\textgreater\newline
		\tab\tab Tipo de procedimiento\newline
		\tab\textless /rdfs:label\textgreater\newline
		\tab\textless rdfs:subPropertyOf\newline
		\tab\tab rdf:resource=``\#tipoProcedimiento"\ /\textgreater\newline
		\tab\textless rdfs:range\newline
		\tab\tab rdf:resource=``\&xsd;string"\ /\textgreater\newline
		\tab\textless rdfs:domain\newline
		\tab\tab rdf:resource=``\#DemandaAcademicaAcceso"\ /\textgreater\newline
		\textless /owl:DatatypeProperty\textgreater
		\\ \hline
	\end{tabular}
	\caption{Propiedad tipoProcedimiento}
	\label{propiedad-tipoprocedimiento}
\end{table}

\begin{table}[!ht]
	\centering
	\begin{tabular}{|p{.25\textwidth}|p{.9\textwidth}|}
		\hline
		\multicolumn{2}{|l|}{Propiedad: \textbf{titulacion}}
		\\ \hline
		Rango:&
		http://www.w3.org/2001/XMLSchema\#string
		\\ \hline
		Subpropiedad de:&
		ramaConocimiento
		\\ \hline
		Propiedades \newline equivalentes:&
		\begin{itemize}
			\item \url{http://dbpedia.org/page/Academic_degree}
			\item \url{https://www.wikidata.org/wiki/Q189533}
		\end{itemize}
		\\ \hline
		Descripción:&
		\textless owl:DatatypeProperty rdf:about=``\#titulacion"\textgreater\newline
		\tab\textless rdfs:label xml:lang=``es"\textgreater\newline
		\tab\tab Titulacion\newline
		\tab\textless /rdfs:label\textgreater\newline
		\tab\textless rdfs:subPropertyOf\newline
		\tab\tab rdf:resource=``\#ramaConocimiento"\ /\textgreater\newline
		\tab\textless rdfs:range\newline
		\tab\tab rdf:resource=``\&xsd;string"\ /\textgreater\newline
		\tab\textless owl:equivalentProperty\newline
		\tab\tab rdf:resource=``http://dbpedia.org/page/Academic\_degree"\  /\textgreater\newline
		\tab\textless owl:equivalentProperty\newline
		\tab\tab rdf:resource=``https://www.wikidata.org/wiki/Q189533"\  /\textgreater\newline
		\textless /owl:DatatypeProperty\textgreater
		\\ \hline
	\end{tabular}
	\caption{Propiedad titulacion}
	\label{propiedad-titulacion}
\end{table}

\begin{table}[!ht]
	\centering
	\begin{tabular}{|p{.25\textwidth}|p{.9\textwidth}|}
		\hline
		\multicolumn{2}{|l|}{Propiedad: \textbf{titulados}}
		\\ \hline
		Dominio:&
		\begin{itemize}
			\item DemandaAcademicaTitulacion
		\end{itemize}
		\\ \hline
		Rango:&
		http://www.w3.org/2001/XMLSchema\#nonNegativeInteger
		\\ \hline
		Descripción:&
		\textless owl:DatatypeProperty rdf:about=``\#titulados"\textgreater\newline 
		\tab\textless rdfs:label xml:lang=``es"\textgreater\newline
		\tab\tab Titulados\newline
		\tab\textless /rdfs:label\textgreater\newline
		\tab\textless rdfs:range\newline
		\tab\tab rdf:resource=``\&xsd;nonNegativeInteger"\ /\textgreater\newline
		\tab\textless rdfs:domain\newline
		\tab\tab rdf:resource=``\#DemandaAcademicaTitulacion"\ /\textgreater\newline
		\textless /owl:DatatypeProperty\textgreater
		\\ \hline
	\end{tabular}
	\caption{Propiedad titulados}
	\label{propiedad-titulados}
\end{table}

\begin{table}[!ht]
	\centering
	\begin{tabular}{|p{.25\textwidth}|p{.9\textwidth}|}
		\hline
		\multicolumn{2}{|l|}{Propiedad: \textbf{universidad}}
		\\ \hline
		Propiedades \newline equivalentes:&
		\begin{itemize}
			\item \url{http://dbpedia.org/ontology/university}
			\item \url{http://schema.org/CollegeOrUniversity}
			\item \url{https://www.wikidata.org/wiki/Q3918}
		\end{itemize}
		\\ \hline
		Rango:&
		http://www.w3.org/2001/XMLSchema\#string
		\\ \hline
		Descripción:&
		\textless owl:DatatypeProperty rdf:about=``\#universidad"\textgreater\newline 
		\tab\textless rdfs:label xml:lang=``es"\textgreater\newline
		\tab\tab Titulados\newline
		\tab\textless /rdfs:label\textgreater\newline
		\tab\textless rdfs:range\newline
		\tab\tab rdf:resource=``\&xsd;string"\ /\textgreater\newline
		\tab\textless owl:equivalentProperty\newline
		\tab\tab rdf:resource=``http://dbpedia.org/ontology/university"\ /\textgreater\newline
		\tab\textless owl:equivalentProperty\newline
		\tab\tab rdf:resource=``http://schema.org/CollegeOrUniversity"\ /\textgreater\newline
		\tab\textless owl:equivalentProperty\newline
		\tab\tab rdf:resource=``https://www.wikidata.org/wiki/Q3918"\ /\textgreater\newline
		\textless /owl:DatatypeProperty\textgreater
		\\ \hline
	\end{tabular}
	\caption{Propiedad titulados}
	\label{propiedad-titulados}
\end{table}
\chapter{Desarrollo}

Una vez tenemos la ontología completamente diseñada (cuya versión completa se puede encontrar en el \textbf{Anexo I)}, el trabajo restante se divide en dos tareas:

\begin{itemize}
	\item Convertir los datos a un formato triple RDF que siga la ontología que hemos diseñado.
	\item Crear la infraestructura necesaria para disponer de un punto de acceso a la información.
\end{itemize}

\section{Conversión de los datos}

Todos los datos están en archivos en formato CSV siguiendo la misma estructura: la primera fila representa el título del dato y el resto los valores. Para resolver esto la opción más sencilla es desarrollar scripts que nos permitan procesar los archivos originales, aunque se podría hacer en varios lenguajes, se ha elegido Python por ser con el que se está más familiarizado para este tipo de tareas.

\bigskip
El procedimiento consistiría en cargar el archivo con los datos, y después de escribir en el archivo de destino las cabeceras con las definiciones de los espacios de nombres, ir añadiendo cada uno de los datos con el formato necesitado. La forma esquemática sería la siguiente: 

\newpage
\begin{minted}{python}
import csv

id = 0

with open('ORIGEN.csv', 'r') as ifile:
    reader = csv.reader(ifile)
    data = list(reader)

ofile = open('DESTINO.rdf', 'w')
ofile.write("<?xml version=\"1.0\" encoding=\"UTF-8\"?>\n"+
"<rdf:RDF\n"+
"\txmlns=\"http://cabas.ugr.es/resources/\"\n"+
"\txmlns:rdf=\"http://www.w3.org/1999/02/22-rdf-syntax-ns#\"\n"+
"\txmlns:rdfs=\"http://www.w3.org/2000/01/rdf-schema#\"\n"+
"\txmlns:xsd=\"http://www.w3.org/2001/XMLSchema#\"\n"+
"\txmlns:owl=\"http://www.w3.org/2002/07/owl#\"\n"+
"\txmlns:dcterms=\"http://purl.org/dc/terms/\"\n"+
"\txmlns:ugr=\"http://cabas.ugr.es/ontology/ugr#\">\n\n")
ofile.close()

with open('DESTINO.rdf', 'a') as ofile:
    for lines in data:
        if id > 0:
            ofile.write("<rdf:Description rdf:about=\"CLASE#"+
            str(id)+"\">\n"+
            "\t<rdfs:label>"+lines[1]+"</rdfs:label>\n"+
            "\t<ugr:PROPIEDAD_1>"+lines[0]+"</ugr:PROPIEDAD_1>\n"+
            "\t<ugr:PROPIEDAD_2>"+lines[1]+"</ugr:PROPIEDAD_2>\n"+
            "\t<ugr:PROPIEDAD_n>"+lines[n-1]+"</ugr:PROPIEDAD_n>\n"+
            "</rdf:Description>\n\n")
        id += 1

ofile = open('DESTINO.rdf', 'a')
ofile.write("</rdf:RDF>")
ofile.close()
\end{minted}

\section{Infraestructura para el punto de acceso SPARQL}
\chapter{Resultados}

Antes de pasar a las pruebas, debemos comprobar que tanto la ontología como nuestros recursos se han cargado correctamente, ya que en caso contrario no podremos hacer nada. Para hacer esto, vamos ejecutar una consulta que nos devuelta todos los triples {\bf sujeto - predicado - objeto}.

\begin{listing}[!ht]
\begin{minted}[tabsize=2,breaklines]{sparql}
SELECT DISTINCT *
WHERE {
	?s ?p ?o
}
\end{minted}
\caption{Obtiene todos los triples en el sistema}
\end{listing}

\section{Comprobación de la inferencia}

Vamos a empezar a realizar consultas simples e ir subiendo el nivel de complejidad. Por ejemplo, primero obtener el número de hombres y mujeres matriculados en titulaciones de grado en el curso 2015/2016 y después obtener el total de personas matriculadas en esas mismas titulaciones ese mismo curso.
\bigskip

\begin{listing}[!ht]
\begin{minted}[tabsize=2,breaklines]{sparql}
PREFIX ugr: <http://cabas.ugr.es/ontology/ugr#>
PREFIX dcterms: <http://purl.org/dc/terms/>

SELECT ?X ?titulacion ?hombres ?mujeres 
WHERE {
	?X ugr:titulacion ?titulacion .
	?X ugr:hombres ?hombres .
	?X ugr:mujeres ?mujeres .
	?X ugr:curso ?curso .
	?X dcterms:type ?tipo .
	FILTER (?curso = "2015/2016" && ?tipo = "MatriculasGrado")
}
ORDER BY ?titulacion
\end{minted}
\caption{Consulta SPARQL 1}
\end{listing}

\begin{listing}[!ht]
	\begin{minted}[tabsize=2,breaklines]{sparql}
PREFIX ugr: <http://cabas.ugr.es/ontology/ugr#>
PREFIX dcterms: <http://purl.org/dc/terms/>

SELECT ?X ?titulacion (sum(?_personas) as ?personas) 
WHERE {
	?X ugr:titulacion ?titulacion .
	?p rdfs:subPropertyOf* ugr:personas .
	?X ugr:curso ?curso .
	?X dcterms:type ?tipo .
	?X ?p ?_personas .
	FILTER (?curso = "2015/2016" && ?tipo = "MatriculasGrado")
}
GROUP BY ?X ?titulacion
ORDER BY ?titulacion
\end{minted}
\caption{Consulta SPARQL 2}
\end{listing}

\begin{table}[!ht]
	\centering
	\begin{tabular}{|p{.7\textwidth}|p{.1\textwidth}|p{.1\textwidth}|}
		\hline
		\multicolumn{1}{|c|}{\textbf{titulacion}} & \multicolumn{1}{c|}{\textbf{hombres}} & \multicolumn{1}{c|}{\textbf{mujeres}} \\ \hline
		"GRADO CC ACT FISICA Y ED.PRIMA (MELILLA)" & 79 & 25 \\ \hline
		"GRADO"GRADO EN ADMINISTRACION Y DIRECCION DE EMPRESAS (CEUTA)" & 91 & 100 \\ \hline
		"GRADOEN ADMINISTRACION Y DIRECCION DE EMPRESAS" & 740 & 835 \\ \hline
		"GRADO EN ADMINISTRACION Y DIRECCION DE EMPRESAS-DERECHO" & 273 & 454 \\ \hline
	\end{tabular}
	\label{salida_consulta_1}
	\caption{Resumen salida consulta SPARQL 1}
\end{table}

\begin{table}[]
	\centering
	\begin{tabular}{|p{.7\textwidth}|p{.1\textwidth}|}
		\hline
		\multicolumn{1}{|c|}{\textbf{titulacion}} & \multicolumn{1}{c|}{\textbf{personas}} \\ \hline
		"GRADO CC ACT FISICA Y ED.PRIMA (MELILLA)" & 104 \\ \hline
		"GRADO"GRADO EN ADMINISTRACION Y DIRECCION DE EMPRESAS (CEUTA)" & 191 \\ \hline
		"GRADOEN ADMINISTRACION Y DIRECCION DE EMPRESAS" & 1575 \\ \hline
		"GRADO EN ADMINISTRACION Y DIRECCION DE EMPRESAS-DERECHO" & 727 \\ \hline
	\end{tabular}
	\label{salida_consulta_2}
	\caption{Resumen salida consulta SPARQL 2}
\end{table}

Aquí podemos ver un ejemplo simple del motor de inferencia de Virtuoso. Cuando estábamos definiendo la ontología de nuestro sistema, definimos una propiedad llamada {\tt personas}, que aunque no pertenecía al dominio específico de ninguna clase, sí que tenía dos subpropiedades que descendían de ella, {\tt hombres} y {\tt mujeres}. Esto es una situación típica a la hora de procesar cantidades, el obtener totales; pero lo realmente importante es que estamos demostrando que la máquina entiende el significado de los datos, como le hemos dicho que sume todas las cantidades de {\tt personas} y sabe que {\tt hombres} y {\tt mujeres} son {\tt personas}, automáticamente toma la decisión de sumarlos y devolvérnoslo como resultado.
\bigskip

\newpage \
\newpage
En resumen, todo esto quiere decir que nuestra ontología es consistente, ya que acabamos de comprobar la taxonomía al hacer que la máquina razone e infiera la jerarquía de clases habiendo obtenido el resultado esperado. También hemos comprobado que existen instancias de todas las clases y que además dichas instancias cumplen con las especificaciones de las mismas. Por lo que podemos decir que hemos comprobado que el resultado de la ontología es el correcto para lo que se diseño inicialmente.
\chapter{Conclusiones y trabajos futuros}

\section{Conclusiones}

El objetivo de este {\sf Trabajo Fin de Máster} era el de abordar el problema de la obtención de información de los datos contenidos en un portal de datos abiertos mediante peticiones a una interfaz en forma de consulta, de forma que pudiéramos sacar conclusiones sobre la información que ahí estaba contenida.
\bigskip

Para poder extraer el conocimiento que hubiera contenido necesitaríamos que el propio sistema nos ayudara procesando los datos de forma automáticamente de forma que nos fuera más útil entenderlos, pero para esto hay que tener en cuenta un aspecto muy importante; las máquinas no conocen el significado de las cosas, solo sus valores, por lo que tenemos que hacer que los datos sean entendibles por las máquinas.
\bigskip

Como en esta línea de acción se lleva trabajando desde hace bastante tiempo, hemos podido basar todo el desarrollo en estándares que aunque no tengan una gran penetración en el día a día de la {\sf Web}, si que tienen capacidad para hacer cosas grandes. Y esto es así porque el objetivo que se persigue es mucho más ambicioso: hacer que todos los portales de datos de Internet estén interconectados y sus datos vinculados de forma se favorezca la interoperabilidad entre plataformas distintas, pero que seguirán unas mismas estructuras definidas según el campo de actuación.

\newpage
Además, esto también dotaría a las personas de adquirir un mayor conocimiento de una forma más sencilla para ellos al contar con la ayuda de las máquinas. Igualmente, la tendencia es que cada vez más instituciones abran todos sus datos, por los que este tipo de estándares facilitan el acceso al mismo, ya que se puede acceder fácilmente mediante interfaces web que funcionan como capa de abstracción sobre los sistemas gestores de bases de datos que pueden resultarle más difíciles de entender a personas que no tengan un conocimiento informático previo.

\bigskip
El caso particular de este proyecto, se han presentado varias dificultades ya que aunque la ontología diseñada ha permitido que el sistema funcione perfectamente; ha sido la ontología la que se ha tenido que adaptar a una estructura de datos ya existente de forma anterior. Sin lugar a dudas, lo ideal hubiera sido poder partir de un sistema totalmente vacío e ir decidiendo como modelar los datos y qué se busca exactamente compartir.

\section{Trabajos futuros}

Como comentábamos en las conclusiones, una posible extensión de este trabajo sería desarrollar un sistema desde la nada, para adaptarlo a las necesidades que se consideren oportunas después del estudio pertinente.

\bigskip
En cualquier caso, no todas las posibilidades de ampliación llevan por hacer el trabajo desarrollado en este {\sf Trabajo Fin de Máster} quedaran en nada, hacer un sistema más personalizado y cambiar la gestión de los {\sf URI} para cada uno lleve a páginas individuales de los recursos sería también una gran ampliación, además eso haría que el sistema tuviera un gran parecido a un referente en este aspecto como es la DBpedia.

\backmatter

%\begingroup
%	\setlength\parindent{0pt}
%	\input{back/glossary}
%\endgroup

%\begin{appendices}
\addcontentsline{toc}{chapter}{Anexos}
%\addtocontents{toc}{\protect\setcounter{tocdepth}{0}}
\chapter*{Anexo I: Ontología formato RDF/XML}

\label{anexo_i}
\inputminted[tabsize=2,breaklines]{xml}{../../data/semantic/ontology/ugr.rdf}

\chapter*{Anexo II: Ontología formato Terse RDF Triple Language}
\label{anexo_ii}
\inputminted[tabsize=2,breaklines]{text}{../../data/semantic/ontology/ugr.ttl}

\chapter*{Anexo III: Script para la conversión de uno de los conjuntos de datos}
\label{anexo_iii}
\inputminted[tabsize=2,breaklines]{python}{../../data/scripts/OfertaTitulacion.py}

\chapter*{Anexo IV: Ejemplo de conversión de un conjunto de datos}
\label{anexo_iv}

\subsubsection{Datos originales}

\begin{minted}[tabsize=2,breaklines]{text}
RAMA,TITULACION,CENTRO,CAMPUS
CIENCIAS EXPERIMENTALES,GRADO EN BIOLOGI­A,FACULTAD DE CIENCIAS,FUENTENUEVA
CIENCIAS EXPERIMENTALES,GRADO EN BIOQUI­MICA,FACULTAD DE CIENCIAS,FUENTENUEVA
CIENCIAS EXPERIMENTALES,GRADO EN CIENCIA Y TECNOLOGI­A DE LOS ALIMENTOS,FACULTAD DE FARMACIA,CARTUJA
CIENCIAS EXPERIMENTALES,GRADO EN CIENCIAS AMBIENTALES,FACULTAD DE CIENCIAS,FUENTENUEVA
\end{minted}

\subsubsection{Datos convertidos}

\begin{minted}[tabsize=2,breaklines]{xml}
<?xml version="1.0" encoding="UTF-8"?>
<!DOCTYPE rdf:RDF [
	<!ENTITY rdf "http://www.w3.org/1999/02/22-rdf-syntax-ns#" >
	<!ENTITY rdfs "http://www.w3.org/2000/01/rdf-schema#" >
	<!ENTITY xsd "http://www.w3.org/2001/XMLSchema#" >
	<!ENTITY owl "http://www.w3.org/2002/07/owl#" >
	<!ENTITY ugr "http://cabas.ugr.es/ontology/ugr#" >
]>

<rdf:RDF
	xmlns="http://cabas.ugr.es/resources/"
	xmlns:rdf="http://www.w3.org/1999/02/22-rdf-syntax-ns#"
	xmlns:rdfs="http://www.w3.org/2000/01/rdf-schema#"
	xmlns:xsd="http://www.w3.org/2001/XMLSchema#"
	xmlns:owl="http://www.w3.org/2002/07/owl#"
	xmlns:ugr="http://cabas.ugr.es/ontology/ugr#">

<rdf:Description rdf:about="1314#1">
	<rdf:type rdf:resource="#OfertaTitulacionGrado" />
	<ugr:ramaConocimiento>CIENCIAS EXPERIMENTALES</ugr:ramaConocimiento>
	<ugr:titulacion>GRADO EN BIOLOGI­A</ugr:titulacion>
	<ugr:campus>FACULTAD DE CIENCIAS</ugr:campus>
	<ugr:centro>FUENTENUEVA</ugr:centro>
	<ugr:curso>2013/2014</ugr:curso>
</rdf:Description>

<rdf:Description rdf:about="1314#2">
	<rdf:type rdf:resource="#OfertaTitulacionGrado" />
	<ugr:ramaConocimiento>CIENCIAS EXPERIMENTALES</ugr:ramaConocimiento>
	<ugr:titulacion>GRADO EN BIOQUI­MICA</ugr:titulacion>
	<ugr:campus>FACULTAD DE CIENCIAS</ugr:campus>
	<ugr:centro>FUENTENUEVA</ugr:centro>
	<ugr:curso>2013/2014</ugr:curso>
</rdf:Description>

<rdf:Description rdf:about="1314#3">
	<rdf:type rdf:resource="#OfertaTitulacionGrado" />
	<ugr:ramaConocimiento>CIENCIAS EXPERIMENTALES</ugr:ramaConocimiento>
	<ugr:titulacion>GRADO EN CIENCIA Y TECNOLOGI­A DE LOS ALIMENTOS</ugr:titulacion>
	<ugr:campus>FACULTAD DE FARMACIA</ugr:campus>
	<ugr:centro>CARTUJA</ugr:centro>
	<ugr:curso>2013/2014</ugr:curso>
</rdf:Description>

<rdf:Description rdf:about="1314#4">
	<rdf:type rdf:resource="#OfertaTitulacionGrado" />
	<ugr:ramaConocimiento>CIENCIAS EXPERIMENTALES</ugr:ramaConocimiento>
	<ugr:titulacion>GRADO EN CIENCIAS AMBIENTALES</ugr:titulacion>
	<ugr:campus>FACULTAD DE CIENCIAS</ugr:campus>
	<ugr:centro>FUENTENUEVA</ugr:centro>
	<ugr:curso>2013/2014</ugr:curso>
</rdf:Description>

</rdf:RDF>
\end{minted}

\chapter*{Anexo V: Instalación Virtuoso con Ansible}

{\sf Playbook} para desplegar automáticamente una configuración cuyo resultado final sea una instalación de {\sf OpenLink Virtuoso Open-Source Edition} completamente funcional.

\label{anexo_v}
\inputminted[tabsize=2,breaklines]{yaml}{../../ansible/virtuoso.yml}

\chapter*{Anexo VI: Instalación CKAN con Ansible}
%\end{appendices}


\newpage
\begin{thebibliography}{99}
	\addcontentsline{toc}{chapter}{Bibliografía}

\subsubsection*{Referencias bibliográficas consultadas:}

\bibitem{SWB} ``Semantic Web roadmap''. Tim Berners-Lee. 14/10/1998. \url{https://www.w3.org/DesignIssues/Semantic.html}

\bibitem{OLSW} ``Ontology Learning for the Semantic Web. IEEE Intelligent Systems. Volume 16, Issue 2, March/April 2001, Pages 72-79''. Alexander Maedche, Steffen Staab. Marzo/Abril 2001.

\bibitem{ASW} ``Agents and the Semantic Web. IEEE Intelligent Systems. Volume 16, Issue 2, March/April 2001, Pages 30-37''. Marzo/Abril 2001.

\bibitem{SWR} ``Semantic Web: The roles of XML and RDF. IEEE Internet Computing. Volume 4, Issue 5, September 2000, Pages 63-74''. Stefan Decker, Sergey I. Melnik, Frank Van Harmelen, Dieter Fensel, Michel C.A. Klein, Jeen Broekstra, Michael A. Erdmann, Ian Horrocks. Marzo 2003.

\bibitem{OMSA} ``Ontology mapping: The state of the art. Knowledge Engineering Review. Volume 18, Issue 1, March 2003, Pages 1-31''. Yannis Kalfoglou, Marco Schorlemmer. Marzo 2003.

\bibitem{LDDI} ``Linked Data - Design Issues''. Tim Berners-Lee. 18/06/2009. \url{https://www.w3.org/DesignIssues/LinkedData.html}

\bibitem{PSW} ``Programming the Semantic Web''. Toby Segaran, Colin Evans, Jamie Taylor. Julio 2009.

\bibitem{MMCC} ``Metodologías y métodos para la construcción de ontologías. Scientia Et Technica. Nº 50, Abril de 2012. Páginas 133-140''. Abril 2011.

\bibitem{LSQU} ``Learning SPARQL: Querying and Updating with SPARQL 1.1''. Bob DuCharme. Julio 2013.

\bigskip
\subsubsection*{Recursos web consultados}

\bibitem{OWOL} ``OWL Web Ontology Language''. W3C, 10/02/2004. \url{https://www.w3.org/TR/owl-features/}

\bibitem{RDFS} ``SPARQL 1.1 Query Language''. W3C, 21/03/2013. \url{https://www.w3.org/TR/sparql11-query/}

\bibitem{RDFCAS} ``RDF 1.1 Concepts and Abstract Syntax''. W3C, 25/02/2014. \url{https://www.w3.org/TR/2014/REC-rdf11-concepts-20140225/}

\bibitem{RDFS} ``RDF Schema 1.1''. W3C, 25/02/2014. \url{https://www.w3.org/TR/2014/REC-rdf-schema-20140225/}

\bibitem{AD} ``Ansible Documentation''. Red Hat, Inc. , 01/07/2017. \url{http://docs.ansible.com/ansible/index.html}

\bibitem{CD} ``CKAN Documentation''. Open Knowledge Foundation, 2013. \url{http://docs.ckan.org/en/ckan-2.6.2/}

\bibitem{NW} ``NGINX Wiki''. NGINX, Inc., 2017. \url{https://www.nginx.com/resources/wiki/}

\bibitem{OVWSD} ``OpenLink Virtuoso Universal Server Documentation''. OpenLink Software, 2017. \url{http://docs.openlinksw.com/virtuoso/}

\bibitem{OVWSD} ``Vagrant Documentation''. HashiCorp, 2017. \url{https://www.vagrantup.com/docs/}
\end{thebibliography}

\newpage \
\thispagestyle{empty}
\end{document}