\begin{center}
{\LARGE\bfseries\titulo}\\
\end{center}
\begin{center}
\autor\
\end{center}

\textbf{Palabras clave: }{\keywordsEs}

\section*{Resumen}

En este {\sf Trabajo de Fin de Máster} se ha propuesto proporcionar un punto final a un sistema de información basado en {\sf grafos RDF}, de forma que los datos contenidos sean comprensibles y razonables automáticamente por máquinas, utilizando una interfaz para la recuperación de información mediante consultas en el lenguaje {\sf SPARQL}. 

\bigskip
Para hacer esto se ha propuesto transformar los conjuntos de datos que se consideren relevantes de los disponibles en el portal de datos abiertos de la {\sf Universidad de Granada}, {\sf OpenData UGR}, al formato triple RDF siguiendo el modelo de una ontología diseñada para describir y representar la información contenida en los datos almacenados; esto se hará utilizando diferentes estándares definidos por el {\sf World Wide Web Council (W3C)} para la definición de la web semántica como son {\sf RDF}, {\sf RDFS} y {\sf OWL}. Será necesario transformar los datos originales (con diferentes distribuciones, todos en formato {\sf CSV}) en un formato compatible con este sistema, principalmente {\sf RDF/XML}, aunque también se utilizará el formato {\sf Turtle (Terse RDF Triple Language)}. Para ello, se usarán scripts escritos en {\sf Python} para ese propósito.

\bigskip
Como resultado del proyecto tendremos una interfaz mediante la cual usando consultas {\sf SPARQL} podremos obtener información que no consta literalmente en nuestros datos almacenados, pero que gracias a los mecanismos de inferencia el sistema puede procesar automáticamente. Esta información puede sernos útil para sacar conclusiones que a simple vista si analizamos los datos en su estado original es muy posible que no percibiéramos, en definitiva, esta información nos sirve para explotar al máximo el conocimiento que alberga el sistema.

\newpage
\begin{center}
{\LARGE\bfseries\tituloEng}\\
\end{center}
\begin{center}
\autor\
\end{center}

\textbf{Keywords: }{\keywordsEn}

\section*{Abstract}

In this {\sf Master’s End Work} it has been proposed to provide an end point to an {\sf RDF graphs} based information system, so that the data contained are understandable and reasonable automatically by machines, using an interface for information retrieval through queries in the {\sf SPARQL} language.
\bigskip

To do this it has been proposed to transform data sets that are considered relevant to those available in the open data portal of the {\sf University of Granada}, {\sf OpenData UGR}, to the RDF triple format following the model of an ontology designed to describe and represent the information contained in the stored data; this will be done using different standards defined by the {\sf World Wide Web Council (W3C)} for the definition of the semantic web such as {\sf RDF}, {\sf RDFS} and {\sf OWL}. It will be necessary to transform the original data (with different distributions, all in {\sf CSV} format) in a format compatible with this system, mainly {\sf RDF/XML}, although the {\sf Turtle} format ({\sf Terse RDF Triple Language}) will also be used. For that, scripts written in {\sf Python} will be used for this purpose.

\bigskip
As a result of the project we will have an interface whereby using {\sf SPARQL} queries we can get information that is not literally stored in our data, but thanks to the inference mechanisms the system can process automatically. This information can be useful to us to draw conclusions that to the naked eye if we analyze the data in its original state is very possible that we didn't perceive. In short, this information serves to exploit to the maximum the knowledge that houses the system.

\newpage
\thispagestyle{empty}
\
\vspace{3cm}

\noindent\rule[-1ex]{\textwidth}{2pt}\\[4.5ex]

Yo, \textbf{\autor}, alumno de la titulación \textbf{\master} de la \textbf{\escuela\ de la \universidad}, autorizo la ubicación de la siguiente copia de mi Trabajo Fin de Máster (\textit{\titulo}) en la biblioteca del centro para que pueda ser consultada por las personas que lo deseen.

\bigskip
Además, este mismo trabajo es realizado bajo licencia \textbf{Creative Commons Attribution-ShareAlike 4.0} (\url{https://creativecommons.org/licenses/by-sa/4.0/}), dando permiso para copiarlo y redistribuirlo en cualquier medio o formato, también de adaptarlo de la forma que se quiera, pero todo esto siempre y cuando se reconozca la autoría y se distribuya con la misma licencia que el trabajo original. El documento en formato {\sf LaTeX} se puede encontrar en el siguiente repositorio de {\sf GitHub}: \url{https://github.com/germaaan/ProyectoDatosAbiertos}.

\vspace{4cm}

\noindent Fdo: \autor

\vspace{2cm}

\begin{flushright}
\ciudad, a \today
\end{flushright}

\newpage
\thispagestyle{empty}
\
\vspace{3cm}

\noindent\rule[-1ex]{\textwidth}{2pt}\\[4.5ex]

D. \textbf{\tutor}, profesor del \textbf{Departamento de Arquitectura y Tecnología de Computadores} de la \textbf{\universidad}.

\vspace{0.5cm}

\vspace{0.5cm}

\textbf{Informa:}

\vspace{0.5cm}

Que el presente trabajo, titulado \textit{\textbf{\titulo}}, ha sido realizado bajo su supervisión por \textbf{\autor}, y 
autoriza la defensa de dicho trabajo ante el tribunal que corresponda.

\vspace{0.5cm}

Y para que conste, expide y firma el presente informe en \ciudad\ a \today.

\vspace{1cm}

\textbf{El tutor:}

\vspace{5cm}

\noindent \textbf{\tutor}

\chapter*{Agradecimientos}
\thispagestyle{empty}

\vspace{1cm}

A toda las personas de las comunidades en las que participo activamente y que me hacen descubrir todos los días cosas nuevas; son una fuente inagotable de inspiración.