%\begin{appendices}
\addcontentsline{toc}{chapter}{Anexos}
%\addtocontents{toc}{\protect\setcounter{tocdepth}{0}}
\chapter*{Anexo I: Ontología formato RDF/XML}

\chapter*{Anexo II: Ontología formato Terse RDF Triple Language}

\chapter*{Anexo III: Ejemplo recurso formato RDF/XML}

\chapter*{Anexo IV: Ejemplo recurso formato Terse RDF Triple Language}

\chapter*{Anexo V: Instalación Virtuoso con Ansible}
\label{anexo_v}
\begin{minted}{yaml}
# ansible	2.1

---
- hosts: "virtuoso"
  become: "yes"
  remote_user: "ubuntu"
  gather_facts: "no"


  vars:
    user: "vagrant"
    temp_dir: "/tmp"
    version_virtuoso: "7.2.4.2"
    virtuoso_pass: "{{ lookup('env', 'VIRTUOSO_PASSWD') }}"


  tasks:
    - fail:
        msg: "The admin password has no value."
      when: virtuoso_pass == ""

    - name: "Install Python (if not installed)"
      raw: test -e /usr/bin/python || (apt -y update && apt install -y python-minimal)
    - name: "Install Aptitude"
      raw: apt -y update && apt install -y aptitude

    - name: "Update cache"
      apt:
        update_cache: "yes"
        cache_valid_time: "1800"
        upgrade: "safe"
        dpkg_options: "force-confold"
    - name: "Install dependencies"
      apt:
        name: "{{ item }}"
        state: "latest"
        dpkg_options: "force-confold"
      with_items:
        - "autoconf"
        - "automake"
        - "bison"
        - "build-essential"
        - "dpkg-dev"
        - "flex"
        - "gawk"
        - "gperf"
        - "libreadline-dev"
        - "libssl-dev"
        - "libtool"
        - "libxml2-dev"
        - "m4"
        - "make"
        - "odbcinst"
        - "openssl"

    - name: "Download Virtuoso Open-Source"
      get_url:
        url: "https://github.com/openlink/virtuoso-opensource/releases/download/v{{ version_virtuoso }}/virtuoso-opensource-{{ version_virtuoso }}.tar.gz"
        dest: "{{ temp_dir }}"
        mode: "0550"

    - name: "Decompress Virtuoso Open-Source"
      unarchive:
          src: "virtuoso-opensource-{{ version_virtuoso }}.tar.gz"
          dest: "{{ temp_dir }}/virtuoso-opensource-{{ version_virtuoso }}"
      args:
        chdir: "{{ temp_dir }}"

    - name: "Configure Virtuoso Open-Source installation"
      shell: "./configure --prefix=/usr/local/ --with-readline --program-transform-name='s/isql/isql-v/'"
      args:
        chdir: "{{ temp_dir }}/virtuoso-opensource-{{ version_virtuoso }}"
    - name: "Make Virtuoso Open-Source installation"
      shell: "make"
      args:
        chdir: "{{ temp_dir }}/virtuoso-opensource-{{ version_virtuoso }}"
    - name: "Install Virtuoso Open-Source"
      shell: "make install"
      args:
        chdir: "{{ temp_dir }}/virtuoso-opensource-{{ version_virtuoso }}"

    - name: "Set password admin"
      shell: "set password dba {{ virtuoso_pass }}"
      args:
        chdir: "/usr/local/var/lib/virtuoso/db/"
    - name: "Run Virtuoso Open-Source"
      shell: "virtuoso-t +wait +configfile /usr/local/var/lib/virtuoso/db/virtuoso.ini"
      args:
        chdir: "/usr/local/var/lib/virtuoso/db/"
\end{minted}

\chapter*{Anexo VI: Instalación CKAN con Ansible}
%\end{appendices}
