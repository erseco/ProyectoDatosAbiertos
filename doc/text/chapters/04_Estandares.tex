\chapter{Estándares de la Web Semántica}

\section{Ontologías}
Aunque en la introducción hemos explicado brevemente lo que era una ontología, lo que básicamente describíamos como una definición formal de los conceptos del dominio de interés con el que estamos tratando de indicar cuáles son sus tipos, propiedades y relaciones para conocer el estado completo de nuestro dominio de conocimiento. Pues para construir estas ontologías podremos usar dos lenguajes independientes, pero que que se suelen usar conjuntamente: {\sf RDF Schema} y {\sf OWL}.

\section{RDF Schema}
{\sf RDF Schema} (o simplemente {\sf RDFS}) es una extensión del {\sf RDF} original que utiliza su misma estructura y que aparece para solucionar problemas básicos en la definición de ontologías como pueden ser la declaración de clases, las restricciones entre ellas y las restricciones de sus propiedades; ya que en la versión original de {\sf RDF} solo eran posible definir tipos (lo que no es exactamente igual que una clase) y las propiedades de los mismos.

\bigskip
Gracias a que {\sf RDFS} nos provee de los elementos básicos y comunes para la descripción de los datos de nuestro dominio, esto nos puede servir para reutilizar conceptos que son comunes en varios dominios distintos. Para conseguir esto {\sf RDFS} nos permite crear esquemas sencillos usando clases y subclases, además de permitirnos definir propiedades, dominios de aplicación y rangos de valores.

\newpage
\subsection{Clases en RDF Schema}
Las clases son conjuntos de recursos que tienen características comunes y una representación en el mundo real. Tenemos tres clases fundamentales en {\sf RDFS} a partir de las que podemos definir nuestras propias clases, siempre posible crear además una jerarquía de clases.

	\begin{itemize}
		\item {\tt rdfs:Class} es la clase que utilizaremos para definir otras clases.
		\item {\tt rdfs:Property} es la clase a partir de la cual definiremos nuevas propiedades que nos permitan describir nuestros recursos.
		\item {\tt rdfs:Resource} todas las cosas descritas en {\sf RDF} son llamadas {\sf recursos} y son instancias de esta clase, es por eso que esta clase nos permite referenciar cualquier clases desde otra clase.
	\end{itemize}

\subsection{Propiedades en RDF Schema}
Las propiedades nos permitirán describir las características que tienen los distintos recursos, además de las propias relaciones que existen entre distintos recursos.

	\begin{itemize}
		\item {\tt rdf:type} nos permite definir el tipo de un determinado recurso, además, un recursos puede ser una instancia de más de una clase.
		\item {\tt rdfs:subClassOf} es la que nos permitirá definir una jerarquía de clases y subclases, además, una clase puede ser subclase de otras subclases.
		\item {\tt rdfs:subPropertyOf}, al igual que podemos crear subclases, también podemos crear subpropiedades para establecer una jerarquía de las mismas.
	\end{itemize}

\subsection{Restricciones en RDF Schema}	

Las restricciones son las que nos permiten definir las clases sobre la que pueden aplicarse determinadas propiedades y posibles valores:

	\begin{itemize}
		\item {\tt rdfs:domain} definiría las clases sobre las que se puede aplicar una propiedad.
		\item {\tt rdfs:range} definiría los valores que puede tener una propiedad.
	\end{itemize}

\newpage
\section{OWL}

{\sf OWL} ({\sf Ontology Web Language}) es un lenguaje de especificación de ontologías extensión de {\sf RDF Schema}, proporcionando un mayor significado y semántica a las ontologías, ya que nos permite definir restricciones sobre las propiedades (como por ejemplo, los valores que puede tomar una clase) y definir axiomas, sentencias que son siempre ciertas y que nos serán muy útiles a la hora de realizar restricciones (como por ejemplo, que un recurso no puede pertenecer a dos clases que sean disjuntas). Utilizando esta jerarquía de clases y propiedades que forman la ontología, {\sf OWL} se basa en lógicas descriptivas para realizar razonamientos.

\subsection{Sublenguajes de OWL}	
Por otra parte, {\sf OWL} se puede clasificar en tres tipos, según su nivel de expresividad:

\begin{itemize}
	\item \textbf{OWL Lite}. Es la versión más simple y nos provee únicamente de los elementos necesarios para definir una jerarquía de clases y propiedades con restricciones básicas como los tipos de valores, cardinalidades o propiedades de las características (inversa, transitiva, simétrica, funcional e inversa funcinal) entre otras.
	\item \textbf{OWL DL}. Posee mayor expresividad que la anterior, permitiendo además que las jerarquías sean razonadas automáticamente por inferencia que además permitan encontrar posibles inconsistencias mediante el uso de diferentes axiomas (como especificar clases disjuntas)
	\item \textbf{OWL Full}. Es la versión con un nivel mayor de expresividad, pero que está basadas en semánticas diferentes a {\sf OWL Lite} y {\sf OWL DL}. Por ejemplo, en {\sf OWL Full}, una clase puede ser simultáneamente un colección de individuos y un individuo en sí mismo, algo que no está permitido en {\sf OWL DL}.
\end{itemize}

\subsection{Clases e instancias en OWL}	
Después de hablar sobre los tipos de clasificación de {\sf OWL}, tenemos que especificar sus componentes generales, por lo que vamos a comenzar por las clases y las instancias.

\bigskip

Todas las clases {\tt (owl:Class)} son a su vez subclases de una única superclase {\tt (owl:Thing)}. Además las subclases e instancias se definen con {\sf RDFS} mediante las propiedades {\tt rdfs:subClassOf} y {\tt rdfs:type} respectivamente. También una clase puede crearse como la intersección de varias clases {\tt (owl:intersectionOf)}, la unión de varias clases {\tt (owl:unionOf)} o el complemento de otra clase {\tt (owl:complementOf)}.

\subsection{Propiedades en OWL}	

Podemos definir propiedades literales de una clase con {\tt owl:DatatypeProperty}, indicando además el tipo de dato de {\sf XML Schema} del que es dicha propiedad; además, también podemos indicar las propiedades que relacionan distintas clases con {\tt owl:ObjectProperty}. Además, {\sf OWL}  nos permite definir diferentes tipos de propiedades:

	\begin{itemize}
		\item \textbf{Transitiva {\tt (owl:TransitiveProperty)}}: si el sujeto A está relacionado con el sujeto B por medio de la propiedad P y por otro lado el sujeto B está relacionado con el sujeto C también por medio de la propiedad P, si la propiedad P está definida como una propiedad transitiva, se puede inferir que el sujeto A está relacionado con el sujeto C por medio de la propiedad P.
		\item \textbf{Simétrica {\tt (owl:SymmetricProperty)}}: si el sujeto A está relacionado con el sujeto B por medio de la propiedad P, si la propiedad P está definida como una propiedad simétrica, se puede inferir que el sujeto B está relacionado con el sujeto A por medio de la propiedad P.
		\item \textbf{Funcional {\tt (owl:FunctionalProperty)}}: si el sujeto A está relacionado con el sujeto B por medio de la propiedad P y por otro lado el sujeto B está relacionado con el sujeto C también por medio de la propiedad P, si la propiedad P está definida como una propiedad funcional, se puede inferir que el sujeto B y el sujeto C son el mismo sujeto.
		\item \textbf{Inversa {\tt (owl:InverseOf)}}: si la propiedad P1 está definida como la propiedad inversa de la propiedad P2, entonces si el sujeto A está relacionado con el sujeto B por medio de la propiedad P1, se puede inferir que el sujeto B está relacionado con el sujeto A por medio de la propiedad P2.
		\item \textbf{Inversa funcional {\tt (owl:InverseFunctionalProperty)}}: si la propiedad P está definida como propiedad inversa funcional, entonces si el sujeto A está relacionado con el sujeto B por medio de la propiedad P, se puede inferir que el sujeto A y el sujeto B son en realidad el mismo sujeto.
	\end{itemize}

\newpage
\subsection{Restricciones y axiomas en OWL}	
Una vez definidas las clases y las propiedades solo nos queda ver como se definen las restricciones sobre las propiedades. Además del dominio y el rango que ya podíamos definir con {\sf RDFS}, {\sf OWL} nos permite hacer restricciones más descriptivas como indicar de qué forma las instancias de una clase pueden tomar como únicos valores las instancias de otra clase ({\tt owl:allValuesFrom}, {\tt owl:someValuesFrom}, {\tt owl:hasValue}) o la cardinalidad de los mismos ({\tt owl:maxCardinality}, {\tt owl:minCardinality}, {\tt owl:cardinality}).

\bigskip
Los axiomas son sentencias que siempre son ciertas, {\sf OWL DL} se basa en la lógica descriptiva para partiendo de la jerarquía y restricciones de clases poder inferir de forma automática razonamientos sobre los datos que tiene. En concreto, podemos definir desde controlar la integridad de nuestros datos mediante la definición de clases o propiedades disjuntas ({\tt owl:disjointWith}, {\tt owl:AllDisjointClasses}, {\tt owl:AllDisjointProperties}); o indicar que nuestras clases y propiedades son equivalentes a clases y propiedades de otros vocabularios ({\tt owl:equivalentClass}, {\tt owl:equivalentProperty}). Esto último es lo que realmente le da utilidad a tener estructuras con un marco común, ya que es lo que permite la interoperabilidad, permiten que existan los datos enlazados.

\section{SPARQL}

{\sf SPARQL} es el lenguaje recomendado por la {\sf W3C} para acceder a datos almacenados en {\sf RDF}. Las consultas {\sf SPARQL} se hacen sobre un almacén de datos {\sf RDF}, pero además de la propia consulta será necesario especificar el espacio de nombres mediante el {\sf URI} de las ontologías cuyos términos vamos a usar a la hora de realizar las consultas. Además, independientemente del lenguaje en el que estén almacenados los datos ({\sf RDF/XML}, {\sf N3}, {\sf Turtle}, {\sf RDFa}) las consultas se escribirán con sintaxis {\sf Turtle}, lo que además de aportar simplicidad tiene otras ventajas como permite abreviar los {\sf URIs} mediante el uso de prefijos.

\bigskip
Al igual que los triples {\sf RDF}, las consultas {\sf SPARQL} se contruyen a partir de tres elementos: sujeto, predicado y objeto; el sujeto y el objeto serían variables que se relacionan a través del predicado, el valor de cada uno de estos pares de variables es el resultado que obtendremos de una consulta {\sf SPARQL}. Todas las variables a utilizar en la consulta (cuyos nombres siempre llevan un signo de interrogación delante para indicar que son variables) son declaradas con la sentencia {\tt SELECT} al indicar los datos que se van a seleccionar de entre toda la información almacenada, será después con la sentencia {\tt WHERE} donde al indicar cómo se relacionan sujetos y predicados cuando filtremos los datos que vamos a obtener. Un ejemplo de consulta simple: 

\begin{listing}[!ht]
\begin{minted}{sparql}
PREFIX ugr: <http://cabas.ugr.es/ontology/ugr#>

SELECT ?X ?nombre
WHERE {
    ?X ugr:titulacion ?nombre
}
LIMIT 5
\end{minted}
\caption{Consulta {\bf SPARQL} de prueba}
\end{listing}

Esta consulta lo que nos daría como resultado son todos los vínculos de los recursos existentes en nuestros datos que se corresponden al nombre de una titulación. Algo similar a lo que podemos ver en la siguiente tabla.

\begin{table}[!ht]
	\centering
	\begin{tabular}{|p{.45\textwidth}|p{.50\textwidth}|}
		\hline
		\textbf{X} &
		\textbf{nombre}
		\\ \hline
		ENLACE\_A\_RECURSO\#1&
		"TITULACIÓN 1"
		\\ \hline
		ENLACE\_A\_RECURSO\#2&
		"TITULACIÓN 2"
		\\ \hline
		ENLACE\_A\_RECURSO\#3&
		"TITULACIÓN 3"
		\\ \hline
		ENLACE\_A\_RECURSO\#4&
		"TITULACIÓN 4"
		\\ \hline
		ENLACE\_A\_RECURSO\#5&
		"TITULACIÓN 5"
		\\ \hline
	\end{tabular}
	\caption{Resultado consulta de ejemplo}
	\label{consulta-ejemplo}
\end{table}

\subsection{Operadores en SPARQL}

Además de las sentencias {\tt SELECT} y {\tt WHERE}, tenemos varios operadores que podemos utilizar a la hora de crear consultas; no obstante, ni siquiera todas las consultas tienen por qué ser de selección, podemos realizar consultas de otros tipos diferentes:

\begin{itemize}
	\item {\tt CONSTRUCT}: nos devuelve el resultado en forma de triples {\sf RDF} con el formato que le hayamos indicado como plantilla.
	\item {\tt ASK}: nos devuelve como resultado si la consulta que hemos realizado tiene solución o no, pero a diferencia de {\tt SELECT} no nos devuelve la información correspondiente.
	\item {\tt DESCRIBE}: devuelve toda la información contenida sobre un recurso solicitado.
\end{itemize}

Y como ya decíamos, hay una serie de operadores que nos permiten hacer consultas de una mayor complejidad:

\begin{itemize}
	\item {\tt UNION}: nos permite obtener el conjunto de resultados de dos consultas distintas, como por ejemplo, todos los resultados de dos consultas {\tt SELECT} distintas se podrían agrupar en una misma respuesta.
	\item {\tt OPTIONAL}: nos permite obtener como resultado el conjunto de recursos que cumplan alguna de las condiciones indicadas en la clausula {\tt WHERE}.
	\item {\tt FILTER}: nos permite obtener como resultado el conjunto de recursos que cumplan con las condiciones indicadas en la clausula {\tt WHERE} añadiendo además criterios de selección arbitrarios.
\end{itemize}

Y para terminar lo único que nos faltaría es hablar sobre los modificadores que podemos usar en las consultas {\sf SPARQL}:

\begin{itemize}
	\item {\tt ORDER BY}: nos permite indicar bajo qué criterio queremos que se ordenen los resultados obtenidos.
	\item {\tt DISTINCT}: nos permite obtener una solución sin soluciones duplicadas.
	\item {\tt REDUCED}: nos permite obtener un resultado en que las soluciones podrán aparecer duplicadas o no en función de otros aspectos, como que el conjunto fuera demasiado grande para comprobar uno por uno que no hay duplicados.
	\item {\tt LIMIT}: nos permite indicar el número máximo de resultados que queremos obtener.
	\item {\tt OFFSET}: nos permite indicar el número de resultados que queremos despreciar antes de devolver la solución, como podría ser usar {\tt OFFSET 10} para devolver a partir de 11º resultado (considerando que haya más de 10 resultados.)
\end{itemize}

\section{Razonamiento e inferencia}

\begin{quote}``\textit{Se entiende por razonamiento a la facultad que permite resolver problemas, extraer conclusiones y aprender de manera consciente de los hechos, estableciendo conexiones causales y lógicas necesarias entre ellos."}\footnote{https://es.wikipedia.org/wiki/Razonamiento}.\end{quote}

\bigskip
Hemos dicho que el principal objetivo es conseguir que los datos sean entendibles por las máquinas, que puedan razonar sobre ellos y sacar conclusiones; precisamente para eso nos basaremos en la lógica. La lógica estudia las condiciones bajo las cuales siguiendo una serie de pasos elementales se puede pasar de una serie de premisas a una conclusión, cuando estamos tratando con máquinas el problema que encontramos en su forma de procesar la lógica es que no dispone de los mecanismos adecuados para conocer cómo y en qué orden deben realizarse esos pasos elementales, algo que si es capaz de hacer la mente humana de forma natural. 

\bigskip
Esto es lo que intenta solucionar la inteligencia artificial, y en nuestro caso concreto es lo que queremos conseguir, que a través de las ontologías definidas se puedan realizar inferencias sobre la información proporcionada por la ontología descrita, que podamos obtener resultados producto del razonamiento que lleven a que la máquina saque sus propias conclusiones.

\bigskip
Por ejemplo, si tenemos varias clases que entre sus datos cuentan con la propiedad \textbf{hombres} y la propiedad \textbf{mujeres} en referencia al número de hombres o mujeres que se han matriculado en una determinada titulación, proceden de un mismo país o cualquier caso similar, nuestro razonamiento humano nos hace saber que los \textbf{hombres} y las \textbf{mujeres} son \textbf{personas}, por lo que si quisiéramos saber cuántas \textbf{personas} hay matriculadas en una titulación, lo que estaríamos haciendo es sumar el número de \textbf{hombres} y de \textbf{mujeres}; sin embargo, este razonamiento no es obvio para una máquina, necesita una regla que le diga que los \textbf{hombres} y las \textbf{mujeres} son \textbf{personas}.

\bigskip
Aquí es donde vuelve a entrar en juego {\sf RDFS}, ya que como hemos indicado antes, su vocabulario nos permite establecer jerarquías, por lo tanto podríamos especificar una jerarquía en la que tenemos una propiedad que es \textbf{persona} y luego indicar que \textbf{hombres} y \textbf{mujeres} son subpropiedades de \textbf{persona}. Gracias a esta propiedad que acabamos de definir, si la máquina quiere contar número de \textbf{personas}, pero estas no están definidas explícitamente como tales, podrá inferir que lo que debe hacer es contar los \textbf{hombres} y las \textbf{mujeres}.