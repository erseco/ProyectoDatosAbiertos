\chapter{Objetivos}

El principal objetivo de este {\sf Trabajo Fin de Máster} consiste en obtener información sobre los datos contenidos en el portal de datos abiertos de la {\sf Universidad de Granada} mediante peticiones a una interfaz web. Para conseguir el resultado final esperado debemos cumplir los siguientes objetivos:

\begin{enumerate}
	\item Desarrollo de ontología que permita describir y representar la información contenida en los datos almacenados en dicho portal.
	\item Procesamiento de los conjuntos de datos del portal de datos abiertos de la {\sf Universidad de Granada} para convertirlos del formato {\sf CSV} actual al formato {\sf triple RDF}.
	\item Proveer de un punto de acceso a un sistema de recuperación de datos mediante {\sf SPARQL}, que nos permite recuperar información sobre los datos en {\sf RDF} generados.
\end{enumerate}

Una vez finalizado todo el proceso, además de haber obtenido la posibilidad de obtener la información desde un punto de acceso, tendremos varios conjunto de datos en formato {\sf triple RDF} que queremos que sean públicos al igual que los datos originales en formato {\sf CSV}, por lo que también serán liberados con la licencia original: {\sf Open Data Commons Attribution License}\footnote{\url{http://www.opendefinition.org/licenses/odc-by}}.