\chapter{Desarrollo}

Una vez tenemos la ontología completamente diseñada (cuya versión completa se puede encontrar en el \textbf{Anexo I)}, el trabajo restante se divide en dos tareas:

\begin{itemize}
	\item Convertir los datos a un formato triple RDF que siga la ontología que hemos diseñado.
	\item Crear la infraestructura necesaria para disponer de un punto de acceso a la información.
\end{itemize}

\section{Conversión de los datos}

Todos los datos están en archivos en formato CSV siguiendo la misma estructura: la primera fila representa el título del dato y el resto los valores. Para resolver esto la opción más sencilla es desarrollar scripts que nos permitan procesar los archivos originales, aunque se podría hacer en varios lenguajes, se ha elegido Python por ser con el que se está más familiarizado para este tipo de tareas.

\bigskip
El procedimiento consistiría en cargar el archivo con los datos, y después de escribir en el archivo de destino las cabeceras con las definiciones de los espacios de nombres, ir añadiendo cada uno de los datos con el formato necesitado. La forma esquemática sería la siguiente: 

\newpage
\begin{minted}{python}
import csv

id = 0

with open('ORIGEN.csv', 'r') as ifile:
    reader = csv.reader(ifile)
    data = list(reader)

ofile = open('DESTINO.rdf', 'w')
ofile.write("<?xml version=\"1.0\" encoding=\"UTF-8\"?>\n"+
"<rdf:RDF\n"+
"\txmlns=\"http://cabas.ugr.es/resources/\"\n"+
"\txmlns:rdf=\"http://www.w3.org/1999/02/22-rdf-syntax-ns#\"\n"+
"\txmlns:rdfs=\"http://www.w3.org/2000/01/rdf-schema#\"\n"+
"\txmlns:xsd=\"http://www.w3.org/2001/XMLSchema#\"\n"+
"\txmlns:owl=\"http://www.w3.org/2002/07/owl#\"\n"+
"\txmlns:dcterms=\"http://purl.org/dc/terms/\"\n"+
"\txmlns:ugr=\"http://cabas.ugr.es/ontology/ugr#\">\n\n")
ofile.close()

with open('DESTINO.rdf', 'a') as ofile:
    for lines in data:
        if id > 0:
            ofile.write("<rdf:Description rdf:about=\"CLASE#"+
            str(id)+"\">\n"+
            "\t<rdfs:label>"+lines[1]+"</rdfs:label>\n"+
            "\t<ugr:PROPIEDAD_1>"+lines[0]+"</ugr:PROPIEDAD_1>\n"+
            "\t<ugr:PROPIEDAD_2>"+lines[1]+"</ugr:PROPIEDAD_2>\n"+
            "\t<ugr:PROPIEDAD_n>"+lines[n-1]+"</ugr:PROPIEDAD_n>\n"+
            "</rdf:Description>\n\n")
        id += 1

ofile = open('DESTINO.rdf', 'a')
ofile.write("</rdf:RDF>")
ofile.close()
\end{minted}

\section{Infraestructura para el punto de acceso SPARQL}