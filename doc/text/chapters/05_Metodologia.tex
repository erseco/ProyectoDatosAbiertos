\chapter{Metodología}

\section{Dominio y alcance de la ontología}

La ontología propuesta tiene el objetivo de describir la información que está contenida en los diferentes conjuntos de datos del portal de datos abiertos de la {\sf Universidad de Granada}. Por eso el dominio que abarcará será el datos de carácter universitario como son información demanda académica, matriculaciones o tasas académicas entre otros.
\bigskip
Esto lleva a que la ontología sea definida con el objetivo final de que pueda cubrir todos los conceptos que consideremos interesantes para luego obtener información que, aunque quizás no se pueda percibir a simple vista, la obtendremos de forma simplificada a través de las consultas que realizaremos al sistema.

\section{Ontologías existentes}

Aunque trabajamos con información típica que podemos encontrar en cualquier universidad, la organización propia no suele ser la misma de un organismo a otro. En cualquier caso, podemos encontrar ontologías existentes similares a la que vamos a desarrollar que se han utilizado en otras universidades como la {\sf Universitat Pompeu Fabra} o {\sf Universidad Pablo de Olavide}, pero el primer problema que encontramos y que impide que pudiéramos reutilizar conceptos definidos en su ontología es que al tener la mayoría de datos de tipo general (como son las matrículas) de forma individualizada (los datos no están agrupados, son los datos anonimizados que representan a personas físicas), no se pueden adaptar a nuestros orígenes de datos, donde todos los datos están agrupados.

\newpage
En cualquier caso, como otro los factores que queremos conseguir es fomentar la interoperabilidad entre sistemas sí que vamos a utilizar otras ontologías de nivel superior, como son {\sf DBpedia}\footnote{\url{http://dbpedia.org/ontology/}} o  {\sf Wikidata}\footnote{\url{https://www.wikidata.org/wiki/}}, la importancia que tiene esta última es que por ejemplo nos permitiría cruzar nuestros datos con la información contenida en la {\sf Wikipedia}, lo que podría ser muy interesante de cara a posibles futuros trabajos en los que se usara la ontología aquí desarrollada.

\section{Descripción de los conjuntos de datos}

Los conjuntos de datos con los que se ha decidido trabajar son los siguientes:

\begin{itemize}
	\item \textbf{Demanda académica: procedimientos acceso} contiene información relativa al número total de solicitudes de matrícula demandadas en la universidad. Datos desde el curso 2012/2013 hasta el curso 2014/2015.
	\item \textbf{Demanda académica: titulaciones} contiene información sobre la demanda de matrícula en relación con las plazas ofertadas en titulaciones oficiales de grado en la universidad. Datos desde el curso 2014/2015 hasta el curso 2014/2015.
	\item \textbf{Matrículas: grado} contiene información relacionada con las matrículas de titulaciones de grado realizadas en la universidad, agrupándola por rama de conocimiento, titulación y sexo del estudiante. Datos desde el curso 2010/2011 hasta el curso 2014/2015.
	\item \textbf{Matrículas: posgrado} contiene información relacionada con las matrículas de titulaciones de posgrado realizadas en la universidad, agrupándola por rama de conocimiento, titulación y sexo del estudiante. Datos desde el curso 2010/2011 hasta el curso 2014/2015.
	\item \textbf{Número medio de créditos} contiene información relacionada con el número medio de créditos de los estudiantes de la universidad, agrupándola por rama de conocimiento, plan de estudios, número medio de créditos matriculados, número medio de créditos presentados y número medio de créditos superados. Datos desde el curso 2012/2013 hasta el curso 2013/2014.
	\item \textbf{Oferta de titulaciones: doctorado} contiene información relativa a la oferta de titulaciones para estudios de doctorado, agrupándola por titulación, rama de conocimiento, centro y campus. Datos desde el curso 2013/2014 hasta el curso 2015/2016.
	\item \textbf{Oferta de titulaciones: grado} contiene información relativa a la oferta de titulaciones para estudios de grado, agrupándola por titulación, rama de conocimiento, centro y campus. Datos desde el curso 2013/2014 hasta el curso 2015/2016.
	\item \textbf{Oferta de titulaciones: másteres oficiales} contiene información relativa a la oferta de titulaciones para estudios de másteres oficiales, agrupándola por titulación, rama de conocimiento, centro y campus. Datos desde el curso 2013/2014 hasta el curso 2015/2016.
	\item \textbf{Origen geográfico de estudiantes por país} contiene información relativa al origen geográfico de los estudiantes de la universidad, agrupándola por país de origen y sexo del estudiante. Datos desde el curso 2013/2014 hasta el curso 2015/2016.
	\item \textbf{Origen geográfico de estudiantes por provincia} contiene información relativa al origen geográfico de los estudiantes de la universidad, agrupándola por provincia de origen y sexo del estudiante. Datos desde el curso 2013/2014 hasta el curso 2015/2016.
	\item \textbf{Tasas académicas por titulaciones} contiene información relativa a la tasa de rendimiento, tasa de éxito, tasa de abandono inicial, tasa de eficiencia, tasa de graduación y tasa de abandono de los estudiantes según la titulación que estén estudiando en la universidad, agrupándola por titulación, tasa de rendimiento, tasa de éxito, tasa de abandono inicial, tasa de eficiencia, tasa de graduación y tasa de abandono. Datos desde el curso 2011/2012 hasta el curso 2015/2016.
\end{itemize}

El motivo de haber seleccionado solo estos 11 conjuntos de datos de entre los 40 totales, es que dado la gran cantidad de datos que podemos encontrar en el portal, podemos encontrarnos datos de todo tipo; podemos encontrar datos de fácil interpretación como son los datos relacionados con matrículas, pero también podemos encontranos por ejemplo datos de carácter económico que utiliza una serie de códigos que nos permiten hacer un uso tan natural de la información que contienen, por eso en una primera aproximación solo vamos a incluir datos de los que como personas podríamos sacar información válida a simple vista. Además, si no se tienen datos de varios años tampoco se considerará un conjunto de datos interesante ya que no se puede comprobar si esos datos representan un hecho anecdótico o una tendencia.

\newpage
\subsection{Descripción de las clases del sistema}

Aclarado este punto, lo siguiente es concretar que cada uno de estos conjuntos de datos se corresponderá con una clase en nuestra ontología.

\begin{itemize}
	\item \textbf{Demanda académica: procedimientos acceso} $\rightarrow$ {\tt DemandaAcademicaAcceso}
	\item \textbf{Demanda académica: titulaciones} $\rightarrow$ {\tt DemandaAcademicaTitulacion}
	\item \textbf{Matrículas: grado} $\rightarrow$ {\tt MatriculasGrado}
	\item \textbf{Matrículas: posgrado} $\rightarrow$ {\tt MatriculasPosgrado}
	\item \textbf{Número medio de créditos} $\rightarrow$ {\tt NumMedioCreditos}
	\item \textbf{Oferta de titulaciones: doctorado} $\rightarrow$ {\tt OfertaTitulacionDoctorado}
	\item \textbf{Oferta de titulaciones: grado} $\rightarrow$ {\tt OfertaTitulacionGrado}
	\item \textbf{Oferta de titulaciones: másteres oficiales} $\rightarrow$ {\tt OfertaTitulacionMaster}
	\item \textbf{Origen geográfico de estudiantes por país} $\rightarrow$ {\tt OrigenPais}
	\item \textbf{Origen geográfico de estudiantes por provincia} $\rightarrow$ {\tt OrigenProvincia}
	\item \textbf{Tasas académicas por titulaciones} $\rightarrow$ {\tt TasasAcademicasTitulacion}
\end{itemize}

\section{Descripción de las propiedades}

Con las clases definidas, lo siguiente sería definir sus atributos, las relaciones y restricciones para tener la estructura completa de nuestro sistema. Para esto usaremos {\tt owl:DatatypeProperty} que es subclases de la clase {\tt rdf:Property} definida en el estándar {\sf RDF}. Con {\tt owl:DatatypeProperty} podemos vincular valores del tipo que definamos con instancias de las clases que hemos definido.

\bigskip
En lo que se refiere a los atributos, debemos tener en cuenta que hay atributos que se repiten en varias clases, por lo que tendremos que especificarlo en los dominios de dichas propiedades.

\begin{itemize}
	\item \textbf{Campus} representa cada uno de los campus universitarios en los que está dividida la universidad.
	\item \textbf{Centro} representa cada uno de los centros de la universidad, ya sean facultades, escuelas o centros adscritos.
	\item \textbf{Cupo general} representa el número de matrículas realizadas por el grupo de estudiantes que pertenece al cupo general de estudiantes.
	\item \textbf{Curso} representa el curso académico al que pertenecen los datos.
	\item \textbf{Deportistas} representa el número de matrículas realizadas por el grupo de estudiantes que pertenece al cupo de deportistas de alto nivel o alto rendimiento.
	\item \textbf{Discapacitados} representa el número de matrículas realizadas por el grupo de estudiantes que pertenece al cupo de personas con minusvalías reconocidas.
	\item \textbf{Doctorado} representa las titulaciones de doctorado que se ofertan en la universidad.
	\item \textbf{Estado} representa el estado de las solicitudes de matrículas demandadas en la universidad.
	\item \textbf{Grado} representa las titulaciones de grado que se ofertan en la universidad.
	\item \textbf{Hombres} representa el número de estudiantes del sexo masculino que estudian alguna titulación en la universidad.
	\item \textbf{Máster} representa las titulaciones de másteres oficiales que se ofertan en la universidad.
	\item \textbf{Mayores 25} representa el número de matrículas realizadas por el grupo de estudiantes que pertenece al cupo de personas mayores de 25 años.
	\item \textbf{Mayores 40 y 45 } representa el número de matrículas realizadas por el grupo de estudiantes que pertenece al cupo de personas mayores de 40 y de 45 años.
	\item \textbf{Mujeres} representa el número de estudiantes del sexo masculino que estudian alguna titulación en la universidad.
	\item \textbf{Número medio de créditos matriculados}: representa el número medio de créditos matriculados por los estudiantes matriculados en titulaciones de una misma rama de conocimiento.
	\item \textbf{Número medio de créditos presentados}: representa el número medio de créditos a los que estudiantes matriculados en titulaciones de una misma rama de conocimiento se han presentado a las evaluaciones oficiales.
	\item \textbf{Número medio de créditos superados}: representa el número medio de créditos a los que estudiantes matriculados en titulaciones de una misma rama de conocimiento han superado las evaluaciones oficiales.
	\item \textbf{País de origen} representa el número de estudiantes procedentes de un mismo país de origen que estudian alguna titulación en la universidad.
	\item \textbf{Personas} representa el número de estudiantes que estudian alguna titulación en la universidad.
	\item \textbf{Plan de estudios} representa el plan de estudios al que pertenece una titulación.
	\item \textbf{Plazas ofertadas} representa el número de plazas ofertadas para una determinada titulación.
	\item \textbf{Provincia} representa el número de estudiantes procedentes de una misma provincia del territorio español que estudian alguna titulación en la universidad.
	\item \textbf{Rama} representa la rama de conocimiento a la que pertenece alguna titulación.
	\item \textbf{Tasa de abandono} representa el porcentaje entre el número total de estudiantes de nuevo ingreso en una titulación que debieron obtener el título el año académico anterior y que no se han matriculado ni en ese año académico ni en el anterior.
	\item \textbf{Tasa de abandono inicial} representa el porcentaje entre los estudiantes matriculados en una determinada titulación en un curso académico que no se matricularon en dicha titulación en los dos años siguientes y el número total de estudiantes que accedieron a esa misma titulación en ese mismo curso académico.
	\item \textbf{Tasa de eficiencia} representa el porcentaje entre el número total de créditos de la titulación a los que deberían haberse matriculado el conjunto de estudiantes graduados en un año académico y el número total de créditos que se matricularon finalmente.
	\item \textbf{Tasa de graduación} representa el porcentaje de estudiantes que finalizan una titulación en el tiempo previsto por el plan de estudios o en un año académico más y el número de estudiantes que entraron en esa misma titulación.
	\item \textbf{Tasa de rendimiento} representa el porcentaje entre el número total de créditos superados (menos los créditos adaptados, convalidados y reconocidos) por los estudiantes de una titulación y el número total de créditos matriculados.
	\item \textbf{Tipo de procedimiento} representa la forma de acceso por la que los estudiantes han realizado la solicitud de matrícula en la universidad.
	\item \textbf{Titulación} representa las titulaciones de grado que se ofertan en la universidad.
	\item \textbf{Titulados} representa el número de matrículas realizadas por el grupo de estudiantes que pertenece al cupo de titulados universitarios.
	\item \textbf{Universidad} representa el conjunto de todos los campus de la universidad.
\end{itemize}

Con todos los atributos descritos ahora solo nos queda indicar cómo los vamos a referenciar en nuestra ontología y el tipo de dato que serán.

\begin{itemize}
	\item \textbf{Campus} $\rightarrow$ {\tt campus}. Tipo de dato cadena de caracteres.
	\item \textbf{Centro} $\rightarrow$ {\tt centro}. Tipo de dato cadena de caracteres.
	\item \textbf{Cupo general} $\rightarrow$ {\tt cupoGral}. Tipo de dato numérico entero no negativo.
	\item \textbf{Curso} $\rightarrow$ {\tt curso}. Tipo de dato cadena de caracteres.
	\item \textbf{Deportistas} $\rightarrow$ {\tt deportistas}. Tipo de dato numérico entero no negativo.
	\item \textbf{Discapacitados} $\rightarrow$ {\tt discapacitados}. Tipo de dato numérico entero no negativo.
	\item \textbf{Doctorado} $\rightarrow$ {\tt doctorado}. Tipo de dato cadena de caracteres.
	\item \textbf{Estado} $\rightarrow$ {\tt estado}. Tipo de dato cadena de caracteres.
	\item \textbf{Grado} $\rightarrow$ {\tt grado}. Tipo de dato cadena de caracteres.
	\item \textbf{Hombres} $\rightarrow$ {\tt hombres}. Tipo de dato numérico entero no negativo.
	\item \textbf{Máster} $\rightarrow$ {\tt master}. Tipo de dato cadena de caracteres.
	\item \textbf{Mayores 25} $\rightarrow$ {\tt mayor25}. Tipo de dato numérico entero no negativo.
	\item \textbf{Mayores 40 y 45 } $\rightarrow$ {\tt mayor40}. Tipo de dato numérico entero no negativo.
	\item \textbf{Mujeres} $\rightarrow$ {\tt mujeres}. Tipo de dato numérico entero no negativo.
	\item \textbf{Número medio de créditos matriculados} $\rightarrow$ {\tt creditosMatriculados}. Tipo de dato numérico real no negativo.
	\item \textbf{Número medio de créditos presentados} $\rightarrow$ {\tt creditosPresentados}. Tipo de dato numérico real no negativo.
	\item \textbf{Número medio de créditos superados} $\rightarrow$ {\tt creditosSuperados}. Tipo de dato numérico real no negativo.
	\item \textbf{País de origen} $\rightarrow$ {\tt pais}. Tipo de dato cadena de caracteres.
	\item \textbf{Personas} $\rightarrow$ {\tt personas}. Tipo de dato numérico entero no negativo.
	\item \textbf{Plan de estudios} $\rightarrow$ {\tt planEstudios}. Tipo de dato cadena de caracteres.
	\item \textbf{Plazas ofertadas} $\rightarrow$ {\tt plazasOfertadas}. Tipo de dato numérico entero no negativo.
	\item \textbf{Provincia} $\rightarrow$ {\tt provincia}. Tipo de dato cadena de caracteres.
	\item \textbf{Rama de conocimiento} $\rightarrow$ {\tt ramaConocimiento}. Tipo de dato cadena de caracteres.
	\item \textbf{Tasa de abandono} $\rightarrow$ {\tt tasaAbandono}. Tipo de dato numérico real no negativo.
	\item \textbf{Tasa de abandono inicial} $\rightarrow$ {\tt tasaAbandonoInicial}. Tipo de dato numérico real no negativo.
	\item \textbf{Tasa de eficiencia} $\rightarrow$ {\tt tasaEficiencia}. Tipo de dato numérico real no negativo.
	\item \textbf{Tasa de graduación} $\rightarrow$ {\tt tasaGraduacion}. Tipo de dato numérico real no negativo.
	\item \textbf{Tasa de rendimiento} $\rightarrow$ {\tt tasaRendimiento}. Tipo de dato numérico real no negativo.
	\item \textbf{Tipo de procedimiento} $\rightarrow$ {\tt tipoProcedimiento}. Tipo de dato cadena de caracteres.
	\item \textbf{Titulación} $\rightarrow$ {\tt titulacion}. Tipo de dato cadena de caracteres.
	\item \textbf{Titulados} $\rightarrow$ {\tt titulados}. Tipo de dato numérico entero no negativo
	\item \textbf{Universidad} $\rightarrow$ {\tt universidad}. Tipo de dato cadena de caracteres.
\end{itemize}

\newpage
\section{Vocabularios usados}

En el diseño de la ontología todas las clases son del lenguaje definido propio, sí que se han usado otros vocabularios para definir los metadatos de la ontología ({\sf Dublin Core}\footnote{\url{http://dublincore.org/documents/2012/06/14/dcmi-terms/}}, {\sf FOAF}\footnote{\url{http://xmlns.com/foaf/spec/}}, {\sf Creative Commons}\footnote{\url{https://creativecommons.org/ns}} y {\sf VANN}\footnote{\url{http://vocab.org/vann/}}). \bigskip

Para la definición del propio vocabulario (cuyo prefijo de espacio de nombres preferido es \textbf{ugr}), solo se han utilizado los estándares definidos {\sf RDF}, {\sf RDFS}, {\sf OWL} y {\sf XMLS}\footnote{\url{https://www.w3.org/2001/XMLSchema}} (para los tipos de datos); sin embargo, para conseguir el objetivo de conseguir disponer de datos enlazados en las diferentes propiedades se referencian propiedades equivalentes dentro de otros vocabularios de nivel superior como {\sf DBpedia}\footnote{\url{http://dbpedia.org/ontology/}} {\sf Wikidata}\footnote{\url{https://www.wikidata.org/wiki/}} o {\sf Scheme.org}\footnote{\url{http://schema.org/}}.

\section{Definición de las clases del sistema}

Las tablas 3.1 a 3.11 contienen la lista de propiedades de cada una de las clases, así como un ejemplo de una instancia de la clase en formato {\sf RDF/XML}. 

\section{Definición de las propiedades del sistema y sus restricciones}

Las tablas 3.12 a 3.41 contienen el dominio (las clases en las que se encuentran presentes), el rango (el tipo de dato), subpropiedad de (en caso de que lo sea), las propiedades equivalentes en otros vocabularios de nivel superior y su descripción dentro de la ontología en formato {\sf RDF/XML}.
\bigskip

\begin{table}[!ht]
	\centering
	\begin{tabular}{|p{.17\textwidth}|p{.9\textwidth}|}
		\hline
		\multicolumn{2}{|l|}{Clase: \textbf{DemandaAcademicaAcceso}}
		\\ \hline
		Propiedades:&
		\begin{itemize}
			\item tipoProcedimiento
			\item estado
			\item hombres
			\item mujeres
			\item curso
		\end{itemize}
		\\ \hline
		Ejemplo:&
		\textless rdf:Description \newline
		\tab rdf:about=``/DemandaAcademicaAcceso/2012-2013\#1"\textgreater \newline
		\tab \textless rdf:type rdf:resource=``\#DemandaAcademicaAcceso"\ /\textgreater \newline
		\tab \textless ugr:tipoProcedimiento\textgreater \newline\tab\tab CONVOCATORIA ORDINARIA DE JUNIO: PRUEBA DE ACCESO A LA UNIVERSIDAD PARA ESTUDIANTES PROVENIENTES DE BACHILLERATO Y DE CICLOS FORMATIVOS DE GRADO SUPERIOR \newline\tab\textless /ugr:tipoProcedimiento\textgreater \newline
		\tab \textless ugr:estado\textgreater \newline\tab\tab PRESENTADOS FASE GENERAL\newline\tab\textless /ugr:estado\textgreater  \newline
		\tab \textless ugr:hombres rdf:datatype=``\&xsd;nonNegativeInteger"\textgreater \newline\tab\tab2264 \newline \tab \textless /ugr:hombres\textgreater \newline
		\tab \textless ugr:mujeres rdf:datatype=``\&xsd;nonNegativeInteger"\textgreater \newline\tab\tab2877 \newline \tab \textless /ugr:mujeres\textgreater  \newline
		\tab \textless ugr:curso\textgreater \newline\tab\tab2012/2013\newline\tab\textless /ugr:curso\textgreater  \newline
		\textless /rdf:Description\textgreater 
		\\ \hline
	\end{tabular}
	\caption{Clase DemandaAcademicaAcceso}
	\label{clase-demandaacademicaacceso}
\end{table}

\begin{table}[!ht]
	\centering
	\begin{tabular}{|p{.17\textwidth}|p{.9\textwidth}|}
		\hline
		\multicolumn{2}{|l|}{Clase: \textbf{DemandaAcademicaTitulacion}}
		\\ \hline
		Propiedades:&
		\begin{itemize}
			\item titulacion
			\item plazasOfertadas
			\item cupoGral
			\item mayor25
			\item mayor40
			\item titulados
			\item discapacitados
			\item deportistas
		\end{itemize}
		\\ \hline
		Ejemplo:&
		\textless rdf:Description \newline \tab rdf:about=``/DemandaAcademicaTitulacion/2014-2015\#1"\textgreater \newline
		\tab \textless rdf:type rdf:resource=``\#DemandaAcademicaTitulacion"\ /\textgreater \newline
		\tab \textless ugr:titulacion\textgreater \newline\tab\tab ADMINISTRACIÓN Y DIRECCIÓN DE EMPRESAS\newline\tab\textless /ugr:titulacion\textgreater \newline
		\tab \textless ugr:plazasOfertadas rdf:datatype=``\&xsd;nonNegativeInteger"\textgreater  \newline \tab \tab 281\newline\tab\textless /ugr:plazasOfertadas\textgreater \newline
		\tab \textless ugr:cupoGral rdf:datatype=``\&xsd;nonNegativeInteger"\textgreater \newline \tab \tab 271\newline\tab\textless /ugr:cupoGral\textgreater 
		\tab \newline \tab \textless ugr:mayor25 rdf:datatype=``\&xsd;nonNegativeInteger"\textgreater \newline \tab \tab 6\newline\tab\textless /ugr:mayor25\textgreater 
		\tab \newline \tab \textless ugr:mayor40 rdf:datatype=``\&xsd;nonNegativeInteger"\textgreater \newline \tab \tab 0\newline\tab\textless /ugr:mayor40\textgreater 
		\tab \newline \tab \textless ugr:titulados rdf:datatype=``\&xsd;nonNegativeInteger"\textgreater \newline \tab \tab 3\newline\tab\textless /ugr:titulados\textgreater 
		\tab \newline \tab 
		\textless ugr:discapacitados rdf:datatype=``\&xsd;nonNegativeInteger"\textgreater \newline \tab \tab 1\newline\tab\textless /ugr:discapacitados\textgreater 
		\tab \newline \tab \textless ugr:deportistas rdf:datatype=``\&xsd;nonNegativeInteger"\textgreater \newline \tab \tab 1\newline\tab\textless /ugr:deportistas\textgreater 
		\tab \newline \tab \textless ugr:curso\textgreater \newline\tab\tab2014/2015\newline\tab\textless /ugr:curso\textgreater \newline
		\textless /rdf:Description\textgreater 
		\\ \hline
	\end{tabular}
	\caption{Clase DemandaAcademicaTitulacion}
	\label{clase-demandaacademicatitulacion}
\end{table}

\begin{table}[!ht]
	\centering
	\begin{tabular}{|p{.17\textwidth}|p{.9\textwidth}|}
		\hline
		\multicolumn{2}{|l|}{Clase: \textbf{MatriculasGrado}}
		\\ \hline
		Propiedades:&
		\begin{itemize}
			\item ramaConocimiento
			\item titulacion
			\item hombres
			\item mujeres
			\item curso
		\end{itemize}
		\\ \hline
		Ejemplo:&
		\textless rdf:Description rdf:about=``/MatriculasGrado/2010-2011\#1"\textgreater \newline
		\tab \textless rdf:type rdf:resource=``\#MatriculasGrado"\ /\textgreater \newline
		\tab \textless ugr:ramaConocimiento\textgreater \newline \tab\tab ARTES Y HUMANIDADES\newline\tab\textless /ugr:ramaConocimiento\textgreater \newline
		\tab \textless ugr:titulacion\textgreater \newline\tab\tab GRADO EN BELLAS ARTES\newline\tab\textless /ugr:titulacion\textgreater \newline
		\tab \textless ugr:hombres rdf:datatype=``\&xsd;nonNegativeInteger"\textgreater \newline\tab\tab70\newline\tab\textless /ugr:hombres\textgreater 
		\tab \newline\tab\textless ugr:mujeres rdf:datatype=``\&xsd;nonNegativeInteger"\textgreater \newline\tab\tab159\newline\tab\textless /ugr:mujeres\textgreater 
		\tab \newline\tab\textless ugr:curso\textgreater \newline\tab\tab2010/2011\newline\tab\textless /ugr:curso\textgreater \newline
		\textless /rdf:Description\textgreater 
		\\ \hline
	\end{tabular}
	\caption{Clase MatriculasGrado}
	\label{clase-matriculasgrado}
\end{table}

\begin{table}[!ht]
	\centering
	\begin{tabular}{|p{.17\textwidth}|p{.9\textwidth}|}
		\hline
		\multicolumn{2}{|l|}{Clase: \textbf{MatriculasPosgrado}}
		\\ \hline
		Propiedades:&
		\begin{itemize}
			\item titulacion
			\item hombres
			\item mujeres
			\item curso
		\end{itemize}
		\\ \hline
		Ejemplo:&
		\textless rdf:Description \newline\tab  rdf:about=``/MatriculasPosgrado/2010-2011\#1"\textgreater \newline
		\tab \textless rdf:type rdf:resource=``\#MatriculasPosgrado"\ /\textgreater \newline
		\tab \textless ugr:titulacion\textgreater \newline\tab\tab MASTER ERASMUS MUNDUS EN EL COLOR EN LA INFORMATICA Y LA TECNOLOGIA DE LOS MEDIOS / ERASMUS MUNDUS IN COLOR IN INFORMATICS AND MEDIA TECHNOLOGY (CIMET) \newline\tab\textless /ugr:titulacion\textgreater \newline
		\tab \textless ugr:hombres rdf:datatype=``\&xsd;nonNegativeInteger"\textgreater \newline\tab\tab9\newline\tab\textless /ugr:hombres\textgreater 
		\tab \newline\tab\textless ugr:mujeres rdf:datatype=``\&xsd;nonNegativeInteger"\textgreater \newline\tab\tab9\newline\tab\textless /ugr:mujeres\textgreater 
		\tab \newline\tab\textless ugr:curso\textgreater \newline\tab\tab2010/2011\newline\tab\textless /ugr:curso\textgreater \newline
		\textless /rdf:Description\textgreater 
		\\ \hline
	\end{tabular}
	\caption{Clase MatriculasPosgrado}
	\label{clase-matriculasposgrado}
\end{table}

\begin{table}[!ht]
	\centering
	\begin{tabular}{|p{.17\textwidth}|p{.9\textwidth}|}
		\hline
		\multicolumn{2}{|l|}{Clase: \textbf{NumMedioCreditos}}
		\\ \hline
		Propiedades:&
		\begin{itemize}
			\item planEstudios
			\item ramaConocimiento
			\item creditosMatriculados
			\item creditosPresentados
			\item creditosSuperados
			\item curso
		\end{itemize}
		\\ \hline
		Ejemplo:&
		\textless rdf:Description \newline \tab rdf:about=``/NumMedioCreditos/2012-2013\#1"\textgreater \newline
		\tab \textless rdf:type rdf:resource=``\#NumMedioCreditos"\ /\textgreater 
		\tab \newline\tab\textless ugr:planEstudios\textgreater \newline\tab\tab PRIMER/SEGUNDO CICLO\newline\tab\textless /ugr:planEstudios\textgreater 
		\tab \newline\tab\textless ugr:ramaConocimiento\textgreater \newline\tab\tab ARTES Y HUMANIDADES\newline\tab\textless /ugr:ramaConocimiento\textgreater 
		\newline \tab \textless ugr:creditosMatriculados rdf:datatype=``\&xsd;decimal"\textgreater \newline\tab\tab 52.40\newline\tab\textless /ugr:creditosMatriculados\textgreater \newline
		\tab \textless ugr:creditosPresentados rdf:datatype=``\&xsd;decimal"\textgreater \newline\tab\tab 40.06\newline\tab\textless /ugr:creditosPresentados\textgreater \newline
		\tab \textless ugr:creditosSuperados rdf:datatype=``\&xsd;decimal"\textgreater \newline\tab\tab 35.08\newline\tab\textless /ugr:creditosSuperados\textgreater \newline
		\tab \textless ugr:curso\textgreater \newline\tab\tab 2012/2013\newline\tab\textless /ugr:curso\textgreater 
		\newline\textless /rdf:Description\textgreater 
		\\ \hline
	\end{tabular}
	\caption{Clase NumMedioCreditos}
	\label{clase-nummediocreditos}
\end{table}

\begin{table}[!ht]
	\centering
	\begin{tabular}{|p{.17\textwidth}|p{.9\textwidth}|}
		\hline
		\multicolumn{2}{|l|}{Clase: \textbf{OfertaTitulacionDoctorado}}
		\\ \hline
		Propiedades:&
		\begin{itemize}
			\item rama
			\item titulacion
			\item campus
			\item centro
			\item curso
		\end{itemize}
		\\ \hline
		Ejemplo:&
		\textless rdf:Description \newline\tab rdf:about=``/OfertaTitulacionDoctorado/2013-2014\#1"\textgreater \newline
		\tab \textless rdf:type rdf:resource=``\#OfertaTitulacionDoctorado"\ /\textgreater 
		\newline \tab \textless ugr:ramaConocimiento\textgreater \newline\tab\tab ARTES Y HUMANIDADES\newline\tab\textless /ugr:ramaConocimiento\textgreater 
		\newline\tab \textless ugr:titulacion\textgreater \newline\tab\tab PROGRAMA DE DOCTORADO EN BIOMEDICINA\newline\tab \textless/ugr:titulacion\textgreater 
		\newline\tab \textless ugr:campus\textgreater \newline\tab\tab ESCUELA DE DOCTORADO\newline\tab\textless /ugr:campus\textgreater 
		\newline\tab \textless ugr:centro\textgreater \newline\tab\tab UGR\newline\tab\textless /ugr:centro\textgreater 
		\newline\tab \textless ugr:curso\textgreater \newline\tab\tab 2013/2014\newline\tab\textless /ugr:curso\textgreater 
		\newline\textless /rdf:Description\textgreater 
		\\ \hline
	\end{tabular}
	\caption{Clase OfertaTitulacionDoctorado}
	\label{clase-ofertatitulaciondoctorado}
\end{table}

\begin{table}[!ht]
	\centering
	\begin{tabular}{|p{.17\textwidth}|p{.9\textwidth}|}
		\hline
		\multicolumn{2}{|l|}{Clase: \textbf{OfertaTitulacionGrado}}
		\\ \hline
		Propiedades:&
		\begin{itemize}
			\item ramaConocimiento
			\item titulacion
			\item campus
			\item centro
			\item curso
		\end{itemize}
		\\ \hline
		Ejemplo:&
		\textless rdf:Description \newline\tab rdf:about=``/OfertaTitulacionGrado/2013-2014\#1"\textgreater \newline
		\tab \textless rdf:type rdf:resource=``\#OfertaTitulacionGrado"\ /\textgreater 
		\newline \tab \textless ugr:ramaConocimiento\textgreater \newline\tab\tab CIENCIAS\newline\tab\textless /ugr:ramaConocimiento\textgreater 
		\newline\tab \textless ugr:titulacion\textgreater \newline\tab\tab BIOLOGIA\newline\tab \textless/ugr:titulacion\textgreater 
		\newline\tab \textless ugr:campus\textgreater \newline\tab\tab FUENTENUEVA\newline\tab\textless /ugr:campus\textgreater 
		\newline\tab \textless ugr:centro\textgreater \newline\tab\tab FACULTAD DE CIENCIAS\newline\tab\textless /ugr:centro\textgreater 
		\newline\tab \textless ugr:curso\textgreater \newline\tab\tab 2013/2014\newline\tab\textless /ugr:curso\textgreater 
		\newline\textless /rdf:Description\textgreater 
		\\ \hline
	\end{tabular}
	\caption{Clase OfertaTitulacionGrado}
	\label{clase-ofertatitulaciongrado}
\end{table}

\begin{table}[!ht]
	\centering
	\begin{tabular}{|p{.17\textwidth}|p{.9\textwidth}|}
		\hline
		\multicolumn{2}{|l|}{Clase: \textbf{OfertaTitulacionMaster}}
		\\ \hline
		Propiedades:&
		\begin{itemize}
			\item ramaConocimiento
			\item titulacion
			\item campus
			\item centro
			\item curso
		\end{itemize}
		\\ \hline
		Ejemplo:&
		\textless rdf:Description \newline\tab rdf:about=``/OfertaTitulacionMaster/2013-2014\#1"\textgreater \newline
		\tab \textless rdf:type rdf:resource=``\#OfertaTitulacionMaster"\ /\textgreater 
		\newline \tab \textless ugr:ramaConocimiento\textgreater \newline\tab\tab  ARTES Y HUMANIDADES\newline\tab\textless /ugr:ramaConocimiento\textgreater 
		\newline\tab \textless ugr:titulacion\textgreater \newline\tab\tab MASTER UNIVERSITARIO EN ARQUEOLOGIA (M71.56.1)\newline\tab \textless/ugr:titulacion\textgreater 
		\newline\tab \textless ugr:campus\textgreater \newline\tab\tab CARTUJA\newline\tab\textless /ugr:campus\textgreater 
		\newline\tab \textless ugr:centro\textgreater \newline\tab\tab FACULTAD DE FILOSOFÍA Y LETRAS\newline\tab\textless /ugr:centro\textgreater 
		\newline\tab \textless ugr:curso\textgreater \newline\tab\tab 2013/2014\newline\tab\textless /ugr:curso\textgreater 
		\newline\textless /rdf:Description\textgreater 
		\\ \hline
	\end{tabular}
	\caption{Clase OfertaTitulacionMaster}
	\label{clase-ofertatitulacionmaster}
\end{table}

\begin{table}[!ht]
	\centering
	\begin{tabular}{|p{.17\textwidth}|p{.9\textwidth}|}
		\hline
		\multicolumn{2}{|l|}{Clase: \textbf{OrigenPais}}
		\\ \hline
		Propiedades:&
		\begin{itemize}
			\item pais
			\item hombres
			\item mujeres
			\item curso
		\end{itemize}
		\\ \hline
		Ejemplo:&
		\textless rdf:Description \newline\tab rdf:about=``/OrigenPais/2013-2014\#1"\textgreater 
		\tab \newline\tab \textless rdf:type rdf:resource=``\#OrigenPais"\ /\textgreater 
		\newline \tab \textless ugr:pais\textgreater \newline\tab\tab ALBANIA\newline\tab\textless /ugr:pais\textgreater 
		\newline\tab \textless ugr:hombres rdf:datatype=``\&xsd;nonNegativeInteger"\textgreater \newline\tab\tab 3\newline\tab\textless /ugr:hombres\textgreater 
		\newline\tab \textless ugr:mujeres rdf:datatype=``\&xsd;nonNegativeInteger"\textgreater \newline\tab\tab 2\newline\tab\textless /ugr:mujeres\textgreater 
		\newline\tab \textless ugr:curso\textgreater \newline\tab\tab 2013/2014\newline\tab\textless /ugr:curso\textgreater 
		\newline\textless /rdf:Description\textgreater 
		\\ \hline
	\end{tabular}
	\caption{Clase OrigenPais}
	\label{clase-origenpais}
\end{table}

\begin{table}[!ht]
	\centering
	\begin{tabular}{|p{.17\textwidth}|p{.9\textwidth}|}
		\hline
		\multicolumn{2}{|l|}{Clase: \textbf{OrigenProvincia}}
		\\ \hline
		Propiedades:&
		\begin{itemize}
			\item provincia
			\item hombres
			\item mujeres
			\item curso
		\end{itemize}
		\\ \hline
		Ejemplo:&
		\textless rdf:Description \newline\tab rdf:about=``/OrigenProvincia/2013-2014\#1"\textgreater 
		\tab \newline\tab \textless rdf:type rdf:resource=``\#OrigenProvincia"\ /\textgreater 
		\newline \tab \textless ugr:pais\textgreater \newline\tab\tab ALAVA\newline\tab\textless /ugr:pais\textgreater 
		\newline\tab \textless ugr:hombres rdf:datatype=``\&xsd;nonNegativeInteger"\textgreater \newline\tab\tab 24\newline\tab\textless /ugr:hombres\textgreater 
		\newline\tab \textless ugr:mujeres rdf:datatype=``\&xsd;nonNegativeInteger"\textgreater \newline\tab\tab 47\newline\tab\textless /ugr:mujeres\textgreater 
		\newline\tab \textless ugr:curso\textgreater \newline\tab\tab 2013/2014\newline\tab\textless /ugr:curso\textgreater 
		\newline\textless /rdf:Description\textgreater 
		\\ \hline
	\end{tabular}
	\caption{Clase OrigenProvincia}
	\label{clase-origenprovincia}
\end{table}

\begin{table}[!ht]
	\centering
	\begin{tabular}{|p{.17\textwidth}|p{.9\textwidth}|}
		\hline
		\multicolumn{2}{|l|}{Clase: \textbf{TasasAcademicasTitulacion}}
		\\ \hline
		Propiedades:&
		\begin{itemize}
			\item titulacion
			\item tasaRendimiento
			\item tasaExito
			\item tasaAbandonoInicial
			\item tasaEficiencia
			\item tasaGraduacion
			\item tasaAbandono
		\end{itemize}
		\\ \hline
		Ejemplo:&
		\textless rdf:Description\newline\tab rdf:about=``/TasasAcademicasTitulacion/2015-2016\#1"\textgreater 
		\newline\tab \textless rdf:type rdf:resource=``\#TasasAcademicasTitulacion"\ /\textgreater 
		\newline\tab \textless ugr:titulacion\textgreater \newline\tab\tab GRADUADO EN BIOLOGIA\newline\tab\textless /ugr:titulacion\textgreater 
		\newline\tab \textless ugr:tasaRendimiento rdf:datatype=``\&xsd;decimal"\textgreater \newline\tab\tab 74.02\newline\tab\textless /ugr:tasaRendimiento\textgreater 
		\newline\tab \textless ugr:tasaExito rdf:datatype=``\&xsd;decimal"\textgreater \newline\tab\tab 83.12\newline\tab\textless /ugr:tasaExito\textgreater 
		\newline\tab \textless ugr:tasaAbandonoInicial rdf:datatype=``\&xsd;decimal"\textgreater \newline\tab\tab 12.27\newline\tab\textless /ugr:tasaAbandonoInicial\textgreater 
		\newline\tab \textless ugr:tasaEficiencia rdf:datatype=``\&xsd;decimal"\textgreater \newline\tab\tab 98\newline\tab\textless /ugr:tasaEficiencia\textgreater 
		\newline\tab \textless ugr:tasaGraduacion rdf:datatype=``\&xsd;decimal"\textgreater \newline\tab\tab 32.73\newline\tab\textless /ugr:tasaGraduacion\textgreater 
		\newline\tab \textless ugr:tasaAbandono rdf:datatype=``\&xsd;decimal"\textgreater \newline\tab\tab30\newline\tab\textless /ugr:tasaAbandono\textgreater 
		\newline\tab \textless ugr:curso\textgreater \newline\tab\tab 2015/2016\newline\tab\textless /ugr:curso\textgreater 
		\newline\textless /rdf:Description\textgreater 
		\\ \hline
	\end{tabular}
	\caption{Clase TasasAcademicasTitulacion}
	\label{clase-tasasacademicastitulacion}
\end{table}

\begin{table}[!ht]
	\centering
	\begin{tabular}{|p{.25\textwidth}|p{.9\textwidth}|}
		\hline
		\multicolumn{2}{|l|}{Propiedad: \textbf{campus}}
		\\ \hline
		Dominio:&
		\begin{itemize}
			\item OfertaTitulacionDoctorado
			\item OfertaTitulacionGrado
			\item OfertaTitulacionMaster
		\end{itemize}
		\\ \hline
		Rango:&
		http://www.w3.org/2001/XMLSchema\#string
		\\ \hline
		Subpropiedad de:&
		universidad
		\\ \hline
		Propiedades \newline equivalentes:&
		\begin{itemize}
			\item \url{http://dbpedia.org/ontology/campus}
			\item \url{https://www.wikidata.org/wiki/Q209465}
		\end{itemize}
		\\ \hline
		Descripción:&
		\textless owl:DatatypeProperty rdf:about=``\#campus"\textgreater\newline 
		\tab\textless rdfs:label xml:lang=``es"\textgreater\newline
		\tab\tab Campus\newline
		\tab\textless /rdfs:label\textgreater\newline
		\tab\textless rdfs:subPropertyOf\newline
		\tab\tab rdf:resource=``\#universidad"\ /\textgreater\newline
		\tab\textless rdfs:range\newline
		\tab\tab rdf:resource=``\&xsd;string"\ /\textgreater\newline
		\tab\textless rdfs:domain\newline
		\tab\tab rdf:resource=``\#OfertaTitulacionDoctorado"\ /\textgreater\newline
		\tab\textless rdfs:domain\newline
		\tab\tab rdf:resource=``\#OfertaTitulacionGrado"\ /\textgreater\newline
		\tab\textless rdfs:domain\newline
		\tab\tab rdf:resource=``\#OfertaTitulacionMaster"\ /\textgreater\newline
		\tab\textless owl:equivalentProperty\newline
		\tab\tab rdf:resource=``http://dbpedia.org/ontology/campus"\ /\textgreater\newline
		\tab\textless owl:equivalentProperty\newline
		\tab\tab rdf:resource=``https://www.wikidata.org/wiki/Q209465"\ /\textgreater\newline
		\textless /owl:DatatypeProperty\textgreater
		\\ \hline
	\end{tabular}
	\caption{Propiedad campus}
	\label{propiedad-campus}
\end{table}

\begin{table}[!ht]
	\centering
	\begin{tabular}{|p{.25\textwidth}|p{.9\textwidth}|}
		\hline
		\multicolumn{2}{|l|}{Propiedad: \textbf{centro}}
		\\ \hline
		Dominio:&
		\begin{itemize}
			\item OfertaTitulacionDoctorado
			\item OfertaTitulacionGrado
			\item OfertaTitulacionMaster
		\end{itemize}
		\\ \hline
		Rango:&
		http://www.w3.org/2001/XMLSchema\#string
		\\ \hline
		Subpropiedad de:&
		campus
		\\ \hline
		Descripción:&
		\textless owl:DatatypeProperty rdf:about=``\#centro"\textgreater\newline 
		\tab\textless rdfs:label xml:lang=``es"\textgreater\newline
		\tab\tab Centro\newline
		\tab\textless /rdfs:label\textgreater\newline
		\tab\textless rdfs:subPropertyOf\newline
		\tab\tab rdf:resource=``\#campus"\ /\textgreater\newline
		\tab\textless rdfs:range\newline
		\tab\tab rdf:resource=``\&xsd;string"\ /\textgreater\newline
		\tab\textless rdfs:domain\newline
		\tab\tab rdf:resource=``\#OfertaTitulacionDoctorado"\ /\textgreater\newline
		\tab\textless rdfs:domain\newline
		\tab\tab rdf:resource=``\#OfertaTitulacionGrado"\ /\textgreater\newline
		\tab\textless rdfs:domain\newline
		\tab\tab rdf:resource=``\#OfertaTitulacionMaster"\ /\textgreater\newline
		\textless /owl:DatatypeProperty\textgreater
		\\ \hline
	\end{tabular}
	\caption{Propiedad centro}
	\label{propiedad-centro}
\end{table}

\begin{table}[!ht]
	\centering
	\begin{tabular}{|p{.25\textwidth}|p{.9\textwidth}|}
		\hline
		\multicolumn{2}{|l|}{Propiedad: \textbf{cupoGral}}
		\\ \hline
		Dominio:&
		\begin{itemize}
			\item DemandaAcademicaTitulacion
		\end{itemize}
		\\ \hline
		Rango:&
		http://www.w3.org/2001/XMLSchema\#decimal
		\\ \hline
		Descripción:&
		\textless owl:DatatypeProperty rdf:about=``\#cupoGral"\textgreater\newline 
		\tab\textless rdfs:label xml:lang=``es"\textgreater\newline
		\tab\tab Cupo general\newline
		\tab\textless /rdfs:label\textgreater\newline
		\tab\textless rdfs:range\newline
		\tab\tab rdf:resource=``\&xsd;decimal"\ /\textgreater\newline
		\tab\textless rdfs:domain\newline
		\tab\tab rdf:resource=``\#DemandaAcademicaTitulacion"\ /\textgreater\newline
		\textless /owl:DatatypeProperty\textgreater
		\\ \hline
	\end{tabular}
	\caption{Propiedad cupoGral}
	\label{propiedad-cupogral}
\end{table}

\begin{table}[!ht]
	\centering
	\begin{tabular}{|p{.25\textwidth}|p{.9\textwidth}|}
		\hline
		\multicolumn{2}{|l|}{Propiedad: \textbf{curso}}
		\\ \hline
		Dominio:&
		Todas las clases de ``ugr"
		\\ \hline
		Rango:&
		http://www.w3.org/2001/XMLSchema\#string
		\\ \hline
		Propiedades \newline equivalentes:&
		\begin{itemize}
			\item \url{http://purl.org/dc/terms/coverage}
		\end{itemize}
		\\ \hline
		Descripción:&
		\textless owl:DatatypeProperty rdf:about=``\#curso"\textgreater\newline 
		\tab\textless rdfs:label xml:lang=``es"\textgreater\newline
		\tab\tab Curso\newline
		\tab\textless /rdfs:label\textgreater\newline
		\tab\textless rdfs:range\newline
		\tab\tab rdf:resource=``\&xsd;string"\ /\textgreater\newline
		\tab\textless rdfs:domain\newline
		\tab\tab rdf:resource=``\#DemandaAcademicaAcceso"\ /\textgreater\newline
		\tab\textless rdfs:domain\newline
		\tab\tab rdf:resource=``\#DemandaAcademicaTitulacion"\ /\textgreater\newline
		\tab\textless rdfs:domain\newline
		\tab\tab rdf:resource=``\#MatriculasGrado"\ /\textgreater\newline
		\tab\textless rdfs:domain\newline
		\tab\tab rdf:resource=``\#MatriculasPosgrado"\ /\textgreater\newline
		\tab\textless rdfs:domain\newline
		\tab\tab rdf:resource=``\#NumMedioCreditos"\ /\textgreater\newline
		\tab\textless rdfs:domain\newline
		\tab\tab rdf:resource=``\#OfertaTitulacionDoctorado"\ /\textgreater\newline
		\tab\textless rdfs:domain\newline
		\tab\tab rdf:resource=``\#OfertaTitulacionGrado"\ /\textgreater\newline
		\tab\textless rdfs:domain\newline
		\tab\tab rdf:resource=``\#OfertaTitulacionMaster"\ /\textgreater\newline
		\tab\textless rdfs:domain\newline
		\tab\tab rdf:resource=``\#OrigenPais"\ /\textgreater\newline
		\tab\textless rdfs:domain\newline
		\tab\tab rdf:resource=``\#OrigenProvincia"\ /\textgreater\newline
		\tab\textless rdfs:domain\newline
		\tab\tab rdf:resource=``\#TasasAcademicasTitulacion"\ /\textgreater\newline
		\tab\textless owl:equivalentProperty\newline
		\tab\tab rdf:resource=``http://purl.org/dc/terms/coverage"\  /\textgreater\newline
		\textless /owl:DatatypeProperty\textgreater
		\\ \hline
	\end{tabular}
	\caption{Propiedad curso}
	\label{propiedad-curso}
\end{table}

\begin{table}[!ht]
	\centering
	\begin{tabular}{|p{.25\textwidth}|p{.9\textwidth}|}
		\hline
		\multicolumn{2}{|l|}{Propiedad: \textbf{deportistas}}
		\\ \hline
		Dominio:&
		\begin{itemize}
			\item DemandaAcademicaTitulacion
		\end{itemize}
		\\ \hline
		Rango:&
		http://www.w3.org/2001/XMLSchema\#nonNegativeInteger
		\\ \hline
		Propiedades \newline equivalentes:&
		\begin{itemize}
			\item \url{http://dbpedia.org/ontology/Athlete}
			\item \url{http://schema.org/athlete}
			\item \url{https://www.wikidata.org/wiki/Q2066131}
		\end{itemize}
		\\ \hline
		Descripción:&
		\textless owl:DatatypeProperty rdf:about=``\#deportistas"\textgreater\newline 
		\tab\textless rdfs:label xml:lang=``es"\textgreater\newline
		\tab\tab Deportistas\newline
		\tab\textless /rdfs:label\textgreater\newline
		\tab\textless rdfs:range\newline
		\tab\tab rdf:resource=``\&xsd;nonNegativeInteger"\ /\textgreater\newline
		\tab\textless rdfs:domain\newline
		\tab\tab rdf:resource=``\#DemandaAcademicaTitulacion"\ /\textgreater\newline
		\tab\textless owl:equivalentProperty\newline
		\tab\tab rdf:resource=``http://dbpedia.org/ontology/Athlete"\  /\textgreater\newline
		\tab\textless owl:equivalentProperty\newline
		\tab\tab rdf:resource=``http://schema.org/athlete"\  /\textgreater\newline
		\tab\textless owl:equivalentProperty\newline
		\tab\tab rdf:resource=``https://www.wikidata.org/wiki/Q2066131"\  /\textgreater\newline
		\textless /owl:DatatypeProperty\textgreater
		\\ \hline
	\end{tabular}
	\caption{Propiedad deportistas}
	\label{propiedad-deportistas}
\end{table}


\begin{table}[!ht]
	\centering
	\begin{tabular}{|p{.25\textwidth}|p{.9\textwidth}|}
		\hline
		\multicolumn{2}{|l|}{Propiedad: \textbf{discapacitados}}
		\\ \hline
		Dominio:&
		\begin{itemize}
			\item DemandaAcademicaTitulacion
		\end{itemize}
		\\ \hline
		Rango:&
		http://www.w3.org/2001/XMLSchema\#nonNegativeInteger
		\\ \hline
		Propiedades \newline equivalentes:&
		\begin{itemize}
			\item \url{https://www.wikidata.org/wiki/Q15978181}
		\end{itemize}
		\\ \hline
		Descripción:&
		\textless owl:DatatypeProperty rdf:about=``\#discapacitados"\textgreater\newline 
		\tab\textless rdfs:label xml:lang=``es"\textgreater\newline
		\tab\tab Discapacitados\newline
		\tab\textless /rdfs:label\textgreater\newline
		\tab\textless rdfs:range\newline
		\tab\tab rdf:resource=``\&xsd;nonNegativeInteger"\ /\textgreater\newline
		\tab\textless rdfs:domain\newline
		\tab\tab rdf:resource=``\#DemandaAcademicaTitulacion"\ /\textgreater\newline
		\tab\textless owl:equivalentProperty\newline
		\tab\tab rdf:resource=``https://www.wikidata.org/wiki/Q15978181"\  /\textgreater\newline
		\textless /owl:DatatypeProperty\textgreater
		\\ \hline
	\end{tabular}
	\caption{Propiedad discapacitados}
	\label{propiedad-discapacitados}
\end{table}

\begin{table}[!ht]
	\centering
	\begin{tabular}{|p{.25\textwidth}|p{.9\textwidth}|}
		\hline
		\multicolumn{2}{|l|}{Propiedad: \textbf{doctorado}}
		\\ \hline
		Dominio:&
		\begin{itemize}
			\item OfertaTitulacionDoctorado
		\end{itemize}
		\\ \hline
		Rango:&
		http://www.w3.org/2001/XMLSchema\#string
		\\ \hline
		Subpropiedad de:&
		titulacion
		\\ \hline
		Propiedades \newline equivalentes:&
		\begin{itemize}
			\item \url{http://dbpedia.org/page/Doctorate}
			\item \url{https://www.wikidata.org/wiki/Q849697}
		\end{itemize}
		\\ \hline
		Descripción:&
		\textless owl:DatatypeProperty rdf:about=``\#doctorado"\textgreater\newline 
		\tab\textless rdfs:label xml:lang=``es"\textgreater\newline
		\tab\tab Doctorado\newline
		\tab\textless /rdfs:label\textgreater\newline
		\tab\textless rdfs:subPropertyOf\newline
		\tab\tab rdf:resource=``\#titulacion"\ /\textgreater\newline
		\tab\textless rdfs:range\newline
		\tab\tab rdf:resource=``\&xsd;string"\ /\textgreater\newline
		\tab\textless rdfs:domain\newline
		\tab\tab rdf:resource=``\#OfertaTitulacionDoctorado"\ /\textgreater\newline
		\tab\textless rdfs:domain\newline
		\tab\textless owl:equivalentProperty\newline
		\tab\tab rdf:resource=``http://dbpedia.org/page/Doctorate"\  /\textgreater\newline
		\tab\textless owl:equivalentProperty\newline
		\tab\tab rdf:resource=``https://www.wikidata.org/wiki/Q849697"\  /\textgreater\newline
		\textless /owl:DatatypeProperty\textgreater
		\\ \hline
	\end{tabular}
	\caption{Propiedad doctorado}
	\label{propiedad-doctorado}
\end{table}

\begin{table}[!ht]
	\centering
	\begin{tabular}{|p{.25\textwidth}|p{.9\textwidth}|}
		\hline
		\multicolumn{2}{|l|}{Propiedad: \textbf{estado}}
		\\ \hline
		Dominio:&
		\begin{itemize}
			\item DemandaAcademicaAcceso
		\end{itemize}
		\\ \hline
		Rango:&
		http://www.w3.org/2001/XMLSchema\#string
		\\ \hline
		Descripción:&
		\textless owl:DatatypeProperty rdf:about=``\#estado"\textgreater\newline 
		\tab\textless rdfs:label xml:lang=``es"\textgreater\newline
		\tab\tab Estado\newline
		\tab\textless /rdfs:label\textgreater\newline
		\tab\textless rdfs:range\newline
		\tab\tab rdf:resource=``\&xsd;string"\ /\textgreater\newline
		\tab\textless rdfs:domain\newline
		\tab\tab rdf:resource=``\#DemandaAcademicaAcceso"\ /\textgreater\newline
		\textless /owl:DatatypeProperty\textgreater
		\\ \hline
	\end{tabular}
	\caption{Propiedad estado}
	\label{propiedad-estado}
\end{table}

\begin{table}[!ht]
	\centering
	\begin{tabular}{|p{.25\textwidth}|p{.9\textwidth}|}
		\hline
		\multicolumn{2}{|l|}{Propiedad: \textbf{grado}}
		\\ \hline
		Dominio:&
		\begin{itemize}
			\item DemandaAcademicaTitulacion
			\item MatriculasGrado
			\item OfertaTitulacionGrado
			\item TasasAcademicasTitulacion
		\end{itemize}
		\\ \hline
		Rango:&
		http://www.w3.org/2001/XMLSchema\#string
		\\ \hline
		Subpropiedad de:&
		titulacion
		\\ \hline
		Propiedades \newline equivalentes:&
		\begin{itemize}
			\item \url{http://dbpedia.org/page/Bachelor\%27s_degree}
			\item \url{https://www.wikidata.org/wiki/Q6008527}
		\end{itemize}
		\\ \hline
		Descripción:&
		\textless owl:DatatypeProperty rdf:about=``\#grado"\textgreater\newline 
		\tab\textless rdfs:label xml:lang=``es"\textgreater\newline
		\tab\tab Grado\newline
		\tab\textless /rdfs:label\textgreater\newline
		\tab\textless rdfs:subPropertyOf\newline
		\tab\tab rdf:resource=``\#titulacion"\ /\textgreater\newline
		\tab\textless rdfs:range\newline
		\tab\tab rdf:resource=``\&xsd;string"\ /\textgreater\newline
		\tab\textless rdfs:domain\newline
		\tab\tab rdf:resource=``\#DemandaAcademicaTitulacion"\ /\textgreater\newline
		\tab\textless rdfs:domain\newline
		\tab\tab rdf:resource=``\#MatriculasGrado"\ /\textgreater\newline
		\tab\textless rdfs:domain\newline
		\tab\tab rdf:resource=``\#OfertaTitulacionGrado"\ /\textgreater\newline
		\tab\textless rdfs:domain\newline
		\tab\tab rdf:resource=``\#TasasAcademicasTitulacion"\ /\textgreater\newline
		\tab\textless rdfs:domain\newline
		\tab\textless owl:equivalentProperty\newline
		\tab\tab rdf:resource=``http://dbpedia.org/page/Bachelor\%27s\_degree"\  /\textgreater\newline
		\tab\textless owl:equivalentProperty\newline
		\tab\tab rdf:resource=``https://www.wikidata.org/wiki/Q6008527"\  /\textgreater\newline
		\textless /owl:DatatypeProperty\textgreater
		\\ \hline
	\end{tabular}
	\caption{Propiedad grado}
	\label{propiedad-grado}
\end{table}

\begin{table}[!ht]
	\centering
	\begin{tabular}{|p{.25\textwidth}|p{.9\textwidth}|}
		\hline
		\multicolumn{2}{|l|}{Propiedad: \textbf{hombres}}
		\\ \hline
		Dominio:&
		\begin{itemize}
			\item DemandaAcademicaAcceso
			\item MatriculasGrado
			\item MatriculasPosgrado
			\item OrigenPais
			\item OrigenProvincia
		\end{itemize}
		\\ \hline
		Rango:&
		http://www.w3.org/2001/XMLSchema\#nonNegativeInteger
		\\ \hline
		Subpropiedad de:&
		personas
		\\ \hline
		Propiedades \newline equivalentes:&
		\begin{itemize}
			\item \url{http://dbpedia.org/page/Man}
			\item \url{https://schema.org/Male}
			\item \url{https://www.wikidata.org/wiki/Q8441}
		\end{itemize}
		\\ \hline
		Descripción:&
		\textless owl:DatatypeProperty rdf:about=``\#hombres"\textgreater\newline 
		\tab\textless rdfs:label xml:lang=``es"\textgreater\newline
		\tab\tab Hombres\newline
		\tab\textless /rdfs:label\textgreater\newline
		\tab\textless rdfs:subPropertyOf\newline
		\tab\tab rdf:resource=``\#personas"\ /\textgreater\newline
		\tab\textless rdfs:range\newline
		\tab\tab rdf:resource=``\&xsd;nonNegativeInteger"\ /\textgreater\newline
		\tab\textless rdfs:domain\newline
		\tab\tab rdf:resource=``\#DemandaAcademicaAcceso"\ /\textgreater\newline
		\tab\textless rdfs:domain\newline
		\tab\tab rdf:resource=``\#MatriculasGrado"\ /\textgreater\newline
		\tab\textless rdfs:domain\newline
		\tab\tab rdf:resource=``\#MatriculasPosgrado"\ /\textgreater\newline
		\tab\textless rdfs:domain\newline
		\tab\tab rdf:resource=``\#OrigenPais"\ /\textgreater\newline
		\tab\textless rdfs:domain\newline
		\tab\tab rdf:resource=``\#OrigenProvincia"\ /\textgreater\newline
		\tab\textless owl:equivalentProperty\newline
		\tab\tab rdf:resource=``http://dbpedia.org/page/Man"\  /\textgreater\newline
		\tab\textless owl:equivalentProperty\newline
		\tab\tab rdf:resource=``https://schema.org/Male"\  /\textgreater\newline
		\tab\textless owl:equivalentProperty\newline
		\tab\tab rdf:resource=``https://www.wikidata.org/wiki/Q8441"\  /\textgreater\newline
		\textless /owl:DatatypeProperty\textgreater
		\\ \hline
	\end{tabular}
	\caption{Propiedad hombres}
	\label{propiedad-hombres}
\end{table}

\begin{table}[!ht]
	\centering
	\begin{tabular}{|p{.25\textwidth}|p{.9\textwidth}|}
		\hline
		\multicolumn{2}{|l|}{Propiedad: \textbf{master}}
		\\ \hline
		Dominio:&
		\begin{itemize}
			\item MatriculasPosgrado
			\item OfertaTitulacionMaster
		\end{itemize}
		\\ \hline
		Rango:&
		http://www.w3.org/2001/XMLSchema\#string
		\\ \hline
		Subpropiedad de:&
		titulacion
		\\ \hline
		Propiedades \newline equivalentes:&
		\begin{itemize}
			\item \url{http://dbpedia-live.openlinksw.com/page/Master's_degree}
			\item \url{https://www.wikidata.org/wiki/Q183816}
		\end{itemize}
		\\ \hline
		Descripción:&
		\textless owl:DatatypeProperty rdf:about=``\#master"\textgreater\newline 
		\tab\textless rdfs:label xml:lang=``es"\textgreater\newline
		\tab\tab Máster\newline
		\tab\textless /rdfs:label\textgreater\newline
		\tab\textless rdfs:subPropertyOf\newline
		\tab\tab rdf:resource=``\#titulacion"\ /\textgreater\newline
		\tab\textless rdfs:range\newline
		\tab\tab rdf:resource=``\&xsd;string"\ /\textgreater\newline
		\tab\textless rdfs:domain\newline
		\tab\tab rdf:resource=``\#MatriculasPosgrado"\ /\textgreater\newline
		\tab\textless rdfs:domain\newline
		\tab\tab rdf:resource=``\#OfertaTitulacionMaster"\ /\textgreater\newline
		\tab\textless rdfs:domain\newline
		\tab\textless owl:equivalentProperty\newline
		\tab\tab rdf:resource=\newline\tab\tab``http://dbpedia-live.openlinksw.com/page/Master's\_degree"\  /\textgreater\newline
		\tab\textless owl:equivalentProperty\newline
		\tab\tab rdf:resource=``https://www.wikidata.org/wiki/Q183816"\  /\textgreater\newline
		\textless /owl:DatatypeProperty\textgreater
		\\ \hline
	\end{tabular}
	\caption{Propiedad master}
	\label{propiedad-master}
\end{table}

\begin{table}[!ht]
	\centering
	\begin{tabular}{|p{.25\textwidth}|p{.9\textwidth}|}
		\hline
		\multicolumn{2}{|l|}{Propiedad: \textbf{mayor25}}
		\\ \hline
		Dominio:&
		\begin{itemize}
			\item DemandaAcademicaTitulacion
		\end{itemize}
		\\ \hline
		Rango:&
		http://www.w3.org/2001/XMLSchema\#nonNegativeInteger
		\\ \hline
		Descripción:&
		\textless owl:DatatypeProperty rdf:about=``\#mayor25"\textgreater\newline 
		\tab\textless rdfs:label xml:lang=``es"\textgreater\newline
		\tab\tab Mayores de 25\newline
		\tab\textless /rdfs:label\textgreater\newline
		\tab\textless rdfs:range\newline
		\tab\tab rdf:resource=``\&xsd;nonNegativeInteger"\ /\textgreater\newline
		\tab\textless rdfs:domain\newline
		\tab\tab rdf:resource=``\#DemandaAcademicaTitulacion"\ /\textgreater\newline
		\textless /owl:DatatypeProperty\textgreater
		\\ \hline
	\end{tabular}
	\caption{Propiedad mayor25}
	\label{propiedad-mayor25}
\end{table}

\begin{table}[!ht]
	\centering
	\begin{tabular}{|p{.25\textwidth}|p{.9\textwidth}|}
		\hline
		\multicolumn{2}{|l|}{Propiedad: \textbf{mayor40}}
		\\ \hline
		Dominio:&
		\begin{itemize}
			\item DemandaAcademicaTitulacion
		\end{itemize}
		\\ \hline
		Rango:&
		http://www.w3.org/2001/XMLSchema\#nonNegativeInteger
		\\ \hline
		Descripción:&
		\textless owl:DatatypeProperty rdf:about=``\#mayor40"\textgreater\newline 
		\tab\textless rdfs:label xml:lang=``es"\textgreater\newline
		\tab\tab Mayores de 40 y 45\newline
		\tab\textless /rdfs:label\textgreater\newline
		\tab\textless rdfs:range\newline
		\tab\tab rdf:resource=``\&xsd;nonNegativeInteger"\ /\textgreater\newline
		\tab\textless rdfs:domain\newline
		\tab\tab rdf:resource=``\#DemandaAcademicaTitulacion"\ /\textgreater\newline
		\textless /owl:DatatypeProperty\textgreater
		\\ \hline
	\end{tabular}
	\caption{Propiedad mayor40}
	\label{propiedad-mayor40}
\end{table}

\begin{table}[!ht]
	\centering
	\begin{tabular}{|p{.25\textwidth}|p{.9\textwidth}|}
		\hline
		\multicolumn{2}{|l|}{Propiedad: \textbf{mujeres}}
		\\ \hline
		Dominio:&
		\begin{itemize}
			\item DemandaAcademicaAcceso
			\item MatriculasGrado
			\item MatriculasPosgrado
			\item OrigenPais
			\item OrigenProvincia
		\end{itemize}
		\\ \hline
		Rango:&
		http://www.w3.org/2001/XMLSchema\#nonNegativeInteger
		\\ \hline
		Subpropiedad de:&
		personas
		\\ \hline
		Propiedades \newline equivalentes:&
		\begin{itemize}
			\item \url{http://dbpedia.org/page/Woman}
			\item \url{https://schema.org/Female}
			\item \url{https://www.wikidata.org/wiki/Q467}
		\end{itemize}
		\\ \hline
		Descripción:&
		\textless owl:DatatypeProperty rdf:about=``\#mujeres"\textgreater\newline 
		\tab\textless rdfs:label xml:lang=``es"\textgreater\newline
		\tab\tab Mujeres\newline
		\tab\textless /rdfs:label\textgreater\newline
		\tab\textless rdfs:subPropertyOf\newline
		\tab\tab rdf:resource=``\#personas"\ /\textgreater\newline
		\tab\textless rdfs:range\newline
		\tab\tab rdf:resource=``\&xsd;nonNegativeInteger"\ /\textgreater\newline
		\tab\textless rdfs:domain\newline
		\tab\tab rdf:resource=``\#DemandaAcademicaAcceso"\ /\textgreater\newline
		\tab\textless rdfs:domain\newline
		\tab\tab rdf:resource=``\#MatriculasGrado"\ /\textgreater\newline
		\tab\textless rdfs:domain\newline
		\tab\tab rdf:resource=``\#MatriculasPosgrado"\ /\textgreater\newline
		\tab\textless rdfs:domain\newline
		\tab\tab rdf:resource=``\#OrigenPais"\ /\textgreater\newline
		\tab\textless rdfs:domain\newline
		\tab\tab rdf:resource=``\#OrigenProvincia"\ /\textgreater\newline
		\tab\textless owl:equivalentProperty\newline
		\tab\tab rdf:resource=``http://dbpedia.org/page/Woman"\  /\textgreater\newline
		\tab\textless owl:equivalentProperty\newline
		\tab\tab rdf:resource=``https://schema.org/Female"\  /\textgreater\newline
		\tab\textless owl:equivalentProperty\newline
		\tab\tab rdf:resource=``https://www.wikidata.org/wiki/Q467"\  /\textgreater\newline
		\textless /owl:DatatypeProperty\textgreater
		\\ \hline
	\end{tabular}
	\caption{Propiedad mujeres}
	\label{propiedad-mujeres}
\end{table}

\begin{table}[!ht]
	\centering
	\begin{tabular}{|p{.25\textwidth}|p{.9\textwidth}|}
		\hline
		\multicolumn{2}{|l|}{Propiedad: \textbf{creditosMatriculados}}
		\\ \hline
		Dominio:&
		\begin{itemize}
			\item NumMedioCreditos
		\end{itemize}
		\\ \hline
		Rango:&
		http://www.w3.org/2001/XMLSchema\#decimal
		\\ \hline
		Descripción:&
		\textless owl:DatatypeProperty rdf:about=``\#creditosMatriculados"\textgreater\newline 
		\tab\textless rdfs:label xml:lang=``es"\textgreater\newline
		\tab\tab Número medio de créditos matriculados\newline
		\tab\textless /rdfs:label\textgreater\newline
		\tab\textless rdfs:range\newline
		\tab\tab rdf:resource=``\&xsd;decimal"\ /\textgreater\newline
		\tab\textless rdfs:domain\newline
		\tab\tab rdf:resource=``\#NumMedioCreditos"\ /\textgreater\newline
		\textless /owl:DatatypeProperty\textgreater
		\\ \hline
	\end{tabular}
	\caption{Propiedad creditosMatriculados}
	\label{propiedad-creditosmatriculados}
\end{table}

\begin{table}[!ht]
	\centering
	\begin{tabular}{|p{.25\textwidth}|p{.9\textwidth}|}
		\hline
		\multicolumn{2}{|l|}{Propiedad: \textbf{creditosPresentados}}
		\\ \hline
		Dominio:&
		\begin{itemize}
			\item NumMedioCreditos
		\end{itemize}
		\\ \hline
		Rango:&
		http://www.w3.org/2001/XMLSchema\#decimal
		\\ \hline
		Descripción:&
		\textless owl:DatatypeProperty rdf:about=``\#creditosPresentados"\textgreater\newline 
		\tab\textless rdfs:label xml:lang=``es"\textgreater\newline
		\tab\tab Número medio de créditos presentados\newline
		\tab\textless /rdfs:label\textgreater\newline
		\tab\textless rdfs:range\newline
		\tab\tab rdf:resource=``\&xsd;decimal"\ /\textgreater\newline
		\tab\textless rdfs:domain\newline
		\tab\tab rdf:resource=``\#NumMedioCreditos"\ /\textgreater\newline
		\textless /owl:DatatypeProperty\textgreater
		\\ \hline
	\end{tabular}
	\caption{Propiedad creditosPresentados}
	\label{propiedad-creditospresentados}
\end{table}

\begin{table}[!ht]
	\centering
	\begin{tabular}{|p{.25\textwidth}|p{.9\textwidth}|}
		\hline
		\multicolumn{2}{|l|}{Propiedad: \textbf{creditosSuperados}}
		\\ \hline
		Dominio:&
		\begin{itemize}
			\item NumMedioCreditos
		\end{itemize}
		\\ \hline
		Rango:&
		http://www.w3.org/2001/XMLSchema\#decimal
		\\ \hline
		Descripción:&
		\textless owl:DatatypeProperty rdf:about=``\#creditosSuperados"\textgreater\newline 
		\tab\textless rdfs:label xml:lang=``es"\textgreater\newline
		\tab\tab Número medio de créditos superados\newline
		\tab\textless /rdfs:label\textgreater\newline
		\tab\textless rdfs:range\newline
		\tab\tab rdf:resource=``\&xsd;decimal"\ /\textgreater\newline
		\tab\textless rdfs:domain\newline
		\tab\tab rdf:resource=``\#NumMedioCreditos"\ /\textgreater\newline
		\textless /owl:DatatypeProperty\textgreater
		\\ \hline
	\end{tabular}
	\caption{Propiedad creditosSuperados}
	\label{propiedad-creditossuperados}
\end{table}

\begin{table}[!ht]
	\centering
	\begin{tabular}{|p{.25\textwidth}|p{.9\textwidth}|}
		\hline
		\multicolumn{2}{|l|}{Propiedad: \textbf{pais}}
		\\ \hline
		Dominio:&
		\begin{itemize}
			\item OrigenPais
		\end{itemize}
		\\ \hline
		Rango:&
		http://www.w3.org/2001/XMLSchema\#string
		\\ \hline
		Propiedades \newline equivalentes:&
		\begin{itemize}
			\item \url{http://dbpedia.org/ontology/country}
			\item \url{http://schema.org/Country}
			\item \url{https://www.wikidata.org/wiki/Q6256}
		\end{itemize}
		\\ \hline
		Descripción:&
		\textless owl:DatatypeProperty rdf:about=``\#pais"\textgreater\newline 
		\tab\textless rdfs:label xml:lang=``es"\textgreater\newline
		\tab\tab Pais de origen\newline
		\tab\textless /rdfs:label\textgreater\newline
		\tab\textless rdfs:range\newline
		\tab\tab rdf:resource=``\&xsd;string"\ /\textgreater\newline
		\tab\textless rdfs:domain\newline
		\tab\tab rdf:resource=``\#OrigenPais"\ /\textgreater\newline
		\tab\textless owl:equivalentProperty\newline
		\tab\tab rdf:resource=``http://dbpedia.org/ontology/country"\  /\textgreater\newline
		\tab\textless owl:equivalentProperty\newline
		\tab\tab rdf:resource=``http://schema.org/Country"\  /\textgreater\newline
		\tab\textless owl:equivalentProperty\newline
		\tab\tab rdf:resource=``https://www.wikidata.org/wiki/Q6256"\  /\textgreater\newline
		\textless /owl:DatatypeProperty\textgreater
		\\ \hline
	\end{tabular}
	\caption{Propiedad pais}
	\label{propiedad-pais}
\end{table}

\begin{table}[!ht]
	\centering
	\begin{tabular}{|p{.25\textwidth}|p{.9\textwidth}|}
		\hline
		\multicolumn{2}{|l|}{Propiedad: \textbf{personas}}
		\\ \hline
		Rango:&
		http://www.w3.org/2001/XMLSchema\#nonNegativeInteger
		\\ \hline
		Propiedades \newline equivalentes:&
		\begin{itemize}
			\item \url{http://dbpedia.org/ontology/person}
			\item \url{http://schema.org/Person}
			\item \url{https://www.wikidata.org/wiki/Q215627}
		\end{itemize}
		\\ \hline
		Descripción:&
		\textless owl:DatatypeProperty rdf:about=``\#personas"\textgreater\newline 
		\tab\textless rdfs:label xml:lang=``es"\textgreater\newline
		\tab\tab Personas\newline
		\tab\textless /rdfs:label\textgreater\newline
		\tab\textless rdfs:range\newline
		\tab\tab rdf:resource=``\&xsd;nonNegativeInteger"\ /\textgreater\newline
		\tab\textless owl:equivalentProperty\newline
		\tab\tab rdf:resource=``http://dbpedia.org/ontology/person"\  /\textgreater\newline
		\tab\textless owl:equivalentProperty\newline
		\tab\tab rdf:resource=``http://schema.org/Person"\  /\textgreater\newline
		\tab\textless owl:equivalentProperty\newline
		\tab\tab rdf:resource=``https://www.wikidata.org/wiki/Q215627"\  /\textgreater\newline
		\textless /owl:DatatypeProperty\textgreater
		\\ \hline
	\end{tabular}
	\caption{Propiedad personas}
	\label{propiedad-personas}
\end{table}

\begin{table}[!ht]
	\centering
	\begin{tabular}{|p{.25\textwidth}|p{.9\textwidth}|}
		\hline
		\multicolumn{2}{|l|}{Propiedad: \textbf{planEstudios}}
		\\ \hline
		Dominio:&
		\begin{itemize}
			\item NumMedioCreditos
		\end{itemize}
		\\ \hline
		Rango:&
		http://www.w3.org/2001/XMLSchema\#string
		\\ \hline
		Descripción:&
		\textless owl:DatatypeProperty rdf:about=``\#planEstudios"\textgreater\newline 
		\tab\textless rdfs:label xml:lang=``es"\textgreater\newline
		\tab\tab Plan de estudios\newline
		\tab\textless /rdfs:label\textgreater\newline
		\tab\textless rdfs:range\newline
		\tab\tab rdf:resource=``\&xsd;string"\ /\textgreater\newline
		\tab\textless rdfs:domain\newline
		\tab\tab rdf:resource=``\#NumMedioCreditos"\ /\textgreater\newline
		\textless /owl:DatatypeProperty\textgreater
		\\ \hline
	\end{tabular}
	\caption{Propiedad planEstudios}
	\label{propiedad-planestudios}
\end{table}

\begin{table}[!ht]
	\centering
	\begin{tabular}{|p{.25\textwidth}|p{.9\textwidth}|}
		\hline
		\multicolumn{2}{|l|}{Propiedad: \textbf{plazasOfertadas}}
		\\ \hline
		Dominio:&
		\begin{itemize}
			\item DemandaAcademicaTitulacion
		\end{itemize}
		\\ \hline
		Rango:&
		http://www.w3.org/2001/XMLSchema\#nonNegativeInteger
		\\ \hline
		Descripción:&
		\textless owl:DatatypeProperty rdf:about=``\#plazasOfertadas"\textgreater\newline 
		\tab\textless rdfs:label xml:lang=``es"\textgreater\newline
		\tab\tab Plazas ofertadas\newline
		\tab\textless /rdfs:label\textgreater\newline
		\tab\textless rdfs:range\newline
		\tab\tab rdf:resource=``\&xsd;nonNegativeInteger"\ /\textgreater\newline
		\tab\textless rdfs:domain\newline
		\tab\tab rdf:resource=``\#DemandaAcademicaTitulacion"\ /\textgreater\newline
		\textless /owl:DatatypeProperty\textgreater
		\\ \hline
	\end{tabular}
	\caption{Propiedad plazasOfertadas}
	\label{propiedad-plazasOfertadas}
\end{table}

\begin{table}[!ht]
	\centering
	\begin{tabular}{|p{.25\textwidth}|p{.9\textwidth}|}
		\hline
		\multicolumn{2}{|l|}{Propiedad: \textbf{provincia}}
		\\ \hline
		Dominio:&
		\begin{itemize}
			\item OrigenProvincia
		\end{itemize}
		\\ \hline
		Rango:&
		http://www.w3.org/2001/XMLSchema\#string
		\\ \hline
		Propiedades \newline equivalentes:&
		\begin{itemize}
			\item \url{http://dbpedia.org/ontology/province}
			\item \url{http://schema.org/State}
			\item \url{https://www.wikidata.org/wiki/Q34876}
		\end{itemize}
		\\ \hline
		Descripción:&
		\textless owl:DatatypeProperty rdf:about=``\#provincia"\textgreater\newline 
		\tab\textless rdfs:label xml:lang=``es"\textgreater\newline
		\tab\tab Provincia\newline
		\tab\textless /rdfs:label\textgreater\newline
		\tab\textless rdfs:range\newline
		\tab\tab rdf:resource=``\&xsd;string"\ /\textgreater\newline
		\tab\textless rdfs:domain\newline
		\tab\tab rdf:resource=``\#OrigenProvincia"\ /\textgreater\newline
		\tab\textless owl:equivalentProperty\newline
		\tab\tab rdf:resource=``http://dbpedia.org/ontology/province"\  /\textgreater\newline
		\tab\textless owl:equivalentProperty\newline
		\tab\tab rdf:resource=``http://schema.org/State"\  /\textgreater\newline
		\tab\textless owl:equivalentProperty\newline
		\tab\tab rdf:resource=``https://www.wikidata.org/wiki/Q34876"\  /\textgreater\newline
		\textless /owl:DatatypeProperty\textgreater
		\\ \hline
	\end{tabular}
	\caption{Propiedad provincia}
	\label{propiedad-provincia}
\end{table}

\begin{table}[!ht]
	\centering
	\begin{tabular}{|p{.25\textwidth}|p{.9\textwidth}|}
		\hline
		\multicolumn{2}{|l|}{Propiedad: \textbf{ramaConocimiento}}
		\\ \hline
		Dominio:&
		\begin{itemize}
			\item MatriculasGrado
			\item MatriculasPosgrado
			\item NumMedioCreditos
			\item OfertaTitulacionDoctorado
			\item OfertaTitulacionGrado
			\item OfertaTitulacionMaster
		\end{itemize}
		\\ \hline
		Rango:&
		http://www.w3.org/2001/XMLSchema\#string
		\\ \hline
		Descripción:&
		\textless owl:DatatypeProperty rdf:about=``\#ramaConocimiento"\textgreater\newline 
		\tab\textless rdfs:label xml:lang=``es"\textgreater\newline
		\tab\tab Rama de conocimiento\newline
		\tab\textless /rdfs:label\textgreater\newline
		\tab\textless rdfs:range\newline
		\tab\tab rdf:resource=``\&xsd;string"\ /\textgreater\newline
		\tab\textless rdfs:domain\newline
		\tab\tab rdf:resource=``\#MatriculasGrado"\ /\textgreater\newline
		\tab\textless rdfs:domain\newline
		\tab\tab rdf:resource=``\#MatriculasPosgrado"\ /\textgreater\newline
		\tab\textless rdfs:domain\newline
		\tab\tab rdf:resource=``\#NumMedioCreditos"\ /\textgreater\newline
		\tab\textless rdfs:domain\newline
		\tab\tab rdf:resource=``\#OfertaTitulacionDoctorado"\ /\textgreater\newline
		\tab\textless rdfs:domain\newline
		\tab\tab rdf:resource=``\#OfertaTitulacionGrado"\ /\textgreater\newline
		\tab\textless rdfs:domain\newline
		\tab\tab rdf:resource=``\#OfertaTitulacionMaster"\ /\textgreater\newline
		\textless /owl:DatatypeProperty\textgreater
		\\ \hline
	\end{tabular}
	\caption{Propiedad ramaConocimiento}
	\label{propiedad-ramaconocimiento}
\end{table}

\begin{table}[!ht]
	\centering
	\begin{tabular}{|p{.25\textwidth}|p{.9\textwidth}|}
		\hline
		\multicolumn{2}{|l|}{Propiedad: \textbf{tasaAbandono}}
		\\ \hline
		Dominio:&
		\begin{itemize}
			\item TasasAcademicasTitulacion
		\end{itemize}
		\\ \hline
		Rango:&
		http://www.w3.org/2001/XMLSchema\#decimal
		\\ \hline
		Descripción:&
		\textless owl:DatatypeProperty rdf:about=``\#tasaAbandono"\textgreater\newline 
		\tab\textless rdfs:label xml:lang=``es"\textgreater\newline
		\tab\tab Tasa de abandono\newline
		\tab\textless /rdfs:label\textgreater\newline
		\tab\textless rdfs:range\newline
		\tab\tab rdf:resource=``\&xsd;decimal"\ /\textgreater\newline
		\tab\textless rdfs:domain\newline
		\tab\tab rdf:resource=``\#TasasAcademicasTitulacion"\ /\textgreater\newline
		\textless /owl:DatatypeProperty\textgreater
		\\ \hline
	\end{tabular}
	\caption{Propiedad tasaAbandono}
	\label{propiedad-tasaabandono}
\end{table}

\begin{table}[!ht]
	\centering
	\begin{tabular}{|p{.25\textwidth}|p{.9\textwidth}|}
		\hline
		\multicolumn{2}{|l|}{Propiedad: \textbf{tasaAbandonoInicial}}
		\\ \hline
		Dominio:&
		\begin{itemize}
			\item TasasAcademicasTitulacion
		\end{itemize}
		\\ \hline
		Rango:&
		http://www.w3.org/2001/XMLSchema\#decimal
		\\ \hline
		Descripción:&
		\textless owl:DatatypeProperty rdf:about=``\#tasaAbandonoInicial"\textgreater\newline 
		\tab\textless rdfs:label xml:lang=``es"\textgreater\newline
		\tab\tab Tasa de abandono inicial\newline
		\tab\textless /rdfs:label\textgreater\newline
		\tab\textless rdfs:range\newline
		\tab\tab rdf:resource=``\&xsd;decimal"\ /\textgreater\newline
		\tab\textless rdfs:domain\newline
		\tab\tab rdf:resource=``\#TasasAcademicasTitulacion"\ /\textgreater\newline
		\textless /owl:DatatypeProperty\textgreater
		\\ \hline
	\end{tabular}
	\caption{Propiedad tasaAbandonoInicial}
	\label{propiedad-tasaabandonoinicial}
\end{table}

\begin{table}[!ht]
	\centering
	\begin{tabular}{|p{.25\textwidth}|p{.9\textwidth}|}
		\hline
		\multicolumn{2}{|l|}{Propiedad: \textbf{tasaEficiencia}}
		\\ \hline
		Dominio:&
		\begin{itemize}
			\item TasasAcademicasTitulacion
		\end{itemize}
		\\ \hline
		Rango:&
		http://www.w3.org/2001/XMLSchema\#decimal
		\\ \hline
		Descripción:&
		\textless owl:DatatypeProperty rdf:about=``\#tasaEficiencia"\textgreater\newline 
		\tab\textless rdfs:label xml:lang=``es"\textgreater\newline
		\tab\tab Tasa de eficiencia\newline
		\tab\textless /rdfs:label\textgreater\newline
		\tab\textless rdfs:range\newline
		\tab\tab rdf:resource=``\&xsd;decimal"\ /\textgreater\newline
		\tab\textless rdfs:domain\newline
		\tab\tab rdf:resource=``\#TasasAcademicasTitulacion"\ /\textgreater\newline
		\textless /owl:DatatypeProperty\textgreater
		\\ \hline
	\end{tabular}
	\caption{Propiedad tasaEficiencia}
	\label{propiedad-tasaeficiencia}
\end{table}

\begin{table}[!ht]
	\centering
	\begin{tabular}{|p{.25\textwidth}|p{.9\textwidth}|}
		\hline
		\multicolumn{2}{|l|}{Propiedad: \textbf{tasaGraduacion}}
		\\ \hline
		Dominio:&
		\begin{itemize}
			\item TasasAcademicasTitulacion
		\end{itemize}
		\\ \hline
		Rango:&
		http://www.w3.org/2001/XMLSchema\#decimal
		\\ \hline
		Descripción:&
		\textless owl:DatatypeProperty rdf:about=``\#tasaGraduacion"\textgreater\newline 
		\tab\textless rdfs:label xml:lang=``es"\textgreater\newline
		\tab\tab Tasa de graduación\newline
		\tab\textless /rdfs:label\textgreater\newline
		\tab\textless rdfs:range\newline
		\tab\tab rdf:resource=``\&xsd;decimal"\ /\textgreater\newline
		\tab\textless rdfs:domain\newline
		\tab\tab rdf:resource=``\#TasasAcademicasTitulacion"\ /\textgreater\newline
		\textless /owl:DatatypeProperty\textgreater
		\\ \hline
	\end{tabular}
	\caption{Propiedad tasaGraduacion}
	\label{propiedad-tasagraduacion}
\end{table}

\begin{table}[!ht]
	\centering
	\begin{tabular}{|p{.25\textwidth}|p{.9\textwidth}|}
		\hline
		\multicolumn{2}{|l|}{Propiedad: \textbf{tasaRendimiento}}
		\\ \hline
		Dominio:&
		\begin{itemize}
			\item TasasAcademicasTitulacion
		\end{itemize}
		\\ \hline
		Rango:&
		http://www.w3.org/2001/XMLSchema\#decimal
		\\ \hline
		Descripción:&
		\textless owl:DatatypeProperty rdf:about=``\#tasaRendimiento"\textgreater\newline 
		\tab\textless rdfs:label xml:lang=``es"\textgreater\newline
		\tab\tab Tasa de rendimiento\newline
		\tab\textless /rdfs:label\textgreater\newline
		\tab\textless rdfs:range\newline
		\tab\tab rdf:resource=``\&xsd;decimal"\ /\textgreater\newline
		\tab\textless rdfs:domain\newline
		\tab\tab rdf:resource=``\#TasasAcademicasTitulacion"\ /\textgreater\newline
		\textless /owl:DatatypeProperty\textgreater
		\\ \hline
	\end{tabular}
	\caption{Propiedad tasaRendimiento}
	\label{propiedad-tasarendimiento}
\end{table}

\begin{table}[!ht]
	\centering
	\begin{tabular}{|p{.25\textwidth}|p{.9\textwidth}|}
		\hline
		\multicolumn{2}{|l|}{Propiedad: \textbf{tipoProcedimiento}}
		\\ \hline
		Dominio:&
		\begin{itemize}
			\item DemandaAcademicaAcceso
		\end{itemize}
		\\ \hline
		Rango:&
		http://www.w3.org/2001/XMLSchema\#string
		\\ \hline
		Descripción:&
		\textless owl:DatatypeProperty rdf:about=``\#tipoProcedimiento"\textgreater\newline 
		\tab\textless rdfs:label xml:lang=``es"\textgreater\newline
		\tab\tab Tipo de procedimiento\newline
		\tab\textless /rdfs:label\textgreater\newline
		\tab\textless rdfs:subPropertyOf\newline
		\tab\tab rdf:resource=``\#tipoProcedimiento"\ /\textgreater\newline
		\tab\textless rdfs:range\newline
		\tab\tab rdf:resource=``\&xsd;string"\ /\textgreater\newline
		\tab\textless rdfs:domain\newline
		\tab\tab rdf:resource=``\#DemandaAcademicaAcceso"\ /\textgreater\newline
		\textless /owl:DatatypeProperty\textgreater
		\\ \hline
	\end{tabular}
	\caption{Propiedad tipoProcedimiento}
	\label{propiedad-tipoprocedimiento}
\end{table}

\begin{table}[!ht]
	\centering
	\begin{tabular}{|p{.25\textwidth}|p{.9\textwidth}|}
		\hline
		\multicolumn{2}{|l|}{Propiedad: \textbf{titulacion}}
		\\ \hline
		Rango:&
		http://www.w3.org/2001/XMLSchema\#string
		\\ \hline
		Subpropiedad de:&
		ramaConocimiento
		\\ \hline
		Propiedades \newline equivalentes:&
		\begin{itemize}
			\item \url{http://dbpedia.org/page/Academic_degree}
			\item \url{https://www.wikidata.org/wiki/Q189533}
		\end{itemize}
		\\ \hline
		Descripción:&
		\textless owl:DatatypeProperty rdf:about=``\#titulacion"\textgreater\newline
		\tab\textless rdfs:label xml:lang=``es"\textgreater\newline
		\tab\tab Titulacion\newline
		\tab\textless /rdfs:label\textgreater\newline
		\tab\textless rdfs:subPropertyOf\newline
		\tab\tab rdf:resource=``\#ramaConocimiento"\ /\textgreater\newline
		\tab\textless rdfs:range\newline
		\tab\tab rdf:resource=``\&xsd;string"\ /\textgreater\newline
		\tab\textless owl:equivalentProperty\newline
		\tab\tab rdf:resource=``http://dbpedia.org/page/Academic\_degree"\  /\textgreater\newline
		\tab\textless owl:equivalentProperty\newline
		\tab\tab rdf:resource=``https://www.wikidata.org/wiki/Q189533"\  /\textgreater\newline
		\textless /owl:DatatypeProperty\textgreater
		\\ \hline
	\end{tabular}
	\caption{Propiedad titulacion}
	\label{propiedad-titulacion}
\end{table}

\begin{table}[!ht]
	\centering
	\begin{tabular}{|p{.25\textwidth}|p{.9\textwidth}|}
		\hline
		\multicolumn{2}{|l|}{Propiedad: \textbf{titulados}}
		\\ \hline
		Dominio:&
		\begin{itemize}
			\item DemandaAcademicaTitulacion
		\end{itemize}
		\\ \hline
		Rango:&
		http://www.w3.org/2001/XMLSchema\#nonNegativeInteger
		\\ \hline
		Descripción:&
		\textless owl:DatatypeProperty rdf:about=``\#titulados"\textgreater\newline 
		\tab\textless rdfs:label xml:lang=``es"\textgreater\newline
		\tab\tab Titulados\newline
		\tab\textless /rdfs:label\textgreater\newline
		\tab\textless rdfs:range\newline
		\tab\tab rdf:resource=``\&xsd;nonNegativeInteger"\ /\textgreater\newline
		\tab\textless rdfs:domain\newline
		\tab\tab rdf:resource=``\#DemandaAcademicaTitulacion"\ /\textgreater\newline
		\textless /owl:DatatypeProperty\textgreater
		\\ \hline
	\end{tabular}
	\caption{Propiedad titulados}
	\label{propiedad-titulados}
\end{table}

\begin{table}[!ht]
	\centering
	\begin{tabular}{|p{.25\textwidth}|p{.9\textwidth}|}
		\hline
		\multicolumn{2}{|l|}{Propiedad: \textbf{universidad}}
		\\ \hline
		Propiedades \newline equivalentes:&
		\begin{itemize}
			\item \url{http://dbpedia.org/ontology/university}
			\item \url{http://schema.org/CollegeOrUniversity}
			\item \url{https://www.wikidata.org/wiki/Q3918}
		\end{itemize}
		\\ \hline
		Rango:&
		http://www.w3.org/2001/XMLSchema\#string
		\\ \hline
		Descripción:&
		\textless owl:DatatypeProperty rdf:about=``\#universidad"\textgreater\newline 
		\tab\textless rdfs:label xml:lang=``es"\textgreater\newline
		\tab\tab Titulados\newline
		\tab\textless /rdfs:label\textgreater\newline
		\tab\textless rdfs:range\newline
		\tab\tab rdf:resource=``\&xsd;nonNegativeInteger"\ /\textgreater\newline
		\tab\textless owl:equivalentProperty\newline
		\tab\tab rdf:resource=``http://dbpedia.org/ontology/university"\ /\textgreater\newline
		\tab\textless owl:equivalentProperty\newline
		\tab\tab rdf:resource=``http://schema.org/CollegeOrUniversity"\ /\textgreater\newline
		\tab\textless owl:equivalentProperty\newline
		\tab\tab rdf:resource=``https://www.wikidata.org/wiki/Q3918"\ /\textgreater\newline
		\textless /owl:DatatypeProperty\textgreater
		\\ \hline
	\end{tabular}
	\caption{Propiedad titulados}
	\label{propiedad-titulados}
\end{table}